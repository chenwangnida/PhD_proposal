\chapter{Literature Review}\label{C:review}
In this chapter, first, we introduce the background knowledge of Web service in Subsection \ref{service} and service composition in Subsection \ref{servicecomposition}. Second, EC techniques are also briefly introduced in Subsection \ref{ec}. Followed that Section \ref{related} reviews the single-objective service composition using EC and non-EC based techniques in Subsection \ref{singleobjective}.  Subsection  \ref{multiobjective} reviews existing works on multi-objective approaches, many-objective approaches, and preference articulation techniques for multi-objective approaches. Dynamic Web service composition is covered in Subsection \ref{dynamicserivce}. Subsection \ref{Semantic}  discussed service composition supporting preconditions and effects. Lastly, Subsection \ref{summary} summarizes some key reviews and limitations in the literature review.
\section{Background}\label{background}
\subsection{Web Service}\label{service}
%We first introduce the concept of Web services and a formal model from \cite{agarwal2010d5}, where both the functional and non-functional attributes are captured uniformly. Further, A Labelled Transition System \cite{agarwal2009making} is addressed with its abstract and updated model for demonstrating the behaviors of Web services in Subsection \ref{functional}, which emphasizes on side of the functional attributes. After demonstrating these models, we will discuss about the nonfunctional properties of Web services in Subsection \ref{nonfunctional}. Apart from these, we will discuss three main service discover mechanisms in \ref{servicediscovery}.

Web services are self-describing, self-contained software modules available over the internet \cite{erl2004service}. As demonstrated in \cite{erl2004service}, Web services are loosely coupled software modules, where service interfaces enable applications to work cooperatively and defined in a neutral way regardless the platforms, operation systems and programming languages. Besides that, Web services are distributed via internet by internet protocol, such as HTTP, and are described by standard description languages, for example, WSDL. These description languages are mainly used to describe functional properties of Web services in terms of service inputs and service outputs, which satisfy users' functional requirements and provide mechanisms to allow users to search desired service. 

Web services are classified into two groups based on their functionalities:  \emph{information-providing services} and \emph{world-altering services} \cite{mcilraith2001semantic}. The first type of services expect some data returned by giving inputs or nothing. For example, an air velocity transducer Web service reads the wind speed and returns the velocity at the time. This service does not require any inputs. On the other side, a city weather Web service requires a city name and returns weather information for that city. Information-providing services do not produce any side effect to the world. The functionalities of these services are only inputs and outputs. The second type of services not only provide data information but also alters the status of the world by producing side effects. For example, a PayPal service will cause a deduction in the balance of users' bank account. \emph{In this proposal, we mainly focus the first type of services for first three objective. Later on, an extensive study is optionally carried for the second type of services}

As demonstrated above, the functional attributes determine what service really does, while the non-functional attributes often refers to some quality criteria, which is considered for raking services \cite{agarwal2009making}. When several services provide the same functionality, users would prefer a service with a lower cost with the same functionality. Herein we will demonstrate both the functional and non-functional properties in following subsections.

%\emph{A Formal Model of Web Service} is demonstrated here: given a finite set $\wp$ of property types of Web services and a finite set $\vartheta$ of values sets. Each property type is associated with a value set. We view a Web service as a finite set $Q$ of property instances with each property instance $q \in Q$ being of a property type$t(q) \in \wp$ that is associated with a value $v_q \in \vartheta_{t(q)}$. See Fig. \ref{fig:ws}

%\begin{figure}
%\centerline{
%\fbox{
%\includegraphics[width=8cm]{ws.pdf}
%}}
%\caption{Property-Based Web Service Formal Model \cite{agarwal2010d5}.}
%\label{fig:ws}
%\end{figure}

\subsubsection{Functional Properties of Web services}\label{functional}

%\emph{A Functionality Model} is demonstrated for describing all the functionalities of  Web service. This mode is represented as a Labelled Transition System (LTS): $L = (S,T,\rightarrow)$ comprises a set $S$ of states, a set $T$ of transition labels and a labeled transition relation $\rightarrow \subseteq S \times T \times S$. This transition system is established to present the actual behavior of Web services, which consists of a series of states. See Fig. \ref{fig:lts}.

%\begin{figure}
%\centerline{
%\fbox{
%\includegraphics[width=14cm]{LTS.pdf}
%}}
%\caption{ The functionality of a Web Service \cite{agarwal2010d5}.}
%\label{fig:lts}
%\end{figure}
%
%\begin{itemize}
%\item $S_i$ start state includes knowledge available to Web service before the Web service is invoked by inputs; 
%\item $S_w$ state includes the inputs additional to the knowledge in 1; 
%\item A series of state $S_1$ to $S_n$ proceed with corresponding actions $f_n$ that only occurred if each related condition $[w_i]$ is approved to be true. 
%\item $S_o$ state contains all the outputs and all the changes performed in the knowledge base. 
%\item $S_e$ is the end state, which is equivalent to $S_o$, since the knowledge base is not changed.
%\end{itemize}

%The first functionality model presented above is not completely demonstrated for the internal actions, as the services provider does not want reveal all the internal functions, and it is not feasible to list a global set of property name. Therefore, \emph{An Abstract Functionality Model} is modeled by eliminating all the intermediate properties. In the abstract model of a Web service, the functional properties of the Web service could be identified as inputs $i$, pre-state $S_i$, outputs $o$ and post-state $S_o$. These four properties are mapped to a set of input $I$, preconditions $\phi$, a set of outputs $O$ and effects $\varphi$ respectively in third updated functionality model demonstrated below.
The operational characteristics of Web services are related to the functional properties, which demonstrate the behaviors of Web services, i.e., how services can be successfully invoked and what information are returned after execution. In other words, a set of inputs $I$ is required by a service and a set of output $O$ is returned by a service. Sometimes, preconditions $\phi$ are required by some services and effects $\varphi$ are produced along with outputs. For \emph{semantic Web services}, ontology is employed to semantically describe the properties (i.e., inputs and outputs) of Web services, where a knowledge base is created to better enable the interoperability of the functional properties. Sometimes, the preconditions $\phi$ must be hold in the knowledge base before service execution, and they must remain consistent while passing the input $I$. Side effects are also created as effects $\varphi$ along with outputs $O$. All functional properties of Web services are demonstrated in Fig. \ref{fig:ws}.


\begin{figure}
\centerline{
\fbox{
\includegraphics[width=8cm]{ws.pdf}
}}
\caption{Functional Properties of a Web Service.}
\label{fig:ws}
\end{figure}


%Apart from that, the precondition $\phi$ must be hold in the knowledge base before service is invoked by passing the input $I$. To enable the interoperability of the functional properties, ontology reasoner is employed to reason about the properties of Web services. To distinguish the changes between before and after service execution, these changes are modeled as property instances. For example, inputs and outputs assigned as variable names and further referenced in preconditions and effects, which can be distinguished as different instances from the knowledge base. In particular, preconditions is assigned to the description of requirements of inputs using logic formula. Herein the description of the formula considers the following cases that are demonstrated using Planning Domain Description Language (PDDL \cite{fox2003pddl2}):
%\begin{itemize}
%\item Conditions on the type of inputs, e.g., the payment of an online shopping Website is made by Visa or Cash: $\phi:(Format(payment)=Visa \cup Format(payment)=Cash)$.
%\item Relationships among inputs, e.g.,  authoried users are required for an online shopping: $\phi:(Authoried ? Useraccount)$.
%\item Conditions on the value of inputs, e.g., saving account balance has more than 100 dollars: $\phi: (\geq (amounts, saving account), 100)$
%\end{itemize}
%These preconditions must be hold in the state consistently while inputs are being passed to services. Similarly, the effects is restricted to the description of constraints on returned outputs, relationships between inputs and outputs, and changes caused by the service in the knowledge bases.

\subsubsection{Nonfunctional Properties of Web services}\label{nonfunctional}
Apart from the functional properties of Web services discussed above, the non-functional properties of Web services play an important part in composing services. One nonfunctional property often refers to QoS. For example, customers prefer the lowest execution cost and the highest response time and reliability. According to \cite{zeng2003quality}, four most often considered QoS parameters are as follows:
\begin{itemize}
\item \textit{Response time} ($T$) measures the expected delay in seconds between the moment when a request is sent and the moment when the results are received.
\item \textit{Cost} ($C$) is the amount of money that a service requester has to pay for executing the Web service
\item \textit{Reliability} ($R$) is the probability that a request is correctly responded within the maximum expected time frame.
\item \textit{Availability} ($A$) is the probability that a Web service is accessible.
\end{itemize}

Service level agreement (SLA) is also important as another nonfunctional properties of Web services. It is a declarative contract, which clarifies a QoS level that is agreed by service providers and service users \cite{zhao2015toward}. For example,  an SLA is clarified as end-to-end QoS constraints on a business process,  which presents how a certain business goal is achieved \cite{wada2008multiobjective}.

\subsubsection{Web Service Discovery}\label{servicediscovery}
To generate a service composition solution, Web service must provide mechanisms for discoverying required services. Therefore, service discovery becomes a fundamental technique to be considered in all service composition approaches. \cite{agarwal2009d5} discussed three mechanisms of semantic Web service discovery: classification-based approach, functionality-based approach and hybrid approach. Those service discovery techniques are further demonstrated below.

The first service discovery technique makes use of the classes provided by service semantic annotation in WSMO-Lite language. Therefore, service requesters can use class names to express a goal, which is a straightforward discovery from a set of classes. However, classes without clear meaning definition could lead to the issue of incomprehensibility of Web service discovery. For example, several classes may declared in either different terms for the describing the same content, or same terms for describing different content.

The second service discovery technique  does not take classes into account, but consider functional properties of Web service. In particular, a desired functionality description is defined. A discovery algorithm must be developed to handle a match for different input, output, precondition and effects with associated concepts and relations in the provided domain knowledge base. The key idea of the matchmaking is to check whether services accept all the desired inputs provided by users and whether the desired outputs is delivered by services. In addition, this discovery technique also checks for the satisfiability of implications that actual precondition and actual effects must imply the desired precondition and desired effects respectively. The strength of the second is that it potentially meet the demands of all the comprehensible discovery, while the weakness is a lack of efficiency and scalability. 

The third service discovery technique is based on a hybridization of classification and functionality-based discovery. Classification hierarchy is proposed to achieve automatic semantic reasoning in hierarchical functionality. For example, a functionality class is associated with super classes and sub classes for more general functionality class and more specific functionality class respectively. However, to make a consistency of classification hierarchy, the inputs, outputs, precondition and effects of a functionality class must satisfy the conditions that contains all the inputs, outputs, precondition and effects of all the classes it is subsumed by. The advantage of this approach is to achieve better performance combining strengths of the previous two pure classification-based and functionality-based approaches. While the classification hierarchy needs to be kept consistent when a new Web service is published or updated.​

As discussed above, the first and third approach is considered either less effective or demands to build up a consistent ontology for classes and their functional properties. It is not the focus of our research for building up any ontology to support classes or properties.  \emph{In the proposal, we use the second service discovery technique for meeting a comprehensible discovery. That is, ontology reasoning is utilized to approach the semantic matchmaking as a fundamental part of service composition algorithm.}



\subsection{Web Service Composition}\label{servicecomposition}

Since an atomic Web service could not satisfy or fully satisfy users' complex functional requirements, Web service composition is approached by composing Web services together to meet the requirements. Manual service composition is very time-consuming and less productive. Therefore, many approaches have been developed to achieve semi-automated or fully automated service composition. The \emph{semi-automated service composition} is inspired by the business process that requires a prior knowledge to build up abstract workflows \cite{moghaddam2014service}. These workflows consist of abstract services, each of which is defined by input-output pairs. The details of these steps are further discussed in Subsection \ref{lifecycle}. On the other hand, when we are composing services, the interoperability of services is very important in both semi-automated and fully automated Web service composition. In particular, several problems are simultaneously taken into account. That is I/O matchmaking (i.e., a mechanism for ensuring the interoperability ), service discovery for finding services with required functionalities, and service selection for optimizing the quality of service composition. 


\subsubsection{Semi-Automated Web Service Composition}\label{lifecycle}
We firstly demonstrate semi-automated Web service composition, where several steps are shown in Fig. \ref{fig:lifecycle}. The detail of the service composition lifecycle is discussed as follows:

\begin{figure}
\centerline{
\fbox{
\includegraphics[width=14cm]{compositionLifecycle.pdf}
}}
\caption{Semi-automated Web service composition lifecycle \cite{moghaddam2014service}.}
\label{fig:lifecycle}
\end{figure}

\begin{enumerate}
 \item \textit{Service request:} The first step in service composition is to collect users' requirements for composition goal that comprises of the functional (i.e., correct data flow ) and non-functional side (i.e., QoS). This step is achieved by building up an abstract workflow including a series of tasks with clearly defined functionalities. Those tasks could be completed by selecting proper concrete services to reach desired QoS. 
 \item \textit{Service discovery:} Once the goal is clearly specified, concrete Web services will be selected for each task regarding its functional requirement. Often, more than one concrete Web service is likely to be found to match one service discovery task. However, those matched Web services are always different in QoS.
 \item \textit{Service selection:} At this stage, many techniques have been studied to select Web services to best match each task for the satisfying functional requirement of each task and overall QoS of the business process. Therefore, a plan of service composition is created ahead of execution.
 \item \textit{Service execution:} the process instance is monitored for any changes or services failures during service execution. In this stage, some actions are to be taken for adapting the changes.
\end{enumerate}

\subsubsection{Fully Automated Web Service Composition}\label{fully}

There is a distinction between semi-automated and fully automated approaches. During the service request of semi-automated approach, the abstract workflow is already provided. However, there is no abstract workflow provided for fully automated service composition. The service request for fully automated service composition approaches only consists of task inputs and task outputs, which specify the information gathered from users and information expected to be returned respectively. Often, \emph{fully automated service composition} rely on some algorithms (e.g., Graphplan algorithm \cite{blum1997fast}) to achieve service composition, where service workflow is gradually built up along with service discovery and service selection. 

Fig. \ref{fig:wsc_example} shows a popular Web service composition example from a travel domain. In this scenario, the agency provides customers a serial of services to satisfy their complex requirements. These services include flight bookings, accommodation reservations, bus services and map generations. In Fig. \ref{fig:wsc_example}, the task inputs (i.e., $TravelDepartureDate$, $TravelReturnDate$, $HomeCity$ and $ConferenceCity$ ) are gathered from customers, and task outputs (i.e., $BusTicket$, $FlightTicket$, $HotelReservation$ and $StreepMap$) are expected to be returned. The component services are gradually discovered and selected to construct a workflow from $TaskInput$ node to $TaskOut$ node, if the inputs of selected services are fulfilled with $TaskInput$ or the outputs of previously selected services. Therefore, all the inputs of all the component services must be satisfied. In this example, we begin by executing the FlightBooking Service and GenerateMap Service, the FilightInformation service books the flights and determines a $arrivalDate$. Then, we use the $arrivalDate$ together with the other given data, we can book the hotel and bus, and generate Map for conference city. Together, these four services produce all required outputs for customers. 

\emph{In our thesis, I concentrate on developing methods for fully automated service composition, where service discovery and service selection are considered as interrelated tasks that are interleaved with the composition algorithm}. 

\begin{figure}
\centerline{
\fbox{
\includegraphics[width=14cm]{wsc_example.pdf}
}}
\caption{An example of service composition for a travel agency.}
\label{fig:wsc_example}
\end{figure}

\subsubsection{Functional Properties of Web Service Composition}
From the example demonstrated above, two characteristics are addressed for the functional properties of Web service composition. One is that all the inputs of component services must be matched by given task inputs or produced outputs, so that service composition can be successfully executed. Another is that task output must be a subset of all the outputs of component Web services, so that a service composition goal is achieved. 


\subsubsection{Semantic Web Service Composition}

Semantic Web service composition is achieved by composing semantic Web services through reasoning about their semantically described properties. Often, the given task inputs or produced outputs of component services often do not perfectly match the required services' inputs. For example, one component service from previous service composition example is pick up for further demonstration. In Fig. \ref{fig:sm_example}, the generateMap service requires inputs $MappedLocation$ and produces output $StreetMap$. Given that $instance-of (ConferenceCity$, $City)$, $instance-of (MappedLocation, Location)$ and $City \sqsubseteq Location$. Based on the ontology-based description, $ConferenceCity$ is less preferable compared to $MappedLocation$. Therefore, quality of semantic matchmaking between given information $ConferenceCity$ and required input $MappedLocation$ is relatively lower. 

\begin{figure}
\centerline{
\fbox{
\includegraphics[width=14cm]{sm_example.pdf}
}}
\caption{An example for demonstrating semantic matchmaking quality.}
\label{fig:sm_example}
\end{figure}

In semantic web service composition, substantial works \cite{bansal2016generalized,lecue2009optimizing, lecue2007making, lecue2006formal, rao2005semantic} utilize description logic (DL) reasoning between input and output parameters of Web services for matchmaking, where different matchmaking types are considered for the matchmaking. They usually result in \emph{different matchmaking quality}. Therefore, exploring an effective mechanism to measure the quality of semantic matchmaking for semantic service composition is one of crucial tasks in our research. 

Firstly, we demonstrate different \emph{Matchmaking Types} discussed above here. Given two concepts $a, b$ in ontology $\mathcal{O}$, four commonly used matchmaking types are often used to describe the level of a match between outputs and inputs \cite{paolucci2002semantic}: 
\begin{itemize}
\item $exact$ returned, if $a$ and $b$ are equivalent ($a \equiv b$), 
\item $plugin$ returned, if $a$ is a sub-concept of $b$ ($a \sqsubseteq b$),
\item $subsume$ returned, if $a$ is a super-concept of $b$ ($a \sqsupseteq b$), 
\item $fail$ returned, if none of previous matchmaking types is returned. 
\end{itemize}

%Often, the similarity of two instances of two knowledge representations encoded in the same ontology is also
Secondly, to our best knowledge, three methods  \cite{lecue2007making, pop2009immune,shet2012new} are utilized for measuring the quality of semantic matchmaking in the domain of service composition. We will demonstrate these three methods below.

The first method measures the quality of matchmaking regarding the different matchmaking types. In \cite{lecue2007making}, the quality of semantic matchmaking is measured by the two quality criterion, matchmaking types and common description rate. They additionally consider $interaction$ matchmaking type ($a \sqcap b$), i.e., if the intersection of $a$ and $b$ is satisfiable. In their work, a causal link \begin{math} sl_{i,j} \stackrel{.}{=} \langle S_i, Sim_{T}(Out\_s_i,In\_s_j),S_j  \rangle \end{math} is created between a input and a output. In particular, both $exact$ match and $plugin$ match are presented as robust causal links, while both $subsume$ match and $intersection$ match are presented as valid casual links. However, valid casual links are not specific enough to be utilized as the input of another Web service. Thus the output requires Extra Description, denoted as \begin{math} In\_s_x \setminus Out\_s_y \end{math}, to enable proper service composition using Eq. \ref{equation2}. As a result, $subsume$ and $intersection$ are transferred to be $exact$ and $plugin$ respectively to formulate a robust link, so common description rate, \begin{math} q_{cd}(sl_{i,j}) \end{math}is calculated based on Extra Description and Least common subsume, denoted as \begin{math} lcs (In\_s_x, Out\_s_y) \end{math}, using Eq. \ref{equation2}..

\begin{equation}
In\_s_x \setminus Out\_s_y \stackrel{.}{=} \underset {\preceq d}{min} \{ B|B\sqcap  Out\_s_y \equiv In\_s_x  \} , since \  Out\_s_y \sqsupseteq In\_s_x
 \label{equation2}
\end{equation}


\begin{equation}
q_{cd}(sl_{i,j}) = \frac{lcs (In\_s_x, Out\_s_y)} {In\_s_x \setminus Out\_s_y + lcs (In\_s_x, Out\_s_y)}
 \label{equation3}
\end{equation}

The second method for measuring the quality of semantic matchmaking utilizes a similarity measurement from information retrieval. In \cite{pop2009immune}, the similarity is calculated by the average value of $F\_Measure(S_i.out_k, S_j.in_k)$. $F\_Measure(S_i.out_k, S_j.in_k)$ measures the similarity of two matched output $S_i.out_k$ and input $S_j.in_k$, and it is calculated based precision and recall between the provided output and required input.


The third method for measuring the quality of semantic matchmaking utilize semantic similarity in \cite{shet2012new}. For concepts $a, b$ in $\mathcal{O}$, semantic similarity, denoted as $sim(a, b)$, is calculated based on an edge counting method in a taxonomy like WorldNet or Ontology using Eq. (\ref{eq_sim}) \cite{shet2012new}. This method has the advantages of a simple calculation and a good performance . In Eq. (\ref{eq_sim}), $N_a$, $N_b$ and $N_c$ measure the distances from concept $a$, concept $b$, and the closest common ancestor $c$ of $a$ and $b$ to the top concept of the ontology $\mathcal{O}$, respectively.  $L$ is the shortest distance between the two concepts, $a$ and $b$, while $D$ is the depth of ontology tree. Also, $\lambda$ equals 1 for neighborhood concepts or 0 for concepts from same hierarchy.

\begin{equation}
sim(a, b){=} \frac{2N_c \cdot e^{-\lambda L/D} }{N_{a}+N_{b}}
\label{eq_sim}
\end{equation}
\noindent For our purposes, $\lambda$ can be set to 0 as we do not measure the similarities of neighborhood concepts, the matching type not considered in this paper. 

\emph{In this paper we are only interested in robust compositions, where only $exact$ and $plugin$ matches are considered, and we suggest to consider the semantic similarity of concepts when comparing different $plugin$ matches}. As argued in \cite{lecue2009optimizing} $plugin$ matches are less preferable than $exact$ matches due to the overheads associated with data processing


\subsubsection{Nonfunctional Properties of Web Service Composition}
The nonfunctional properties of Web service composition is determined by all the QoS of involved concrete Web services in the solutions. The aggregation value of QoS attributes for Web services composition varies with respect to different constructs, which reflects how services associated with each other in a service composition \cite{zeng2003quality}.
\begin{itemize}

\item \emph{Sequence construct}: service composition executes each atomic service associated with a sequence construct in a definite sequence order. The aggregated time ($T$) and execution cost ($C$) is as a sum of time and cost of Web services involved respectively. The aggregated availability and reliability in a sequence construct are calculated by multiplying the availability and reliability of each component Web service according to the probability theory. This construct is shown in Fig. \ref{sequence}.
\item \emph{Parallel construct}: Web services in a parallel construct are executed concurrently. The aggregated execution cost, availability and reliability are calculated in the same way as these in the sequence construct while the aggregated time ($T$) is determined by the most time-consuming path in the composite flow of the solution. This construct is presented in Fig. \ref{parallel}.
\item \emph{Choice construct}: Only one service path is executed in a choice construct depending on the satisfaction of the conditions on each path. In Fig. \ref{choice}, assuming the choice construct has $n$ branches, $p_1,\ldots, p_n$ with  $\sum\limits^d_{k=1}p_k=1$ denote the probabilities of the different branches being selected. For example, The aggregated total cost $C$  is the multiplication of the cost of each branching service and the possibility $p$ of the branch.
\item \emph{Loop construct}: Web services in a loop construct are executed repeatedly until a certain condition is satisfied. In Fig. \ref{loop}, assuming the average number of iterations is $\ell$, and $t$, $c$, $r$, and $a$ are corresponding aggregated value of a composite service. Therefore, For a loop construct, aggregated response time ($T$) and execution cost ($C$) are $p_n \cdot t$ and $p_n \cdot c$ respectively while aggregated availability $A$ and reliability $R$ are the $\ell^{th}$ power of the value of one iteration, i.e., $A=a^\ell$ and $R=r^\ell$.
\end{itemize}


\begin{figure}[h]
\centerline{
\fbox{
\begin{tabular}{p{0.8\linewidth}}
\space\hfill\includegraphics[width=1.8in]{sequence.pdf}\hfill\space\\[0.2cm]
$T=\sum\limits^m_{n=1}t_n$ \hfill $C=\sum\limits^m_{n=1}c_n$ \hfill
$A=\prod\limits^m_{n=1}a_n$ \hfill $R=\prod\limits^m_{n=1}r_n$
\end{tabular}}}
\caption{Sequence construct and calculation of its QoS
\cite{yu2013adaptive}.}
\label{sequence}
%\end{figure}
\vspace{0.3cm}
%\begin{figure}
\centerline{
\fbox{
\begin{tabular}{p{0.8\linewidth}}
\space\hfill\includegraphics[width=1.4in]{parallel.pdf}\hfill\space\\[0.2cm]
\space\hfill$T=MAX\{t_n|n\in\{1,\ldots,m\}\}$\hfill\space\\[0.2cm]
$C=\sum\limits^m_{n=1}c_n$ \hfill $A=\prod\limits^m_{n=1}a_n$ \hfill
$R=\prod\limits^m_{n=1}r_n$
\end{tabular}}}
\caption{Parallel construct and calculation of its QoS
\cite{yu2013adaptive}.}
\label{parallel}
\centerline{
\fbox{
\begin{tabular}{p{0.8\linewidth}}
\space\hfill\includegraphics[width=1.4in]{choice.pdf}\hfill\space\\[0.2cm]
$T=\sum\limits^m_{n=1}p_n \cdot t_n$ \hfill $C=\sum\limits^m_{n=1}p_n \cdot c_n$ \hfill
$A=\prod\limits^m_{n=1}p_n \cdot a_n$ \hfill $R=\prod\limits^m_{n=1}p_n \cdot r_n$
\end{tabular}}}
\caption{Choice construct and calculation of its QoS
\cite{yu2013adaptive}.}
\label{choice}
\centerline{
\fbox{
\begin{tabular}{p{0.8\linewidth}}
\space\hfill\includegraphics[width=1.4in]{loop.pdf}\hfill\space\\[0.2cm]
$T= \ell \cdot t$ \hfill $C=\ell \cdot c$ \hfill
$A=a^\ell$ \hfill $R=r^\ell$
\end{tabular}}}
\caption{Loop construct and calculation of its QoS
\cite{yu2013adaptive}.}
\label{loop}
\end{figure}

\subsection{Evolutionary Computation Techniques Overview}\label{ec}

Evolution Computing (EC) techniques are founded based on the principles of Darwin natural selection. The nature evolution and selection of individuals in a population are automated simulated in EC techniques. In particular, a population of individuals is initialized for directly or indirectly presenting the solutions. Those individual candidates are evolved and evaluated using a fitness function to evaluate the degree of how good (or bad) of each individual. Therefore, it is possible to reach solution with a near-optimal fitness value. EC techniques have been shown its promise in solving combinatorial optimization problems \cite{back1997evolutionary}. This is due to its flexibility in encoding the problems for the representation and its good performance in many problem domains. In particular, they have effective methods to manage the constraints in the combinatorial optimization problems, i.e., coding, penalty functions, repair algorithms, indirect methods of representation and multi-objective optimization \cite{fleming2002evolutionary}. In the context of service composition. Many EC techniques (e.g., Genetic Algorithms (GA) \cite{whitley1994genetic}, Genetic Programming (GP) \cite{koza1992genetic}, Particle Swarm Optimization (PSO) \cite{kennedy1995particle}, and Clonal Section Algorithm \cite{de2002learning}) have been utilized for handling optimization problems for Web service composition. These techniques are briefly introduced here.

GP is considered as a particular application of GA with a set of different encoded genes. In particular, the representation of GA is commonly represented as a linear structure. However, In GP, each individual is commonly represented as a tree structure, which has a terminal set and a function set. In addition, the tree structure is considered be efficiently evaluated recursively. Three genetic operators consisting of reproduction, crossover, and mutation are involved in to generate next generation for both GA and GP. Reproduction operator retains the elite individuals without any changes. Crossover operator replaces one node of one individual with another node of another individual. Mutation operator replaces a randomly selected node in an individual. The whole evaluation process won't stop unless an optimal solution found or a pre-defined number of generation reached.

PSO is one of swarm intelligence (SI) techniques that is based on the behavior of decentralized, self organized system. PSO algorithm is initialized by a group of random particles, which direct or indirect present the solutions. Those particles explores for the optimization position, which is approached by repeating the process of transferring particles position according to both their own best-known position and global best position.

Artificial immune system (AIS) has been studied for performing machine learning, pattern recognition and solving optimization problems.  Clonal Section Algorithm (CSA) is one of AIS for handling optimization problems, and the principle of utilizing Clonal Section Algorithm lie in the features of immune memory, affinity maturation. In particular, the antigen is considered as a fitness function instead of the explicit antigen population. A proportion of antibody, rather than the best affinity antibody, are chosen to proliferation. Further more, speed and accuracy of the immune responses grow higher and higher after each infection even confronting cross-reactive response. Apart from that, hypermutation and receptor editing contribute to avoiding local optimization and selecting optimized solution respectively. 


\section{Related Work}\label{related}

\subsection{Single-Objective Web Service Composition Approaches}\label{singleobjective}

In this section, approaches to QoS-aware Web service composition will be discussed in two distinct groups: EC-based approaches, which mainly rely on the EC techniques to reach the optimal solutions, and non-EC based methods, which do not utilize any bio-inspired methods. However, most of the existing works treat service composition problems as a single-objective optimization problems, where a united QoS score is computed using a simple additive weighting (SAW) technique \cite{hwang1981lecture}.
\subsubsection{EC-based Composition Approaches}
EC-based Web service composition mainly relies on evolutionary computation algorithms for searching optimal solutions. These algorithms are inspired by the behavior of human, animals or even T-cells. To use different EC algorithms, proper representations need to be designed. These representations can directly or indirectly represent the service composition solutions. Herein we mainly discuss some promising research works on QoS-aware Web service composition using  Genetic Algorithm (GA), Genetic Programming (GP), Particle Swarm optimization (PSO), and Clonal Selection Algorithm (CSA).

\textbf{Genetic Algorithm}. 
GA is a very reliable and powerful technique for solving combinatorial optimization problems \cite{srinivas1994genetic}. It has been applied to solve optimization problems for QoS-aware Web service composition \cite{wang2012survey}. \cite{canfora2005approach} developed a GA-based approach for semi-automated QoS-aware service composition, where an abstract workflow is given. In their work, the proposed GA method are compared to a LIP-based method. The experimental finding reveals GA method is preferred when the size of service candidates increases exponentially. \cite{tang2010hybrid} proposed a hybrid approach utilizing GA and local search. In particular, a local optimizer is developed and only recalled in the initial population for improving QoS value. This local search contributes to a better overall performance compared to the one without utilizing local search. In \cite{lecue2009optimizing}, a semi-automated service composition approach is developed for optimizing the quality of semantic matchmaking and some quality criteria of QoS. In particular, the quality of matchmaking is transferred to measure the quality of semantic links, which is measured by two quality aspects: matchmaking type and degree of similarity.

\textbf{Genetic Programming}.
Tree-based representations could be more ideal for practical use, since they can present all composition constructs as inner nodes of trees. GP technique is utilized for handling tree-based representations. \cite {rodriguez2010composition} relies on GP utilizing a context-free grammar for population initialization, and uses a fitness function to penalize invalid individuals throughout evolutionary process. This method is considered to be less efficient as it represents a low rate of fitness convergence. To overcome the disadvantages of \cite {rodriguez2010composition}, \cite{yu2013adaptive} proposes a GP-based approach employing the standard GP to bypass the low rate of convergence and premature convergence. The idea of this paper is to increase the mutation rate while encountering low diversity in the population, and to adopt a higher crossover probability while trapped in local optimization.  During the evolutionary process, the elitism strategy is adopted, in which the best individual produced is reproduced to next generation directly without crossover and mutation. \cite{ma2015hybrid} proposes a hybrid approach combining GP and a greedy algorithm. In particular, a set of DAGs that represent valid solutions are initialized by a random greedy search and transferred into trees using the graph unfolding technique.  In each individual,  terminal nodes are considered as task inputs,  root node as  outputs, and all the inner nodes as atomic Web services. During the reproduction process,  a randomly selected node on one individual is replaced with a new subtree generated by a greedy search to perform mutation while same atomic inner nodes in two random chosen individuals are swapped to perform crossover. However, \cite{da2016genetic} proposes a different transformation algorithm to present composition constructs as the functional nodes of trees. On the whole, all these GP-based approaches \cite{ma2015hybrid,rodriguez2010composition,da2016genetic,yu2013adaptive} consistently ignore the semantic matchmaking quality, and their representations do not preserve semantic matchmaking information and composition constructs simultaneously. 

\textbf{Graph-Based Genetic Programming}.
A graph-evolutionary approach is introduced in \cite{da2015graphevol} with graph-based genetic operators, which are utilized to evolve individuals represented by graph-based representations. Although graph-based representations are capable of presenting all the matchmaking relationships as edges, they hardly present some composition constructs (e.g., loop and choice). \cite{da2016handling} investigated Directed Acyclic Graph with branches using GraphEvol approach \cite{da2015graphevol} to find near-optimal QoS for Web service composition. This approach is compared to the GP approach \cite{da2015gp}. The experiment results reveal a significant improvement in execution time with a slightly tradeoff in the fitness value. However, the branches handling in their approach has a big limitation. That is, any nested choice composition construct is not supported.

\textbf{Particle Swarm optimization}.
PSO is considered to a simple and effective approach for solving combinatorial optimization problems with few parameters settings \cite{long2009environment}. The paper \cite{liu2007hybrid} proposed a hybrid Genetic Particle Swarm optimization Algorithm (GPSA). In particular, GA is employed with only crossover operator to produce new individuals with $n_1$ iterations while PSO is only utilized for local searching (i.e., $C_2$ parameter is set to 0 in the standard velocity updating functional) with $n_2$ iterations. This approach achieves a good balance of global and local optimization through a  mechanism based on two thresholds, which determine the values of $n_1$ and $n_2$. However, this work \cite{liu2007hybrid} only handles semi-automated service composition problems. On the other hand, \cite{da2016particle} proposes a PSO-based fully automated approach to generate a composition solution from an optimized service queue. The idea is to translate the particle location into a service queue as an indirect representation of a composition solution, so finding the best composite solution is to discover the optimized location of the particle in the search space. However, only QoS is optimized in their approach.

\textbf{Clonal Selection Algorithm}.
The paper \cite{yan2006immune} introduces a novel Web service composition approach using an immune algorithm for QoS optimization with a constraint on cost. As a given abstract graph could be broken into several single pipelines, the optimization problem is transferred into getting an optimal executing plan for each single pipeline. In a pipeline, each abstract task could be slotted with several alternative Web services with QoS values labelled to their edges so that a weighted multistage graph is established for optimal paths selection. In the immune algorithm, the service composition problem is encoded using a binary string as an antibody for evaluating the affinity value of the antigen ( i.e.,fitness function ), and the antibody with low concentration will be selected for crossover and mutation. However, the efficiency of creating the weighted multistage graph is to very low. The paper \cite{pop2009immune} introduced an immune-inspired Web service composition approach combining an enhancing planning graph (EPG) and a clonal selection algorithm to solve optimization problem considering both semantic link quality and QoS.  The EPG model is characterized with actions and layers involved in multiple stages, where each action represents clustered Web services, and each layer represents input s or outputs grouped in concepts. During the clonal selection process, the antigen is represented as a fitness function, and the antibody is represented as a binary alphabet to encode EPG. The remaining steps are standard computation procedure for CLONALG, which is a repeated procedure consisting of antibody selection, affinity maturation process, low-affinity antibody replacement.

\subsubsection{AI Based Approaches}

AI Planning techniques have been widely employed for service composition \cite{markou2015non,peer2005web}. The main idea of these techniques considers services as actions that are defined with functional properties ( i.e., inputs, outputs, preconditions and effects) using classic planning algorithms. 

Various AI planning approaches \cite{feng2013dynamic,huang2009effective,rao2006mixed,wang2013genetic, wang2014automated} have been presented to solve semantic Web service composition problems using the Graphplan algorithm \cite{blum1997fast}. \cite{wang2014automated} employs the Graphplan to secure the correctness of overall functionality, which enables atomic Web services to be concretely selected and accurately matched for achieving desired functionality. In addition, conditional branch structure is also handled. The pitfalls of this approach are procuring only linear sequences of actions, and it is hard to deal with QoS optimization. In paper \cite{feng2013dynamic}, service-dependent QoS is modeled for QoS-aware Web service composition. This dependent QoS model is formed in three cases: a default QoS attribute, a partially dependent QoS attribute, and a completely dependent QoS attribute. They are used for the dependency checking base on a backwards Graph building with a breadth-first strategy. However, computation of service dependencies is very intensive for initialization and updating. Some approaches \cite{lecue2007making,sohrabi2009Web} rely some frameworks supported by particular agent programming languages (e.g., Golog \cite{sohrabi2009Web}  and SHOP2 \cite{sirin2004htn}) to composite Web services. In \cite{sohrabi2009Web}, a service composition framework supported by Golog is used to describe the properties of services and users' preferences. Therefore, Golog can effectively perform a  search to reach a terminating situation as a service composition solution. As summarized here, AI planning techniques are considered to be less efficient, and not capable of dealing with optimization solutions for service composition (e.g., generate either optimal QoS or number of services) independently. In addition, they may suffer scalability issues when large service repositories are given. 

\subsubsection{Other Approaches}
The non-EC based approaches do not rely on bio-inspired approaches. They target the optimized service composition solutions by some other methods, such as, integer programming, exhaustive search, local search and so on.

\textbf{Integer Linear Programming (ILP)}. ILP methods are utilized for achieving Web service composition. Generally, an ILP model is created with three inputs: a set of decision variables, an objective function and a set of constraints. The outputs are values of maximized/minimized objective function and decision variables. ILP is flexible for handling QoS  constraints and optimizing problems for QoS-aware service composition. \cite{gao2005Web} define a zero-one IP model for Web service composition based on an abstract service workflow, where services with same functionality but different QoS are classified into a same class. \cite{yoo2008Web} formulated Web service composition problem based on the model introduced in \cite{gao2005Web}. They simultaneously take both QoS and constraints on QoS into account. However, due to an increase in the number of decision variables, ILP may lead to exponentially increased complexity and cost in computation \cite{li2016full}. The resulted huge delay is not allowed in the real world scenarios. In addition, if non-linear function is utilized in ILP, the scalability is also a big problem.
 
\textbf{Dynamic Programming Approach}. Dynamic programming is an effective method for solving problems that can be broken into subproblems, which present many repetitions and substructures. In  \cite{huang2009effective}, an efficient pruning approach is developed combing three techniques --- a forward filtering algorithm for searching task-related services, a modified dynamic programming approach for dealing a subproblem of service composition (i.e., a problem on satisfaction of each concept pool of every graph layer ), and a backward-search method for searching optimal composition solutions. \cite{xu2012towards} forges a problem on solving large-scale service composition efficiently with QoS guarantee, where a dynamic programming algorithm named QDA is developed to optimize every subproblem (i.e., service paths). In particular, best-known QoS are recorded and updated for all added Web services. To derive an execution plan, a depth-first search is utilized to traceback paths with maximum QoS. However, the global best QoS cannot be reached due to a trade-off in the efficiency at an expense of optimal.

\textbf{ER Model-Based Approach (ILP)}. Most of the small and medium business rely on ER database to process information and data. \cite{xu2010semantic} employs ER model to construct domain ontology and semantic Web services. Therefore, semantic Web service composition problem is transferred to reason composite service based on a link path between entities in the ER model. However, loop and switch constructs cannot be effectively handled in their approach, and they do not deal with optimization problem at all.

\subsection{Multi-objective, and Many-objective Web Service Composition Approaches}\label{multiobjective}
%----------------------chen starts---------------------------------------------------
Maximizing or minimizing a single objective function is a most commonly used way to handle optimizing problems in automated Web service composition.  That is a Simple Additive Weighting (SAW) \cite{hwang1981lecture} technique, which presents a utility function for all the individual quality criteria as a whole. This technique optimizes and ranks each Web service composition using a single value for each solution. However,  the limitation of this technique lies in not handing the conflicting quality criteria.  Those conflicting quality criteria are always presented as trade-offs. To overcome this limitation, a set of objectives corresponding to different independent quality criteria are optimized simultaneously. Consequently,  a set of promising solutions that presents many quality criteria trade-offs are returned.


\subsubsection{Multi-objective approaches}\label{MultiObjective}

Many multi-objective techniques \cite{liu2005dynamic,zhang2010qos,yu2013efficient,yin2014hybrid,xiang2014qos,chen2014partial} have been investigated to extensively study QoS-aware Web service composition problems.  A set of optimized solutions is ranked based on a set of independent objectives, i.e., different QoS attributes. In particular, solutions are compared according to their relationship for domination. That is, figure out solutions that clearly dominate the others. For example, given two service composition solutions that are compared based on execution cost $c$ and execution time $t$, solution one, $wsc_1(c=10,t=1)$ and solution two,  $wsc_2(c=13,t=1)$. In our context, $wsc_1$ dominate $wsc_2$ as $wsc_1$ has a same execution time, but a lower execution cost. If given $wsc_3(c=10,t=2)$, $wsc_2$ is a \textit{non-dominant} solution in the relation to $wsc_3$ because of its longer execution time and cheaper execution cost. Therefore, non-dominant solutions are globally produced among both the dominant and non-dominant solutions, i.e., they do not dominate themselves. These solutions are called a \textit{Pareto Front}, which provide a set of non-dominant solutions for users to choose.


\textbf{Multi-objective techniques with GA}. Many approaches to multi-objective Web service composition employs GA \cite{liu2005dynamic}.  \cite{liu2005dynamic} employs a service composition model, called  MCOOP (i.e., multi-constraint and multi-objective optimal path) as Web service composition solutions with only sequence composition construct supported. In the model, different paths are selected from a service composition graph that includes $N$ service group. In each group, services present same functionality with different QoS. They developed a strategy GODSS (Global Optimization of Dynamic Web Service Selection) based on multi-objective Genetic Algorithm, where two points crossover and mutation are applied to their approach to speed up the astringency of the algorithm. The work \cite{wada2012e3} investigated a semi-automated approach to SLA-aware Web service composition.  Each vector-based representation presents three service composition solutions for three group users.  The individuals are randomly initialized, and further evaluated and optimized based on objectives from all the possible combinations of throughput, latency, cost and user category. Two multi-objective genetic algorithms are developed in the work: E-MOGA and Extreme-E. E-MOGA is proposed to search a set of solutions that are equally distributed in the searching space using a fitness function, where a production of domination value,  Manhattan distance to the worst point and sparsity is assigned to feasible individuals as fitness values, and $SLA$ $violation /domination$ $value$ is assigned to infeasible solutions. On the other hand, Extrem-E produces extreme solutions using a fitness function, where weights use a term 1/exp(p-1), where $p$ is the number of objectives and is assigned to the $p^{th}$ objective.

\textbf{Multi-objective techniques with PSO}. The work \cite{yin2014hybrid} combines genetic operators and particle swarm optimization algorithm together to tackle the multi-objective SLA-aware Web service composition problems. A discrete PSO-based method is proposed for  considering different scare of cases. In particular, the updates of particle's velocity and position are achieved by the crossover operator, where both velocity and position of new individual are updated in accordance with positions of \textit{pbest}, \textit{gbest}, and current velocity. On the other hand, mutation strategy is introduced to increase the diversity of particle, and it is performed on the \textit{gbest} particle, if the proposed swarm diversity indicator is below a certain value.

\textbf{Multi-objective techniques with ACO}. Generally, ACO simulates foraging behaviors of a group of ants for optimizing the traversed foraging path, where the strength of pheromones is taken account for. The work \cite{zhang2010qos} turns the service composition problem into path selection problem, where a given abstract workflow is given with different service candidate sets. This work employs a different strategy of "divide and conquer`` for decomposing a given workflow. That is,  two or more abstract execution paths are decomposed from the workflow with no overlapped abstract services. This decomposing strategy results in a much smaller length of the execution paths compared to the works \cite{yu2007efficient}.  Also, a new ACO algorithm is proposed to handle the multi-objective QoS-aware service composition problem. In particular,  the phenomenon is presented as a k-tuple for $k$ objectives, rather than a single value. Apart from that, a different phenomenon updating rule is proposed by considering an employment of a proposed utility function as a global criterion. The paper \cite{wang2014novel} introduces nonfunctional attributes of Web services to include trust degree according to the execution log. A novel adaptive ant colony optimization algorithm is proposed to overcome the slow convergence presented from the traditional ant colony optimization algorithm. In particular, the pheromone strength coefficient is adjusted dynamically to control both the updating and evaporation of pheromone. 
%The experiment results are analyzed in an alternative way. That is, the total Pareto solutions are combined from different compared ACO algorithms, then the accurate rate of each algorithm is calculated based on the compared Pareto solutions identified in the total Pareto solutions. The results also show more Pareto solutions found compared to the traditional ACO methods. However, the experiment is only conducted for the evaluation of a small case study, where only a simple abstract workflow is studied.

\subsubsection{Many-objective approaches}\label{ManyObjective}

Herein, more than three objectives in multi-objective problems (MOPs) are often considered as many-objective problems. Ishibuchi et al. \cite{ishibuchi2008evolutionary} present an analysis of the multi-objective algorithm for handling optimization problems with more than 3 objectives. However, they address that the searching ability is deteriorating while the number of objectives increase exponentially. This is due to a very large size of non-dominated solutions, which make it harder to move solutions towards the Pareto Front.

The work \cite{de2010many} employs NSGA-II to deal with five different quality criteria (i.e., runtime, price, reputation, availability and reliability) for semi-automated Web service composition problem.  To decrease the deterioration, two preference relations proposed by \cite{bentley1997finding} are applied to NSGA-II: Maximum Ranking (MR) and Average Ranking methods (AR). In particular, MR is the best of all the ranking scores from all the objectives, and AR is a sum of all the ranking scores from all the objectives. Therefore, three algorithms (NSGA-II, NSGA-II with MR and NSGA-II with AR) are compared for studying the five different performance metrics ( i.e., hypervolume \cite{zitzler1999evolutionary}, Generational Distance \cite{van2000measuring}, Spread and Coverage \cite{zitzler2000comparison}, and pseudo Pareto front (i.e., a combination of all non-dominated solutions)). The experiment shows NSGA-II with AR outperforms others in both GD and Spread (i.e., more balanced solutions). However, only a certain region of Pareto Front is generated by NSGA-II, rather than a wider distribution one. NSGA-II with MR performs intermediately compared to the other two algorithms. On the whole, this work firstly takes two preference relations into account for solving many-objective service composition problem, and contribute to finding better solutions with many performance metrics. However, this work is only approached in a semi-automated way.


\subsubsection{Preferences Articulation Techniques for Multi-and Many-objective Approaches}\label{PreferencesMultiObjective}
Most of the EMO algorithms focus on generating evenly distributed non-dominated Pareto solutions. Often, an indispensable decision must be made for choosing a small number of solutions, which satisfy customers' preferences. However, in some practical scenarios, some issues in EMO algorithms need to be handled when no or few solutions are gained in the preferred areas. To solve these issues, some user preferences-based approaches have been proposed to search only the space that is preferred by users, and to increases the density of solutions in that space. These approaches can be classified into three groups based on the articulation of preferences \cite{van2000multiobjective}. The first group is prior approaches, where the articulation is done before optimization process; The second group is interactive approaches, where the articulation is done during optimization process; The third group is posteriori approaches, where the articulation is done after optimization process. As argued in \cite{giagkiozis2014pareto}, the third approaches are capable of generating more desired solutions in the preferred region. 


Most of the multi-objective service composition assumes that the user preferences are not known in advance. However,  often, a vague preference is provided by users at least, by whom preferred solutions are roughly specified.  These vague preferences should be taken into decision making and guide us to search for the most interesting areas on the Pareto front. In \cite{branke2005integrating}, they proposed two effective and efficient methods for finding the most relevant parts of Pareto solutions according to users' preferences. In particular, vague trade-offs between different objectives are taken into account when users have some rough thoughts about trade-offs.  The first method is based on a guided multi-objective evolutionary algorithm in \cite{branke2001guidance}, and the second one is approached by a new biased crowding operator. In the first method, the dominance is specified for a maximally acceptable trade-off for each pair of objectives, which is appropriately transferred into two auxiliary objectives. The second method works on finding a biased distribution on the Pareto solutions. The idea of this method is to introduce a biased crowding distance, which controls the expansion of solutions according to their location projected on a proposed hyper plane, i.e., users' specified direction vector.  The experiment shows that both methods can effectively and efficiently find relevant Pareto solutions for users. In \cite{cheng2015reference}, a posteriori articulation method is proposed to find Pareto solutions in the preferred areas using reference vectors, which are created based on the observation of given Pareto solutions. The reference vectors divide the objective space into subspaces, where one individual is selected from each subspace and put into the next generation. Therefore, the newly generated individuals move towards optimal Pareto Front and reference vectors. The experiments show that this method could handle preferences effectively and efficiently. On the whole, none of existing preference articulation techniques focus on Web service composition problems, let alone applicability for comprehensive quality-aware semantic Web service composition. 



\subsection{Dynamic Web Service Composition Approaches}\label{dynamicserivce}
All the previously discussed approaches can be classified into one group that assumes the composition environment is static. The rest approaches can be classified into another group that does not make a closed world assumption. Instead, a real world scenario takes a dynamic composition environment into account. For example, nonfunctional properties of Web services may fluctuate over time or services are failed/newly registered. To address this problem \cite{nasridinov2012qos},  a suitable mechanism for effectively and efficiently handle this problem raise a significant challenge.

\subsubsection{Dynamic Web Service Composition Approaches For Changes in QoS}\label{dynamicQoS}

Service selection is one of the crucial steps when we compose services. In the context of static Web service composition, we always selection services based on the QoS advertised by service providers. However, QoS is fluctuating over the time according to the instances of a service at the run time. This dynamic QoS is formally modeled as an uncertain QoS model. Based on this model, some studies \cite{wen2014probabilistic} have been addressed recently. To efficient provide a relatively small set of services for selection based on the uncertain QoS model, the data structure of R-tree is introduced for spatial query on multidimensional data (i.e., many dimension of QoS attributes) since it can significantly reduce the searching space. This space index technique is one of the key contributions in this paper for efficiently store and retrieve services for service selection. 

Reinforcement learning (RL) is one technique of machine learning for solving sequential decision-making problems to maximize some long-term rewards.  RL is utilized to deal with how actions are taken in an uncertain environment. In our context, this uncertain environment is related to QoS. In \cite{mostafa2015multi}, they combine multi-objective optimization and reinforcement techniques to solve multi-objective service composition problem in an uncertain and dynamic environment. In particular, we service composition is modeled based on Partially Observable Markov Decision Process (POMDP), and solutions to services composition are considered to be a set of decision policies. Each decision policy is considered as a procedure of service selection (i.e., a single workflow). They proposed a method to learn an optimal selection policy to reach optimal solutions.

\cite{nasridinov2012qos} proposes a QoS-aware performance prediction for a self-healing Web service composition system. This system consists of three main phases: monitoring, diagnosis and repair. The monitoring phase is to detect degradation, diagnosis is to identify the source of degradation, and repair is to reselect desired services. To minimize the number of re-selection in the phase of repair, decision tree learning is used to the prediction of the performance based on the QoS. This technique outperforms other classification techniques (i.e., back propagation neural network, support vector machine probabilistic neural network, and group method of data handling and regression tree \cite{mohanty2010web}.

\subsubsection{Dynamic Web Service Composition Approaches For Service Failure}\label{dynamicService}
Traditionally, re-selection of failed service is one of widely used approaches to re-ensure the  service composition plan. The idea of this traditional approach is to restore many alternative service candidates for each component service involved in the composition solution. If a component service confronts a failure, the alternatives are used for the replacement.  A framework, WS-Replication is introduced in \cite{salas2006ws}, which mainly address service failure using the idea of the traditional method. However, those approaches \cite{salas2006ws} suffer a huge cost in computation. To overcome the disadvantage of traditional resection approach,  \cite{wagner2016robust} proposes a method to cluster services as a back up for service failures. This method determines a set of backup services based on their functional properties as different clusters. In particular, a set of backup services is available in both the clusters in and the sub-clusters, from which we can effectively select a suitable service.

As summarized here, existing approaches handle either changes in QoS or service failures. None of these approach handles services newly registered and changes in ontology. Technically, all these approaches do not allow the changes of composition structure, and they mainly focus on efficient resection techniques for desired services.



\subsection{Web Service Composition Approaches Supporting Preconditions and Effects}\label{Semantic}
Apart from the consideration of satisfactions on inputs and outputs of Web services, more complex chaining strategies for semantic services are proposed to handle preconditions and effects. To cope with these complicated requirements, preconditions and effects are well formulated by ontology-based techniques (e.g., DL) for supporting a chaining ability of semantic Web services \cite{wang2014automated}.

Most of the existing service composition is approached merely based on input and outputs, but preconditions and effects are not considered. Therefore, it is hard to generate service composition solutions with some composition constructs (e.g., choice and loop). A generalized semantic Web service composition is introduced in \cite{bansal2016generalized}, preconditions and effects are effectively presented in a conditional directed acyclic graph where conditional nodes created with two outgoing edges representing a satisfied and a unsatisfied case at the run time. They filter the solutions based on a trust rate using Centrality Measure of Social Networks to find trusted Web services. They also implement a semantic Web service composition engine for automated conditional service composition. In \cite{wang2016automatic}, an extensive Graphplan technique is proposed to support preconditions and effects in a direct representation. In particular, a two-level directed graph, called planning graph (PG) is utilized and comprises of two kinds of nodes (i.e., proposition and action ones) and three kinds of edges (i.e., precondition-edges, add-edges, and delete-edges). The proposition and action level is alternated between each other in PG. By utilizing this presentation, the branch structure is supported in the service composition solution. On the whole, these two approaches do not support loop construct, and leave optimization problems aside.


\subsection{Summary and Limitations}\label{summary}

An overview of recent related research on Web service composition is presented in this chapter. The first related research area is \textbf{single-objective Web service composition}, which mainly works on generating composition solutions with optimal QoS or number of services based on the objective functions. On the one hand, EC-based approaches have been widely used for effectively and efficiently solving the above problems. On the other hand, non-EC based approaches, such as AI planning approaches, Integer Linear Programming, Dynamic Programming Approaches are employed in Web service composition, but they may suffer scalability due to an increase in the size of searching space or leave aside optimization problems. None of existing approaches simultaneously optimize QoS and quality of semantic matchmaking in fully automated way.

%The key limitation of single-objective Web service composition approaches is that the importance of semantic matchmaking quality is ignored, which should be simultaneously taken into account with QoS. To cover this necessity, new representations and quality model must be proposed along with service composition methods.

The second related research area is \textbf{EMO-based service composition}, in which different quality criteria of QoS are simultaneously optimized and a near-optimal Pareto Front is produced. Existing works on EMO-based approaches focus on QoS-aware semantic web service composition in a semi-automated way. Sometimes, SLA constraints are considered. However, none of these approaches take QoS and quality of semantic matchmaking simultaneously in a fully automated approach. Apart from that, none of existing preference articulation techniques are investigated in the domain of Web service composition.

\textbf{Dynamic Web service composition} is discussed in the third related research area, where a dynamic service composition environment is handled. On the one hand, existing dynamic Web service composition mainly works on various techniques to efficiently handling changes in QoS and service failures, such as R-tree, decision tree learning and RL. These approaches do not allow the changes of service composition structure. Apart from that, the cost of initial planning is ignored and separated from the adaption of dynamic environment. On the other hand, EC-based approach have not been utilized in dynamic web service composition so far, where the changes of composition structure is allowed and avoid re-planning cost through re-using known plans (composite services) obtained prior. 


The last related research area focuses on \textbf{semantic Web service composition}, which explore a more complex service composition. Most approaches on Web service composition support precondition and effects. However, loop composition construct is not handled and combinatorial optimization problems is left alone. On the whole, none of existing works achieve web service composition supporting precondition and effects for all the composition constructs while simultaneously  optimizing QoS and quality of semantic matchmaking.
