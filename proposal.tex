%% $RCSfile: proj_report_outline.tex,v $
%% $Revision: 1.3 $
%% $Date: 2016/06/10 03:41:54 $
%% $Author: kevin $

\documentclass[11pt
              , a4paper
              , twoside
              , openright
              ]{report}


\usepackage{float} % lets you have non-floating floats

\usepackage{url} % for typesetting urls

%
%  We don't want figures to float so we define
%
\newfloat{fig}{thp}{lof}[chapter]
\floatname{fig}{Figure}

%% These are standard LaTeX definitions for the document
%%      
%% Personal package
\usepackage[ruled,vlined,linesnumbered]{algorithm2e}
\usepackage{color,graphicx,epstopdf,changepage,amsmath,amsthm,multirow,lipsum}
\usepackage[justification=centering]{caption}
\usepackage{blindtext}
\usepackage[inline]{enumitem}
\usepackage{xcolor}
\SetAlgoCaptionSeparator{.\space}
\renewcommand\AlCapFnt{\normalfont\scshape}
\setlength{\algomargin}{0.7cm}
\newtheorem{example}{Example}




\makeatother

\newcommand{\cse}{C}
\newcommand{\gra}{\mathcal{G}}
\newcommand{\tree}{\mathcal{T}}
\newcommand{\rin}{\mathcal{I}}%should we skip the subscript r (for atomic services required inputs are just the inputs)?
\newcommand{\rout}{\mathcal{O}}%should we skip the subscript r (for atomic services required outputs are just the outputs)?
\newcommand{\prin}{\mathcal{PI}}%should we call them available inputs (rather than provided inputs)?


%% We don't want figures to float so we define
\newfloat{fig}{thp}{lof}[chapter]
\floatname{fig}{Figure}
                      
\title{Comprehensive Quality-Aware Automated Semantic Web Service Composition}
\author{Chen Wang}

%% This file can be used for creating a wide range of reports
%%  across various Schools
%%
%% Set up some things, mostly for the front page, for your specific document
%
% Current options are:
% [ecs|msor]              Which school you are in.
%
% [bschonscomp|mcompsci]  Which degree you are doing
%                          You can also specify any other degree by name
%                          (see below)
% [font|image]            Use a font or an image for the VUW logo
%                          The font option will only work on ECS systems
%
%\usepackage[font,ecs,mcompsci]{vuwproject}
\usepackage[image,ecs,mcompsci]{vuwproject}%uncommend for edit at home
% You should specifiy your supervisor here with
     \supervisors{Dr. Hui Ma and Dr. Aaron Chen}
% use \supervisors if there is more than one supervisor

% Unless you've used the bschonscomp or mcompsci
%  options above use
   \otherdegree{PhD}
% here to specify degree

% Comment this out if you want the date printed.
%\date{}

\begin{document}

% Make the page numbering roman, until after the contentWebs, etc.
\frontmatter

%%%%%%%%%%%%%%%%%%%%%%%%%%%%%%%%%%%%%%%%%%%%%%%%%%%%%%%

%%%%%%%%%%%%%%%%%%%%%%%%%%%%%%%%%%%%%%%%%%%%%%%%%%%%%%%
\begin{abstract}
\small{Automated Web service composition is an NP-hard problem and it has been raising much attention in the research community due to the computational challenge and real-world applicability. Existing works either optimize QoS or semantic matchmaking quality, or are semi-automated approaches. The focus of our studies on service composition is to find effective and efficient approaches to comprehensive quality-aware semantic Web service composition, which aims to optimize semantic matchmaking quality and Quality of service (QoS) simultaneously. We will address this problem by achieving the following objectives: (1) developing EC-based approaches that explicitly support the comprehensive quality, (2) developing multi-objective approaches for optimizing the quality criteria involved in the comprehensive quality, (3) developing EC-based  composition approaches for dynamic service composition while handling various changes of composition environment, and  (4) developing EC-based approaches for semantic Web service composition supporting precondition and effects. In our preliminary work, we developed two EC-based approaches to single-objective comprehensive quality-aware automated semantic web service composition.}
\end{abstract}

%%%%%%%%%%%%%%%%%%%%%%%%%%%%%%%%%%%%%%%%%%%%%%%%%%%%%%%

\maketitle

%\include{acknowledge}

\tableofcontents

% we want a list of the figures we defined
\listof{fig}{Figures}

%%%%%%%%%%%%%%%%%%%%%%%%%%%%%%%%%%%%%%%%%%%%%%%%%%%%%%%

\mainmatter

%%%%%%%%%%%%%%%%%%%%%%%%%%%%%%%%%%%%%%%%%%%%%%%%%%%%%%%

% individual chapters included here
\chapter{Introduction}\label{C:intro}

\section{Problem Statement}
\emph{Service-oriented computing} (SOC) is a novel computing paradigm that employs services as fundamental elements to achieve the agile development of cost-efficient and integrable enterprise applications in heterogeneous environments \cite{papazoglou2003service, papazoglou2006p}. One of the primary purposes of SOC is to overcome conflicts due to diverse platforms and programming languages to make integrable and seamless communication among those existing or newly built independent services. \emph{Service Oriented Architecture} (SOA)  could abstractly realise service-oriented paradigm of computing. This accomplishment has been contributing to reuse of software components, from the concept of functions to units and from units to services during the evolution of development in SOA \cite{booth2004web, overdick2007resource}. One of the most typical implementation of SOA is \emph{web service}, which is designated as ``modular, self-describing, self-contained applications that are available on the Internet" \cite{curbera2001web}. Several standards play a significant role in registering, enquiring and grounding web services across the web, such as UDDI \cite{curbera2002unraveling}, WSDL \cite{lausen2007semantic} and SOAP \cite{fensel2011semantic}. \emph{Web service composition} aims to loosely couple a set of web services to provide a value-added composite service that accommodates complex functional and non-functional requirements of service users. 

Two most notable challenges for web service composition are ensuring interoperability of services and achieving Quality of Service (QoS) optimisation \cite{fensel2011semantic}. \emph{Interoperability} of web services presents challenges in syntactic and semantic dimensions. On the one hand, the syntactic dimension is covered by the XML-based technologies \cite{yu2008deploying}, such as $WSDL$ and $SOAP$. In this dimension, most services are composed together merely based on the matching of input-output parameters. On the other hand, the semantic dimension enables a better collaboration through ontology-based semantics \cite{o2005review}, in which many standards have been established, such as OWL-S \cite{martin2004owl}, Web Service Modeling Ontology (WSMO) \cite{lausen2005w3c}, SAWSDL \cite{kopecky2007sawsdl}, and Semantic Web Services Ontology (SWSO) \cite{petrie2016web}. This dimension brings around some other underlying functionality of services (i.e., precondition and postcondition) that could effect the execution of web services and their composition. The interoperability challenge gives birth to \emph{Semantic web services composition}, as a different technique from traditional service composition methods (i.e., only syntactic dimension is presented in web services). Since the quality of semantic matchmaking is always used to measure the goodness of interoperability while composing services. Another challenge is related to QoS optimisation, where QoS represents non-functional attributes of service composition (e.g, cost, time, reliability and availability). Often, a global search is employed to minimise the cost and maximise the reliability of composite services. This challenge gives birth to \emph{QoS-aware service composition} that aims to find composition solutions with optimised QoS. Furthermore, QoS-aware service composition problem is facilitated by \emph{Service Level Agreement (SLA)} \cite{sahai2002automated}, i.e., binding constraints on QoS. This results in the \emph{SLA-aware web service composition}, e.g., constraints on cost, execution time, availability and reliability are separately specified with both lower and upper bounds.

Apart from the two notable challenges discussed above, the environment of service composition is changing in the real world, rather than \emph{static}. E.g, QoS values of services being composed of are fluctuating over time. Services chosen at the planning stage may not available to be invoked at the runtime, it may due to imprecise interface description \cite{ishikawa2011bridging} or hardware failures \cite{guinard2009discovery}. New services can also be registered in the service repository from time to time. Existing techniques for \emph{static web service composition} can not support such a changing environment. Therefore, \emph{Dynamic web service composition} become a very demanding research field with a growing interest and a practical value. Particularly, some mechanisms are required to be developed to automatically detect the changes or recover from the faults \cite{chan2009fault}. Additionally, in context of semantic web service composition, semantics of web services can make the problem of dynamic web service composition more complicated due to the changes in the ontology.


Different approaches \cite{da2016genetic,da2016particle,gupta2015optimization,lecue2009optimizing,ma2015hybrid,qi2010combining,rodriguez2010composition,yu2013adaptive,wang2014automated} have been proposed to solve those composition problems discussed above and they can be classified into two main categories: \emph{semi-automated web service composition} and \emph{fully automated web service composition}. The first composition problem requires human beings to manually create abstract workflows. Generally, researchers assume the pre-defined abstract workflow is given and provided by the users. The optimisation problem in this approach turns to selecting the concrete services with the best possible quality to each abstract service slot in a given workflow. Due to a tremendous growth in industries and enterprise applications, the number of web services has increased dramatically and unprecedentedly. The process of conducting abstract workflows  manually is fraught with difficulties in effectively and efficiently solving composition problems, such as QoS-aware service composition problem. Therefore, full automation of composition process is introduced in web service composition for less human intervention, less time consumption, and high productivity \cite{rao2004survey}. The differences in fully automated approach is that an abstract workflow is not provided, but generated while service are being selected. 


Generating composition plans automatically in discovering and selecting suitable web services is a NP-hard problem \cite{moghaddam2014service}, which means the composition solution is not likely to be found with reasonable computation time in a large searching space. \emph{Artificial Intelligence (AI) planning-based approaches}, \emph{Evolutionary Computation (EC) techniques} and hybrid techniques have been introduced to handle this problem. AI planning problem is utilised to solve automated web service composition problems as a plan making process, from initial states to a set of actions to desired goal states-composite web services, where services are considered as actions triggered by one state (i.e., inputs) and resulted in another state (i.e., outputs). In the second approach, heuristics have been employed to generate near-optimal solutions using a variety of EC techniques have been used in this context, e.g., Genetic Algorithms (GA), Genetic Programming (GP) and Particle Swarm Optimisation (PSO). EC-based techniques have been effectively developed to solve \emph{QoS-aware web service composition} problems with different structures for solution representations. Many representations have been carefully investigated in QoS-aware web service composition problems, since they could significantly affect the performance of fully automated service composition systems. In the third approach, a hybrid of AI planning-based approaches and EC-based approaches \cite{da2016genetic,ma2015hybrid} have been proposed to fulfil the correctness in constructing workflows under users' constraints, while the quality of composition solutions (e.g., QoS) are also optimised according to users' requirements. From the literature, hybrid approaches generally outperform EC-based methods for finding more optimal solutions in the domain of automated QoS-aware web service composition. It becomes a very promising approach to investigate. As summarised here, a few previously addressed challenges for fully automated service compositions lies in jointly optimising QoS and semantic matchmaking quality, dealing with multi-objective optimisation, handling environment changes in dynamic service composition, and composing semantic web services. Those challenges are addressed with key limitations in Section \ref{C:motivation}. 

The overall goal of this thesis is to propose hybrid approaches to comprehensive-quality aware automated web service composition. This comprehensive quality aims to jointly optimise semantic matchmaking quality and QoS. Meanwhile, this new approach also tackles several service composition problems, such as multi-objective optimisation, dynamic web service composition and semantic web service composition based on preconditions and postconditions.

\section{Motivations}\label{C:motivation}
The motivations of this proposed research lies in the requirements from five key aspects that simultaneously account for. 
\begin{enumerate*}
 \item \emph{Various techniques of hybridisation}.
 \item \emph{Hybridisation of quality of semantic matchmaking and quality of service}.
 \item \emph{Muti-objective optimisation}.
 \item \emph{Dynamic semantic service composition}.
 \item \emph{Service composition based on preconditions and effects}.
\end{enumerate*}
Herein these requirements are explicitly discussed below. 
\subsubsection{Various Techniques of Hybridisation}

Various techniques have been utilised to solve service composition problems, such as AI planning, local searching and EC-based techniques \cite{feng2013dynamic,parejo2008qos,qi2010combining,wang2014automated}. AI planning is a prominent technique for handling web service composition problems while always ensuring the correctness of the solution (i.e., all the inputs of involved services are satisfied) \cite{wang2014automated}.  Local search is a exhaustive search technique for solving optimisation problems. In local search, solutions keep moving to the neighbour solutions, driven by some local maximisation criterion until near-optimal solution found \cite{parejo2008qos}. However, this technique has the shortage of being trapped in local optimal. On the other hand, EC-based techniques are outstanding at solving combinatorial optimisation problems for finding the globally optimised solutions in large searching spaces. To take the benefits from various techniques,  various techniques of hybridisation allows escaping local optima easily and improving the rate of convergence rate \cite{renders1996hybrid}.

Traditionally, various techniques are employed to handle service composition problem independently in existing works. Many researchers investigated AI planning techniques for service composition problems using classical planning algorithm, where inputs, outputs, preconditions and effects are well defined along with the actions (i.e, services) \cite{markou2015non,peer2005web}. On the one hand, AI planning ensures both the correctness of functionality and satisfactory of constrains, but it is always considered to be less efficient and less scalable, incapable of solving complicated optimisation problems \cite{parejo2008qos}. On the other hand, some researchers combined AI planning and local search to handle optimisation problems, e.g., a combination of Graphplan \cite{blum1997fast} and Dijkstra’s algorithm is proposed by \cite{feng2013dynamic} to achieve a correct solution with optimised QoS. Yet, many EC techniques have been utilised to handle service composition problems for global optimal solutions. A few researchers also combine both local search and EC-based techniques for efficiently finding composition solution with optimised QoS \cite{parejo2008qos}. From the techniques discussed above, they are problem-specific for either optimising QoS or number of services. In this thesis, more complicated and realistic service composition problems are addressed. To with cope this composition problems featured in all the motivations discussed below (i.e comprehensive quality-aware service composition, multi-objective optimisation, dynamic service composition, and composing semantic web services based on preconditions and postconditions ), new hybrid methods must be proposed to adapt in the problems specified in the thesis. For example, a hybrid method for jointly optimising both semantic matchmaking quality and QoS. 

\subsubsection{Hybridisation of Quality of Semantic Matchmaking and Quality of Service}\label{SC:hybridisation}
Web service compositions are optimised by the well known non-functional attributes (i.e, QoS), when services are composed together based on outputs provided by one service and inputs required by another service. In the domain of semantic web services, often, the provided information (i.e., outputs) does not perfectly match the required information (i.e., inputs) according to their semantic descriptions \cite{lecue2008optimizing}. The quality of the matches (i.e., quality of semantic matchmaking) are one part of the goal for achieving service compositions \cite{lecue2009optimizing}. Therefore, a hybridisation of QoS and semantic matchmaking quality becomes a combinatorial optimisation problem in web service composition. One motivated example from the practical perspective is explained here: many different service compositions can meet a user request but differ significantly in terms of QoS and semantic matchmaking quality. For example, in the classical travel planning context, some component service must be employed to obtain a travel map. Suppose that two services can be considered for this purpose. One service $S$ can provide a street map at a price of 6.72. The other service $S'$ can provide a tourist map at a price of 16.87. Because in our context a tourist map is more desirable than a street map, $S'$ clearly enjoys better semantic matchmaking quality than $S$ but will have negative impact on the QoS of the service composition (i.e., the price is much higher). One can easily imagine that similar challenges frequently occur when looking for service compositions. Hence, a good balance between QoS and semantic matchmaking quality is called for.

Existing works on service composition focus mainly on addressing only one quality aspect discussed above. For the semantic matchmaking quality, it is mainly addressed in works that focus on the discovery of atomic services, i.e., one-to-one matching of user requirements to a single service. Some works \cite{bansal2016generalized,boustil2014semantic,mier2015integrated} on semantic service composition utilise semantic descriptions of web services (e.g., description logic) to ensure the interoperability of web services, but the goal of the composition is often to minimise the number of services or the size of a graph representation for a web service composition. These approaches do not guarantee an optimised QoS of service compositions. On the other hand, huge efforts have been devoted to studying QoS-aware web service composition \cite{da2015graphevol,da2016particle,gupta2015optimization,ma2015hybrid,qi2010combining,yu2013adaptive}. Some of these works do consider the semantic matchmaking while composing solutions, but do not recognise the importance of semantic matchmaking quality for optimising service composition. Therefore, it is not sufficient to only consider one quality aspect for optimising service composition. For these reasons, there is a need to device an comprehensive quality model for jointly optimising the two quality aspects. Apart from that, existing solutions representations may not effectively and efficiently handle this new optimisation problem, so new solution representations need to be proposed to maintain the required information for the optimisation of the two quality aspects using hybrid methods. 


\subsubsection{Multi-Objective Composition Optimisation}
EC-based approaches for handling web service composition problems fall into two groups, depending on the different goals of optimisation for either a single objective or multiple objectives. In single-objective service compositions, one composition solution is always returned by a composition task, where the preferences of each quality component within the single objective (e.g., a weighted sum of different quality criteria) is known by users. However, users do not always have clear preferences when many quality criteria are presented. Therefore, multi-objective is a natural requirements from users to provide a set of trade-off solutions that concern about the conflicting and independent quality criteria. E.g., Premium users do not care cost as much as price-sensitive users do, so premium users usually may prefer a composition solution with lowest execution time,  rather than one with a relatively lower execution time without exceeding a budget. Therefore, a multi-objective  fully automated service composition approach is very demanding for providing a set of solutions due to the following two reasons: first, the preferences of different quality is not clear and hard to determine in advance; second, single-objective optimisation using weighted sum method can not reach solutions in the non-convex regions \cite{kim2006adaptive}.  

Existing research on the automated web service composition mainly concentrates on single objective problems for QoS-aware web service compositions. For example, there is only one solution promoted by a unified QoS ranking score to the users. However, to our best knowledge in multi-objective context, works  \cite{liu2005dynamic,wada2012e3,yao2009qos,yin2014hybrid} on service composition problems are only approached by semi-automated methods to handle the conflicting QoS attributes independently, where the workflow structure is assumed to be pre-existing. On the one hand, simultaneously constructing workflows and selecting proper services for optimising multi-objectives is a very challenge work to complete. On the other hand, some constraints on SLA are also employed to some of these approaches to reach the solutions with desirable level. These constrains raised the complexity of absolute Preto priority relation \cite{garey1979guide}. From above discussion, these is a lack of fully automated approaches to multi-objective web service composition problems for QoS-aware web service composition abiding by constrains on SLA. Moreover, the insufficiency of handling only non-functional attributes (i.e., QoS) has given rise to adding semantic matchmaking quality into simultaneous consideration.

\subsubsection{Dynamic Semantic Web Service Composition}
In a dynamic environment, QoS of the atomic services in service repository is fluctuating over time. Static service composition solution is no longer enough, and requisite actions must be taken if the original composition solution changes in QoS or is not be executable due to any service involved goes offline. Apart from that, newly registered services could also could alter the composition plan as it could significantly contribute to the overall QoS or quality of semantic matchmaking. Therefore, dynamic web service composition is proposed to effectively and efficiently update composition solutions when they are not presented as optimal and/or executable solutions \cite{li2014fault}. 


Most approaches work on effective and efficient methods for service re-selection for each services employed, which do not allow the changes of service composition structure. Apart from that, the cost of initial planning is ignored and separated from the adaption of dynamic environment. Some techniques \cite{andrews2003business,baresi2011self,koning2009vxbpel} endeavoured to update outdated or incorrect compositions, and they allow for dynamic adaptation of the solutions based on implementation of variability constructs at the language level. For example, a composition language extending a typical WS-BPEL \cite{andrews2003business} is proposed for supporting the dynamic adaption using ECA (Event Condition Action) rules, which is utilised for guiding the operations for self-reconfiguration. This approach is difficult to manage, and error-prone. Based on and by extending the previous approaches, variability model \cite{alferez2014dynamic} is proposed to support the adaption. On the other hand, EC-based techniques have been showing their confidence in its behaviour for handling dynamic web service composition to overcome the limitations due to the following reasons: a proper amount of individuals stored could be used for retrieving an alternative composition solution in the case of failure for saving computation cost in the initial planning stage; the stored individuals could be further evolved while taking changes into account and leads to either changes on a concrete service or composition structure. These two advantages of using EC support the adaption of a dynamic environment. Existing EC-based approaches to web service composition have been studied in static scenarios, rather than dynamic ones. Although some works \cite{feng2013dynamic,liu2005dynamic} points out their approaches fit the dynamic problems since the natural features of ACO algorithm (i.e., a continuous optimisation process), there is no dynamic problems defined and further discussed in their papers. Therefore, a lack of research in this field. Given above discussion, it is very advisable to study the effectiveness of EC approaches in a dynamic composition context.


\subsubsection{Automated Web Service Composition Based on Preconditions and Effects}
Apart from considering the satisfactory inputs and production of outputs, some conditional constraints also determine the executability of services.  These conditional constraints lead to multiple possible paths for execution when services are composed together, since inputs and outputs are not everything required for service execution. For example, in the scenario of an online book shopping system \cite{wang2014automated}, services are composed to provide an operation for book shopping.  Users expect purchasing outcome (e.g. receipt) returned if books and customer details (e.g. title, author, customer id) are given. In this case, the users may have specific constraints. If the customer has enough money to pay for the book in full amount, then they would like to do so. Otherwise, the customer would like to pay by instalments. Therefore, the constraints on their current account balance needs to be handled during the execution of the service composition.

Most of the existing approaches to automated web service composition are approached through services represented by only inputs and outputs. However, the underlying functional knowledge base of services (i.e., in the form of preconditions and effects) is not covered \cite{paliwal2012semantics}. On the one hand, a few approaches \cite{bansal2016generalized,DBLP:journals/soca/BoustilMS14} have been explored to achieve compositions that consider precondition and effects using AI planning, since AI planning ensures both the correctness of functionality and satisfactory of constrains. Meanwhile, Exhaustive methods are always utilised with AI planning for tackling optimisation problems. These methods suffer hugely in terms of efficiency, scalability, and computation cost. On the other hand, EC techniques (i.e., heuristic methods) are considered to be more flexible and more efficient. Given the benefits from both AI planning and EC-based techniques, they are motivated to be collectively explored for automated web service composition based on preconditions and effects.
 
\section{Research Goals}
The overall goal of this thesis is to \emph{develop new and effective hybrid approaches for comprehensive quality-aware automated semantic web service composition}. More specifically, the focus will be on developing hybrid approaches that could explicitly support a comprehensive quality model that jointly optimises QoS and semantic matchmaking quality using single-objective method, developing multi-objective approaches for optimising the quality criteria that involved in the decision making of composition solutions selection, and developing hybrid composition techniques for dynamic service composition and handling various changes of composition environment, and developing approaches for semantic web service composition, particularly, considering precondition and effects. This research aim to develop a hybridisation of various composition techniques for effectively handling the several service composition problems discussed above. The research goal described above can be achieved by completing the following set of objectives:


\begin{enumerate}
  \item \label{Obj:1} \textbf{Develop hybrid approaches to comprehensive quality-aware automated web service composition that simultaneously optimises both QoS and semantic matchmaking quality}. Particularly, we extend existing works on QoS-aware service composition by considering jointly optimising the both quality aspects, which is proposed as a comprehensive quality model. Meanwhile, representations of the composition solutions are the key aspect of the approaches, and they must maintain all the required information for the evaluation. Therefore, we will investigate the following sub-objectives to handle this objective.
  \begin{enumerate}
    \item \emph{Propose a comprehensive quality model that addresses QoS and semantic matchmaking quality simultaneously with a desirable balance on both sides.} We aim to establish a quality model with a simple calculation and good performance for the evaluation of our proposed comprehensive quality. Meanwhile, to enable a better evaluation on our approaches, it must support most of existing benchmark datasets, e.g., Web Service Challenge 2009 (WSC09)\cite{kona2009wsc} and OWLS-TC \cite{kuster2008opossum}.
    
    \item \emph{Propose direct and indirect solution representations for comprehensive quality-aware web service composition.} Graph-based and tree-based representations are widely used for directly representing service composition solutions. Graph-based representations are capable of presenting all the semantic matchmaking relationships as edges, but hardly supporting some composition constructs (e.g. loop and choice). Tree-based representations could be more ideal for practical use, since they can present all composition constructs as inner nodes of trees. However, they could hardly maintain all the edge-related relationships supported by graphs. To take advantage of the graph-based and tree-based representations, we aim to propose a tree-like representation representation. In particular, any isomorphic copy in the traditional tree-based representations is removed and labeled with a special symbol $q$, and insert an edge to the root of the copy. Meanwhile, particular genetic operators are developed without breaking the functionality of symbol $q$.
    
    The \emph{indirect representations} do not present the final service composition solutions, they must be decoded to executable service composition. Previous studies have shown their good performances in searching optimal solution for QoS-aware web service composition \cite{da2016memetic,da2016particle}. However, the decoding process could increase the computation time. Apart from that, the indirect representation potentially increases the searching space, due to the changes of the indirect representation may result in the same solutions as discussed in the work \cite{da2016particle}. To overcome these disadvantages, it is advisable to propose more efficient indirect representations.
    
    \item \emph{Propose hybrid methods to effectively and efficiently handle the problem for comprehensive quality-aware automated web service composition.} The reasons of utilising hybrid techniques are briefly discussed in the first motivation. Herein, hybrid approaches are suggested to be developed for supporting both the proposed indirect and indirect representations, as well as the comprehensive quality model. In particularly, we aim to propose hybrid heuristics strategies to provide fast convergence of fitness value and avoid being trapped by the local optimal.
  \end{enumerate}
 
 \item \label{Obj:2} \textbf{Develop multi-objective approaches to optimise the comprehensive quality for fully automated service composition}. In practice, many quality criteria proposed in our comprehensive quality model are often conflicting in natural. Existing works \cite{chen2014partial,xiang2014qos,yin2014hybrid,liu2005dynamic,yu2013efficient,zhang2010qos} mainly concentrated on semi-automated QoS-aware web service composition. Therefore, a study needs to be carried out not only for better understanding of inherent trade-offs among different objectives (e.g., quality of semantic matchmaking and QoS are naturally considered as two conflicting objectives), but also for developing fully automated approaches by utilising cutting-edge multi-objective optimisation algorithms (e.g, NSGA-II \cite{deb2002fast}, SPEA2 \cite{zitzler2001spea2} and MOEA/D \cite{zhang2007moea}). These algorithms are needed for finding a Pareto front of evolved solutions that comprehensively cover users' real interests. Meanwhile, different representations may not perform equally well, so a study on improving the performances of different representations with different fully automated approaches also arouses researchers' interest. Apart from that, SLA consideration needs to be taken into account. It is also necessary to consider customised matchmaking levels to bring the flexibility in meeting different requirements of segmented users (e.g., platinum users, gold users and normal users). The development of this approach can be divided into the following three sub-objectives:
   \begin{enumerate}
   
    \item \label{Obj:2.1} \emph{Develop a EC-based approaches for multi-objective fully automated semantic web service composition}. 
    
    Here we develop a multi-objective optimisation approach by using effective multi-objective EC-based algorithms. (e.g, NSGA-II \cite{deb2002fast}, SPEA2 \cite{zitzler2001spea2} and MOEA/D \cite{zhang2007moea}). We will study different representations and useful modifications of multi-objective EC algorithms simultaneously to improve the effectiveness and efficiency of our service composition system. This sub-objective is also established for in-depth investigation of each quality criteria based on our proposed comprehensive quality model in Objective \ref{Obj:1}.  In particular, both quality of semantic matchmaking and QoS must be optimised independently, since they may represent conflicting interests. It would be interesting to examine different tradeoffs among the service composition solutions with respect to the different quality criterion. Apart from that, fully automated approaches are also developed to overcome the limitation (i.e., semi-automated approaches) in existing works assuming an abstract workflow is given .
   
    \item \emph{Develop hybrid approaches for multi-objective fully automated semantic web service composition}. Once we achieve the sub-objective \ref{Obj:2.1}, the effectiveness and efficiency are the next focus. The EC-based approaches should be extended by introducing some local search.  In particular, some efforts could be made for simultaneously considering the improvements on representations themselves and the combinations with a fast local search. For example, an exhaustive search for the neighbourhood of best the individual within the current population is performed with a relatively higher priority for service selection.

    \item \emph{Develop hybrid approaches for multi-objective fully automated semantic web service composition subject to constraints on SLA and customised matchmaking level}. In real world, satisfaction on given SLA constraints is required in addition to optimising QoS. Therefore, this sub-objective should be further extended to consider some additional constraints on  QoS (i.e., multilevel constraints with lower and upper bound for different individual QoS criterion \cite{yin2014hybrid}). Meanwhile, to satisfy the customised different semantic matchmaking levels (e.g., exact matchmaking level and less strict matchmaking level), extensive methods are also required to cope with the constraints on the different accepted matchmaking level.

   \end{enumerate}
   
\vspace{0.5cm}
 \item \textbf{Develop hybrid techniques to support dynamic semantic web service composition effectively}. Objectives \ref{Obj:1} and \ref{Obj:2} are proposed assuming the environment of service composition is static. In our context, composition environment refers to the registered services in the service repository, non-functional attributes advertised by service providers, and the ontology utilised for describing the resources of web services. On the one hand, existing EC-based approaches are only executed once to generate a composition plan from a given composition task, some factors could significantly impact the execution of the plan. For example, QoS values of services being composed of are fluctuating over time, service chosen at the planning stage may not available to be invoked at the runtime, or newly registered service may need to be considered for reconstructing a better plan. On the other hand, existing non-EC based approaches \cite{nasridinov2012qos,salas2006ws,wagner2016robust,yin2010qos} work on effective and efficient methods for service re-selection for repairing each service employed. Technically, their approaches do not allow the changes of service composition structure. Also, the cost of initialising a composition plan is ignored and separated from their approaches. To effectively handle the two limitations above, three studies are performed as three sub-objectives as follows:

  \begin{enumerate}
 \item \label{Obj:3.1} \emph{Develop EC-based techniques to re-optimise solution candidates for changes in QoS and Ontology.} Traditionally, initial population is created with solution candidates that are further evolved for searching optimal solutions. During the evolutionary computation process, most of service candidates are discarded except the best service candidate identified. Those discarded solution candidates may be promising, since some of them could turn to be alternative best due to the changes in services or ontology. Therefore, instead of discarding the solution candidates, we aim to propose a new and effective EC-based approach to re-optimise these maintained candidates for further use since these candidates preserve both diversity and elitism. 

 \item \emph{Develop hybrid techniques to re-optimise solution candidates for changes in QoS and Ontology.} Once the EC-based techniques to re-optimise solution candidates for changes in QoS and Ontology are achieved,  it should be further studied in developing more efficient and effective approach to handle this problem. We aim to propose an adaptive and hybrid approach to this dynamic problem. Our initial idea is to assign a higher priority to a group of services with changes and a lower priority to a group of services without changes, respectively, for considering of service selection based on a local search during the evolutionary process. The higher priority must be adaptively handled with a proper decreasing rate with respect to each service in the first group. We aim to achieve more effective and efficient performance compared to the EC-based approaches in Objective \ref{Obj:3.1}.
 
 
 \item \emph{Develop hybrid techniques for handling service failure and new service registration using updated candidates in the population.} Apart from the changes in the QoS and Ontology, occasionally existing service may fail and/or new service may be registered. For the case of service failure, some methods must be proposed to replace the un-invokable services or update the plan with new services. We aim to propose some approaches using direct representations, where we could either efficiently mutate the solutions candidates partially on un-invokable atomic services, or its involved parent composition components, or effectively re-generate whole solutions using invokable services in the service repository. For the case of new service registration, giving a priority for newly registered services should be properly considered for service selection. We could discard  a portion of the current population (e.g 50 percentage), and then replenish population based on updated services repository.
 
 \end{enumerate}
   
 \item \textbf{Develop hybrid approaches for semantic web service compositions based on preconditions and effects. (Optional)} We plan to extend most service composition approaches (i.e., satisfactory on inputs and outputs) to include preconditions and effects. These conditional constrains also necessitates the study of various of composition constructs for automated semantic web service composition, e.g., loop and choice. Therefore, three sub-objectives have been proposed as our targets as follow.
 \begin{enumerate}
 
  \item \emph{Develop EC-based techniques for semantic web service composition based on preconditions and effects}. In the problem stated above, inputs and outputs are everything of web services for handling some web services. An initial task is required to be completed. That is, service composition problem is re-modelled by further considering the preconditions and effects. In particular, we need to establish a general matchmaking mechanism of satisfaction on preconditions and effects. Based on the mechanism, sequence and parallel composition constructs are automatically constructed. We aim to develop EC-based approaches to effectively handle this problem. In particular, new representations are needed to be proposed for coping with the newly modelled problem.

  \item \emph{Develop hybrid techniques for semantic web service composition based on preconditions and effects}. Once the EC-based techniques to semantic web service composition based on preconditions and effects are proposed. more effective and efficient techniques shall be developed. In particular, we aim to create hybrid techniques that utilise a hybridisation of various techniques for improving the performances of EC-based techniques for semantic web service composition based on preconditions and effects.
    
   \item \emph{Develop EC-based techniques for semantic web service composition based on preconditions and effects for supporting loops and choice}. We initially extend the matchmaking mechanism of satisfaction on preconditions and effects to support loops and choice composition constructs. To extensively cope with these two constructs, new and effective representations must be studied. Apart from that,  EC-based approaches can be developed to effectively solve this problem.

 
 \end{enumerate}
 
\end{enumerate}

\section{Published Papers}

During the initial stage of this research, the preliminary work was carried out on establishing the comprehensive quality model.  Afterwards, some studies on the direct and indirect representations are completed for one part of Objective \ref{Obj:1}, but the earlier works focus on static web service composition using single-objective optimisation technique. The following are the publications made from the preliminary studies:

\begin{itemize}
 \item WANG, C., MA, H., CHEN, A., AND HARTMANN, S. ''Comprehensive Quality-Aware Automated Semantic Web Service Composition``. \textit{AI 2017: Advances in Artificial Intelligence: 30th Australasian Joint Conference}. 2017, pp. 195-207.
 \item Wang, C., Ma, H., Chen, A., Hartmann, S.: ''GP-Based Approach to Comprehensive quality-aware automated semantic web service composition``. In: SEAL2017: International Conference on Simulated Evolution and Learning(To appear)
\end{itemize}


\section{Organisation of Proposal}The remainder of the proposal is organised as follows: Chapter \ref{C:review} provides a fundamental definition of the web service composition problem and performs a literature review covering a range of works in this field; Chapter \ref{C:preliminary} discusses the preliminary work explores direct and indirect representations for comprehensive quality-aware semantic web service composition using a hybridisation of AI planning techniques and EC-based techniques; Chapter \ref{} presents a plan detailing this project's intended contributions, a project timeline, and a thesis outline.

\chapter{Literature Review}\label{C:review}
In this chapter, we first introduce the background knowledge of web service composition in Section \ref{background}, i.e., functional and nonfunctional properties of web service and web service composition in Subsections \ref{service} and \ref{servicecomposition} respectively. Evolutionary Computation techniques are also briefly introduced in Subsection \ref{ec}. Followed that Section \ref{related} reviews the single-objective service composition for both EC and non-EC based approaches in Subsection \ref{singleobjective}.  Subsection  \ref{multiobjective} reviews existing works in multi-objective approaches and many-objective approaches.  Dynamic web service composition is covered in Subsection \ref{dynamicserivce}. Subsection \ref{Semantic}  discussed semantic service composition based on preconditions and postconditions. Lastly, Subsection \ref{summary} summarises some key reviews and limitations in the literature review.
\section{Background}\label{background}

\subsection{Web Service}\label{service}
Web services are self-describing modules offering functionalities over the internet. The functionalities of web services are often specified by their functional attributes, which satisfy users' functional requirements and provide mechanisms to allow users to search desired service. Web services are classified into two groups based on their functionalities:  \emph{information-providing services} and \emph{world-altering services} \cite{mcilraith2001semantic}. The first type of services expect some data returned by giving inputs or nothing. For example, a service for air velocity transducer reads the wind speed and return the velocity at the time. This service does not require any inputs. On the other side, a service for city weather requires given city name to return the weather information for that specific city. Information-providing services do not produce any side effect to the world. The functionalities of these services are only inputs and outputs. The second type of services not only provide data information but also alters the status of the world by producing side effects. For example, a PayPal service will cause a deduction in the balance of users' bank account. \emph{In this proposal, we mainly focus the first type of services for first three objective. Later on, an extensive study is optimally carried on the second type of services}

In realistic scenarios, the non-functional are also important. For example, users may not prefer a service with higher cost with the same functionality provided by another one. As demonstrated above, the functional attributes determine what service really does, while the non-functional attributes often refers to some quality criteria, which is considered for raking services \cite{agarwal2009making}. We first explain web services using a formal model from \cite{agarwal2010d5} below, where both the functional and non-functional attributes are captured uniformly. Further, A Labelled Transition System \cite{agarwal2009making} is addressed with its abstract and updated model for demonstrating the behaviours of web services in Subsection \ref{functional}, which emphasises on side of the functional attributes. After demonstrating these models, we will discuss about the nonfunctional properties of web services in Subsection \ref{nonfunctional}.

\textbf{A Formal Model of Web Service}. Given a finite set $\wp$ of property types of web services and a finite set $\vartheta$ of values sets. Each property type is associated with a value set. We view a Web service as a finite set $Q$ of property instances with each property instance $q \in Q$ being of a property type$t(q) \in \wp$ that is associated with a value $v_q \in \vartheta_{t(q)}$. See Fig. \ref{fig:ws}

\begin{figure}
\centerline{
\fbox{
\includegraphics[width=8cm]{ws.pdf}
}}
\caption{Property-Based Web Service Formal Model \cite{agarwal2010d5}.}
\label{fig:ws}
\end{figure}

\subsubsection{Functional Properties of Web services}\label{functional}

\textbf{A Functionality Model}. This mode is represented as a Labelled Transition System (LTS): $L = (S,T,\rightarrow)$ comprises a set $S$ of states, a set $T$ of transition labels and a labeled transition relation $\rightarrow \subseteq S \times T \times S$. This transition system is established to present the actual behaviour of web services, which consists of a series of states. See Fig. \ref{fig:lts}.

\begin{figure}
\centerline{
\fbox{
\includegraphics[width=14cm]{LTS.pdf}
}}
\caption{ The functionality of a Web Service \cite{agarwal2010d5}.}
\label{fig:lts}
\end{figure}

\begin{itemize}
\item $S_i$ start state includes knowledge available to web service before the web service is invoked by inputs; 
\item $S_w$ state includes the inputs additional to the knowledge in 1; 
\item A series of state $S_1$ to $S_n$ proceed with corresponding actions $f_n$ that only occurred if each related condition $[w_i]$ is approved to be true. 
\item $S_o$ state contains all the outputs and all the changes performed in the knowledge base. 
\item $S_e$ is the end state, which is equivalent to $S_o$, since the knowledge base is not changed.
\end{itemize}

\textbf{An Abstract Functionality Model}. The first functionality model presented above is not completely demonstrated for the internal actions, as the services provider does not want reveal all the internal functions, and it is not feasible to list a global set of property name. Therefore, an abstract functionality of a web service is modelled by eliminating all the intermediate properties. In the abstract model of a web service, the functional properties of the web service could be identified as inputs $i$, pre-state $S_i$, outputs $o$ and post-state $S_o$. These four properties are mapped to a set of input $I$, preconditions $\phi$, a set of outputs $O$ and postcondition $\varphi$ respectively in third updated functionality model demonstrated below.

\textbf{A Updated functionality Model}. In the updated model, a set of inputs $I$ is required by a service and a set of output $O$ is returned after the successful execution. Apart from that, the precondition $\phi$ must be hold in the knowledge base before service is invoked by passing the input $I$. To enable the interoperability of the functional properties, ontology reasoner is employed to reason about the properties of web services. To distinguish the changes between before and after service execution, these changes are modelled as property instances. For example, inputs and outputs assigned as variable names and further referenced in preconditions and postconditions, which can be distinguished as different instances from the knowledge base. In particular, preconditions is assigned to the description of requirements of inputs using logic formula. Herein the description of the formula considers the following cases that are demonstrated using Planning Domain Description Language (PDDL \cite{fox2003pddl2}):
\begin{itemize}
\item Conditions on the type of inputs, e.g., the payment of an online shopping website is made by Visa or Cash: $\phi:(Format(payment)=Visa \cup Format(payment)=Cash)$.
\item Relationships among inputs, e.g.,  authoried users are required for an online shopping: $\phi:(Authoried ? Useraccount)$.
\item Conditions on the value of inputs, e.g., saving account balance has more than 100 dollars: $\phi: (\geq (amounts, saving account), 100)$
\end{itemize}
These preconditions must be hold in the state consistently while inputs are being passed to services. Similarly, the postcondition is restricted to the description of constraints on returned outputs, relationships between inputs and outputs, and changes caused by the service in the knowledge bases.

\subsubsection{Nonfunctional Properties of Web services}\label{nonfunctional}
Apart from the functional properties of web services discussed above, the non-functional properties of web services play an important part in composing services. For example, customers prefer lowest execution cost with highest response time and reliability. According to \cite{zeng2003quality}, four most often considered QoS parameters are as follows:
\begin{itemize}
\item \textit{Response time} ($T$) measures the expected delay in seconds between the moment when a request is sent and the moment when the results are received.
\item \textit{Cost} ($C$) is the amount of money that a service requester has to pay for executing the web service
\item \textit{Reliability} ($R$) is the probability that a request is correctly responded within the maximum expected time frame.
\item \textit{Availability} ($A$) is the probability that a web service is accessible.
\end{itemize}

\subsubsection{Web Service Discovery}\label{servicediscovery}
To generate service compositions, web service must provide mechanisms for discovery required services. Therefore, service discovery must be one fundamental technique to be considered in all service composition approaches. \cite{agarwal2009d5} discussed three mechanisms of semantic web service discovery: classification-based approach, functionality-based approach and hybrid approach. Those service discovery techniques are further demonstrated below.

The first service discovery technique makes use of the classes provided by service semantic annotation in WSMO-Lite language. Therefore, service requesters can use class names to express a goal offering a straightforward discovery from a set of classes. However, classes without clear meaning definition could lead to the issue of incomprehensibility of web service discovery. For example, several classes may declared in either different terms for the describing the same content or same terms for describing different content.

The second service discovery technique  does not take classes into account, but consider functional properties of web service to include pre-conditions and post-conditions. In particular, a desired functionality description is defined. A discovery algorithm must be developed to handle a matching for different input, output, precondition and postcondition with associated concepts and relations in the provided domain knowledge base. The key idea of the matchmaking is to check whether services accept all the desired inputs provided bu users and whether the desired outputs is delivered by services. In addition, the matchmaking algorithm also checks for the satisfiability of implications that actual precondition and actual postcondition must imply the desired precondition and desired postcondition respectively. The strength of the second is that it potentially meet the demands of all the comprehensible discovery, while the weakness is a lack of efficiency and scalability. 

The third service discovery technique is based on a hybridisation of classification and functionality-based discovery. Classification hierarchy is proposed to achieve automatic semantic reasoning in hierarchical functionality. For example, a functionality class is associated with super classes and sub classes for more general functionality class and more specific functionality class respectively. However, to make a consistency of classification hierarchy, the inputs, outputs, precondition and postcondition of a functionality class must satisfy the conditions that contains all the inputs, outputs, precondition and postcondition of all the classes it is subsumed by. The advantage of this approach is to achieve better performance combining strengths of the previous two pure classifications based and functionality based approaches. While the classification hierarchy needs to be kept consistent when a new web service is published or updated.​

As discussed above, the first and third approach is considered either less effective or demands to build up a consistent ontology for classes and their functional attributes. That is not the focus of our research.  \emph{In the proposal, we use the second service discovery technique for meeting the comprehensible discovery. That is, different types of ontology reasoning are utilised to approach the matchmaking as a fundamental part of service composition algorithm.}



\subsection{Web Service Composition}\label{servicecomposition}

Since one atomic web service could not satisfy or fully satisfy users' complex requirements, web service composition is approached by composing web services together with more sophisticated functionalities to meet the demand. Due to fully human intervention in manual service composition, so it is very time-consuming and less productive. Therefore, many approaches have been developed to achieve semi-automated or fully automated service composition. The semi-automated service composition is inspired by the business process that required prior knowledge to build up the abstract workflow. This workflow can be decomposed into several functional services slots for proper services being fitted. These steps are further discussed in Subsection \ref{lifecycle}. On the other hand, when we are composing services, both semi-automated and fully automated web service composition holds the challenge in the interoperability of services discussed previously. In particular, several problems are simultaneously taken into account. That is I/O matchmaking (i.e., the mechanisms of services for ensuring the interoperability ), discovery relevant services to the optimising the quality of service composition, e.g., overall QoS. Consequently, the following concerns are required to consider in generating composition solution. 


\subsubsection{Web Service Composition Lifecycle}\label{lifecycle}
Typical steps in a workflow-based automated Web service composition solution are shown in Fig. \ref{fig:lifecycle}. The detail of the service lifecycle is discussed as follows:

\begin{figure}
\centerline{
\fbox{
\includegraphics[width=14cm]{compositionLifecycle.pdf}
}}
\caption{ Web service composition lifecycle \cite{moghaddam2014service}.}
\label{fig:lifecycle}
\end{figure}

\begin{enumerate}
 \item \textit{Goal specification:} The first step in service composition is to collect users' requirements for composition goal that comprises of the functional (i.e., correct data flow ) and non-functional side (i.e., QoS). This step is achieved by building up an abstract workflow including a series of tasks with clearly defined functionality. Those tasks could be completed by selecting proper concrete services to reach desired QoS. 
 \item \textit{Service discovery:} Once the goal is clearly specified, concrete web services are to be selected for each task regarding its functional requirement. Often, more than one concrete web service is likely to be found to match the one task. However,  those matched web services are always different in QoS.  Therefore, web services are classified by the functionality of each task, i.e., inputs and outputs.
 \item \textit{Service selection:} At this stage, many techniques have been studied to select web services to best match each task for the satisfying functional requirement of each task and overall business process. Therefore, a plan of service composition is created ahead of execution.

 \item \textit{Service execution:} the process instance is monitored for any changes or services failures during service execution. In this stage, some actions are to be taken for adapting the changes.
\end{enumerate}
The web service lifecycle discussed above is a typical \emph{semi-automated approach}. There is a distinction between semi-automated and fully automated approaches. On the one hand, during the goal specification stage of semi-automated approach, the abstract work is already provided.  On the other hand, an abstract workflow is not provided at the stage of goal specification for \emph{fully automated service composition}. Often, fully automated service composition rely on some algorithms (e.g., Graphplan algorithm \cite{blum1997fast}) to achieve service composition, during which service workflow is gradually built up along with the service discovery and service selection at the same time. In this proposal, I concentrate on developing methods for fully automated service composition since it has been shown the flexibility. Also, herein service discovery and service selection are considered as interrelated tasks that are interleaved with the composition algorithm.

\subsubsection{Functional Properties of Web Service Composition}
Substantial work \cite{bansal2016generalized,lecue2009optimizing, lecue2007making, lecue2006formal, rao2005semantic} on semantic web service composition utilises Description logic (DL) reasoning between input and output parameters of web services for matchmaking. OWL and OWL-S are the most common semantic specifications used currently \cite{petrie2016web}, and they enable automatic selection, composition, and interoperation of Web services to implement complicated composition tasks \cite{martin2004owl}. However, some matchmaking types (discussed below) may penalise the matchmaking quality and are less preferred by users. Therefore, exploring a effective mechanism for measuring the quality of semantic matchmaking in service composition is a very demanding research area.

Given two concepts $a, b$ in ontology $\mathcal{O}$, four commonly used \emph{Matchmaking types} are often used to describe the level of a match between outputs and inputs \cite{paolucci2002semantic}: 
\begin{itemize}
\item the \emph{matchmaking} returns $exact$ if $a$ and $b$ are equivalent ($a \equiv b$), 
\item the \emph{matchmaking} returns $plugin$ if $a$ is a sub-concept of $b$ ($a \sqsubseteq b$),
\item the \emph{matchmaking} returns $subsume$ if $a$ is a super-concept of $b$ ($a \sqsupseteq b$), 
\item the \emph{matchmaking} returns $fail$ if none of previous matchmaking types is returned. 
\end{itemize}

Often, the similarity of two instances of two knowledge representations encoded in the same ontology is also utilised to measure the quality of matchmaking regarding the four matchmaking types discussed above. The work \cite{lecue2007making} additionally consider $interaction$ matchmaking type ($a \sqcap b$), i.e., if the intersection of $a$ and $b$ is satisfiable. In their work, a causal link link \begin{math} sl_{i,j} \stackrel{.}{=} \langle S_i, Sim_{T}(Out\_s_i,In\_s_j),S_j  \rangle \end{math} is created between two functional property instances for a input and a output. In particular, both $exact$ match and $plugin$ match are presented as robust causal links, while both $subsume$ match and $intersection$ match are presented as valid casual links. However, valid casual links are not specific enough to be utilised as the input of another web service. Thus the output requires Extra Description Equation \ref{equation2} to enable proper service composition. As a result, Subsume and Intersection matching type is transferred to be Exact and PlugIn respectively to formulate a robust link. 

\begin{equation}
In\_s_x \setminus Out\_s_y \stackrel{.}{=} \underset {\preceq d}{min} \{ B|B\sqcap  Out\_s_y \equiv In\_s_x  \} , since \  Out\_s_y \sqsupseteq In\_s_x
 \label{equation2}
\end{equation}



\emph{In this paper we are only interested in robust compositions where only $exact$ and $plugin$ matches are considered, see \cite{lecue2009optimizing}. As argued in \cite{lecue2009optimizing} $plugin$ matches are less preferable than $exact$ matches due to the overheads associated with data processing. We suggest to consider the semantic similarity of concepts when comparing different $plugin$ matches.} Herein we demonstrate an example of a robust causal link by between two matched services $S$ and $S'$, noted as $S \rightarrow S'$, if an output $a$ ($a \in {O_S}$) of $S$ serves as the input $b$ ($b \in {O_{S'}}$) of $S'$ satisfying either $a \equiv b$ or $a \sqsubseteq b$.  For concepts $a, b$ in $\mathcal{O}$ the \emph{semantic similarity} $sim(a, b)$ is calculated based on the edge counting method in a taxonomy like WorldNet or Ontology \cite{shet2012new}. This method has the advantages of simple calculation and good performance \cite{shet2012new}. Therefore, the \emph{matchmaking type} and \emph{semantic similarity} of a robust causal link can be defined as follow:

\begin{align}
\label{eq_link}
type_{link} = 
\begin{cases}
	1 & \text{ if $a\equiv b$ ($exact$ match)}\\
	p & \text{ if $a \sqsubseteq b$ ($plugin$ match)}
\end{cases}
,&&
sim_{link} = sim(a,b) = \frac{2N_c}{N_{a}+N_{b}}
\end{align}

\noindent with a suitable parameter $p, 0<p< 1$, and with $N_a$, $N_b$ and $N_c$, which measure the distances from concept $a$, concept $b$, and the closest common ancestor $c$ of $a$ and $b$ to the top concept of the ontology $\mathcal{O}$, respectively. However, if more than one pair of matched output and input exist from service $S$ to service $S'$, $type_e$ and $sim_e$ will take on their average values.

The \emph{semantic matchmaking quality} of the service composition can be obtained by aggregating over all robust causal links as follow:
\begin{align}
MT {=} \prod_{j=1}^{m} type_ {link_{j}}
,&&
SIM {=} \frac{1}{m}\sum_{j=1}^m sim_ {link_{j}}  
\end{align}


\subsubsection{Nonfunctional Properties of Web Service Composition}
The nonfunctional properties of web service composition is determined by all the QoS of involved concrete web services in the solution. The aggregation value of QoS attributes for web services composition varies with respect to different constructs, which reflects how services associated with each other in a service composition \cite{zeng2003quality}.

We use formal expressions as in \cite{ma2012formal} to represent service compositions. We use the constructors $\bullet$, $\parallel$, $+$ and $\ast$ to denote sequential composition, parallel composition, choice, and iteration, respectively. The set of \emph{composite service expressions} is the smallest collection $\mathcal{SC}$ that contains all atomic services and that is closed under sequential composition, parallel composition, choice, and iteration. That is, whenever $\cse_0,\cse_1,\ldots,\cse_d$ are in $\mathcal{SC}$ then $\bullet(\cse_1,\ldots,\cse_d)$, $\parallel(\cse_1,\ldots,\cse_d)$, $+(\cse_1,\ldots,\cse_d)$, and $\ast \cse_0$ are in $\mathcal{SC}$, too. Let $\cse$ be a composite service expression. If $\cse$ denotes an atomic service $S$ then its QoS is given by $QoS_S$.  Otherwise the QoS for $\cse$ can be obtained inductively as summarized in Table~\ref{tbl:QoS_Aggre}. Herein, $p_1,\ldots,p_d$ with $\sum\limits^d_{k=1}p_k=1$ denote the probabilities of the different options of the choice $+$, while $\ell$ denotes the average number of iterations.

\begin{table}[htb]
\centering
\caption{QoS calculation for a composite service expression $\cse$}
\begin{tabular}{l|l|l|l|l}
\hline
 $\cse=$       &$r_\cse=$                              &$a_\cse=$                              &$c_\cse=$                            &$t_\cse=$ \\ \hline
 $\bullet(\cse_1,\ldots,\cse_d)$      &$\prod\limits^d_{k=1}r_{\cse_k}$    &$\prod\limits^d_{k=1}a_{\cse_k}$    &$\sum\limits^d_{k=1}c_{\cse_k}$   &$\sum\limits^d_{k=1}t_{\cse_k}$  \\ \hline
 $\parallel(\cse_1,\ldots,\cse_d)$  &$\prod\limits^d_{k=1}r_{\cse_k}$    &$\prod\limits^d_{k=1}a_{\cse_k}$    &$\sum\limits^d_{k=1}c_{\cse_k}$   &$MAX \{ t_{\cse_k} | k \in \{ 1,...,d \} \}$\\ \hline
 $+(\cse_1,\ldots,\cse_d)$     &$\prod\limits^d_{k=1}p_k\cdot r_{\cse_k}$    &$\prod\limits^d_{k=1}p_k\cdot a_{\cse_k}$    &$\sum\limits^d_{k=1}p_k\cdot c_{\cse_k}$   &$\sum\limits^d_{k=1}p_k\cdot t_{\cse_k}$  \\ \hline
 $\ast \cse_0$         &${r_{\cse_0}}^\ell$  &${a_{\cse_0}}^\ell$  &$\ell\cdot c_{\cse_0}$ &$\ell\cdot t_{\cse_0}$ \\ \hline
\end{tabular}
\label{tbl:QoS_Aggre}
\end{table}

\subsection{Evolutionary Computation Techniques Overview}\label{ec}

Evolution Computing (EC) techniques are founded based on the principles of Darwin natural selection. The nature evolution and selection of individual in a population are automated simulated in EC. In particular, a population of individuals is initialised for directly or indirectly presenting the solutions. Those individual candidates are evolved and evaluated using a fitness function to evaluate the degree of how good (or bad) of each individual. Therefore, it is possible to reach solution with near-optimal fitness. EC have been shown its promise in solving combinatorial optimisation problems \cite{back1997evolutionary}. This is due to its flexibility in encoding the problems for the representation and its good performance in many scenarios. In particular, To manage the constraints in the problems, five main methods have been proposed deal with the constraints: coding, penalty functions, repair algorithm, indirect methods of representation and multi-objective optimisation \cite{fleming2002evolutionary}. In the context of service composition. Many EC techniques have been approached for handling optimisation problems for web service composition, such as Genetic Algorithms (GA) \cite{whitley1994genetic}, Genetic Programming (GP) \cite{koza1992genetic}, Particle Swarm Optimisation (PSO) \cite{kennedy1995particle}, and Clonal Section Algorithm \cite{de2002learning}. These techniques are briefly introduced here.

GP is considered as a particular application of GA with a set of different encoded genes. In particular, the representation of GA is commonly represented as a linear structure. However, In GP, each individual is commonly represented as a tree structure. the tree structure has a terminal set and a function set, where variables, constants and functions are consisted of respectively. Also,  the tree structure is considered be efficiently evaluated recursively. Three genetic operators consisting of reproduction, crossover, and mutation are involved in to generate next generation for both GA and GP. Reproduction operator retains the elite individual without any changes. Crossover operator replaces one node of one individual with another node of another individual. Mutation operator replaces a randomly selected node in an individual. The whole evaluation process won't stop unless an optimised solution found or a pre-defined number of generation reached.

PSO is one of swarm intelligence (SI) that based on the behaviour of decentralised, self organised system. PSO algorithm is initialised by a group of random particles, which direct or indirect present the solutions. Those particles explores for the optimisation position, which is approached by repeating the process of transferring particles position according to both their own best-known position and global best position.

Artificial immune system (AIS) has been studied for performing machine learning, pattern recognition and solving optimisation problems.  Clonal Section Algorithm (CSA) is one of AIS for handling optimisation problems, and the principle of utilising Clonal Section Algorithm lie in the features of immune memory, affinity maturation. In particular, the antigen is considered as a fitness function instead of the explicit antigen population, and a proportion of antibody, rather than the best affinity antibody, are chosen to proliferation. Further more, speed and accuracy of the immune response grow higher and higher after each infection even confronting cross-reactive response. Apart from that, hypermutation and receptor editing contribute to avoiding local optimisation and selecting optimised solution respectively. 


\section{Related Work}\label{related}

\subsection{Single-Objective Web Service Composition Approaches}\label{singleobjective}

In this section, approaches to QoS-aware web service composition is to be discussed in two distinct groups: EC-based approaches, which mainly rely on the EC techniques to reach the optimal solutions, and Non-EC based methods, which do not utilise any bio-inspired methods. However, most of these approaches employ a single-objective fitness function for optimising a united QoS score as a simple Additive Weighting (SAW) technique \cite{hwang1981lecture}.
\subsubsection{EC-based Composition Approaches}
EC-based web service composition mainly relies on evolutionary computation algorithms for searching optimal solutions. These algorithms are inspired by the behaviour of human, animals or even T-cells.  To cope with different EC algorithms, proper representations are to be designed for direct or indirect represent the service composition solutions. Herein we mainly discuss some promising research works on QoS-aware web service composition using  Genetic Algorithm (GA), Genetic Programming (GP), Particle Swarm Optimisation (PSO), and Clonal Selection Algorithm (CSA).

\textbf{Genetic Algorithm}. 
GA is a very reliable and powerful technique for solving combinatorial optimisation problems \cite{srinivas1994genetic}. It has been applied to handle optimisation problems for QoS-aware web service composition \cite{wang2012survey}. \cite{canfora2005approach} developed a GA-based approach for semi-automated QoS-aware service composition, where an abstract workflow is given. In their work, GA methods are compared to linear integer programming. The experimental finding reveals GA method is preferred when the size of service candidates are increasing.  \cite{tang2010hybrid} proposed a hybrid approach utilising GA and local search. In particular, a local optimiser is developed and only recalled in the initial population for improving QoS value. This local search contributes to a better overall performance compared with GA-methods without local search. In \cite{lecue2009optimizing},  a semi-automated service composition approach is developed in their paper for optimising the quality of semantic matchmaking and some quality criteria of QoS. In particular, the quality of matchmaking problem is transferred to measure the quality of semantic links, which is proposed by two quality aspects: matchmaking type and degree of similarity.

\textbf{Genetic Programming}.
Tree-based representations could be more ideal for practical use, since they can present all composition constructs as inner nodes of trees. GP technique is utilised for handling tree-based representations. \cite {rodriguez2010composition} relies on GP utilising a context-free grammar for population initialisation, and uses a fitness function to penalise invalid individuals throughout evolutionary process. This method is considered to be less efficient as it represents a low rate of fitness convergence. To overcome the disadvantages of \cite {rodriguez2010composition}, \cite{yu2013adaptive} proposes a GP-based approach employing the standard GP to bypass the low rate of convergence and premature convergence. The idea of this paper is to increase the mutation rate while encountering low diversity in the population and adopt a higher crossover probability while trapped in local optimisation.  During the evolutionary process, the elitism strategy is adopted, in which the best individual produced is reproduced to next generation directly without crossover and mutation. \cite{ma2015hybrid} proposes a hybrid approach combining GP and a greedy algorithm. In particular, a set of DAGs that represent valid solutions are initialised by a random greedy search and transferred into trees using the graph unfolding technique.  In each individual,  terminal nodes are considered as task inputs,  root node as  outputs, and all the inner nodes as atomic web services. During the reproduction process,  a randomly selected node on one individual is replaced with a new subtree generated by a greedy search to perform mutation while same atomic inner nodes in two random chosen individuals are swapped to perform crossover. However, \cite{da2016genetic} proposes a different transformation algorithm to present composition constructs as the functional nodes of trees. On the whole, all these GP-based approaches \cite{ma2015hybrid,rodriguez2010composition,da2016genetic,yu2013adaptive} consistenly ignore the semantic matchmaking quality, and their representations do not preserve semantic matchmaking information and composition constructs simultaneously. 

\textbf{Graph-Based Genetic Programming}.
A graph-evolutionary approach is introduced in \cite{da2015graphevol} with graph-based genetic operators, which is utilised to evolve graph-based representation. Although graph-based representations are capable of presenting all the matchmaking relationships as edges, they hardly present some composition constructs (e.g. loop and choice). Another paper \cite{da2016handling} investigated Directed Acyclic Graph with branches using GraphEvol approach \cite{da2015graphevol} to find near-optimal QoS solution in web service composition comparing with GP approach in \cite{da2015gp}. The experiment results reveal a significant improvement in execution time while slightly tradeoff in the fitness value. However, the service composition problem for handling branches is not generally formulated, i.e, only works for one choice construct. If more one nested choice constructs, their approach does not work any more.

\textbf{Particle Swarm Optimisation}.
PSO is considered to a simple and effective approach for solving combinatorial optimisation problems with few parameters settings \cite{long2009environment}. The paper \cite{long2009environment} proposed an environmental-aware PSO approach for QoS-aware web service composition. In particular, an improved discrete PSO algorithm is developed for adapt the changes of composition environment (i.e., services) when a same service composition request is called more than one time. The paper \cite{liu2007hybrid} proposed a hybrid Genetic Particle Swarm Optimisation Algorithm (GPSA). In their work, GA is employed with only crossover operator to produce new individuals with $n_1$ iterations while PSO is only utilised for local searching (i.e., $C_2$ parameter is set to 0 in the standard velocity updating functional) with $n_2$ iterations. This approach achieves a good balance of global and local optimisation through a  mechanism based on two thresholds, which determine the values of $n_1$ and $n_2$. However, \cite{liu2007hybrid,long2009environment} handles semi-automated service composition problems. On the other hand, \cite{da2016particle} proposes a PSO-based fully automated approach to generate a composition graph from a queue. The idea is to translate the particle location into a service queue as an indirect representation of composition  graph, so finding the best fitness of the composite graph is to discover the optimised location of the particle in the search space. \cite{da2016particle} proposes a PSO-based fully automated approach to generate a composition graph from an indirect representation, i.e., a service queue. This service queue is mapped to particles' locations.  so finding the best fitness of the composite graph is to discover the optimised location of the particle in the search space. In particular,  the dimension of the particle is set up as the same number as relevant web services, and the index of services is mapped to the location vectors in a particle and put services in a queue in ascending order, from which a graph is decoded using a forward GraphPlan Algorithm. 

\textbf{Clonal Selection Algorithm}.
The paper \cite{yan2006immune} introduces a novel web service composition approach using an immune algorithm for global optimisation considering optimum time under a constraint on cost. As a given abstract graph could be broken into several single pipelines, the optimisation problem is transferred into getting the optimum executing plan for the single pipelines. In pipelines, each involved tasked could be slotted with several alternative web services with QoS values labelled to their edges so that a weighted multistage graph is established for further longest path selection. In the immune algorithm, the service composition problem is encoded using a binary string as an antibody for evaluating the affinity value regarding the antigen ( fitness function ), and the antibody with low concentration will be selected in a high probability of crossover and mutation for new antibody generation. However, the efficiency of creating the weighted multistage graph would be considered to be less efficient. The paper \cite{pop2009immune} introduced an immune-inspired web service composition approach combining an enhancing planning graph (EPG) and a clonal selection algorithm to solve optimisation problem considering both semantic link quality and QoS.  The EPG model is characterised with action and Layer involved in multiple stages, where each action represents clustered web service, and each layer represents input or output parameters grouped in concepts.   During the clonal selection process, the antigen is represented as a fitness function, and the antibody is represented as a binary alphabet to encode EPG.  The remaining steps are standard computation procedure for CLONALG consisting of generating clones from selected antibody, affinity maturation process and replacing low-affinity antibody and re-select antibody to continue all the whole procedure. At last, the approach is proved to reach an optimal solution or a near-optimal solution in the experiments under trip and social event attendance planning domains.

\subsubsection{AI Based Approaches}

AI Planning techniques have been widely employed for service composition \cite{markou2015non,peer2005web}. The main idea of these techniques considers services as actions that are defined with functional properties ( i.e., inputs, outputs, preconditions and postconditions) to generate validate service compositions using classic planning algorithms. 

Various AI planning approaches \cite{feng2013dynamic,huang2009effective,rao2006mixed,wang2013genetic, wang2014automated} have been presented to solve semantic web service composition problems using the Graphplan \cite{blum1997fast} algorithm. \cite{wang2014automated} employs the Graphplan to secure the correctness of overall functionality, which enables atomic web services to be concretely selected and accurately matched for achieving desired functionality. In particular, conditional branch structure is also correctly handled. The pitfalls of this approach are procuring only linear sequences of actions, and it is hard to deal with QoS optimisation. In paper \cite{feng2013dynamic}, service-dependent QoS is modelled and considered for QoS-aware web service composition. This dependent QoS model is formed in three cases: a default QoS attribute, a partially dependent QoS attribute, and a completely dependent QoS attribute, and they are used for the dependency checking base on a backwards Graph building with a breadth-first strategy. However, computation of service dependencies is very intensive for initialization and updating. Some approaches \cite{lecue2007making,sohrabi2009web} rely some frameworks supported by particular agent programming languages (e.g., Golog \cite{sohrabi2009web}  and SHOP2 \cite{sirin2004htn}) to composite web services. In \cite{sohrabi2009web}, a service composition framework supported by Golog. Golog is used to present the generic procedure, and situation calculus and first order language (FOL) are used to describe the properties of services and users' preferences. Therefore, Golog can effectively perform a constant search to reach a terminating situation as a service composition solution.

As summarised here, given the desired solutions generated to meet users' complex requirements, AI planning techniques are considered to be less efficient, and not capable of dealing with optimisation solutions for service compostion (e.g., generate either optimal QoS or number of services ) independently. In addition, they may suffer scalability issues when large repositories are given. 

\subsubsection{Other Approaches}
The non-EC based approaches do not rely on bio-inspired approaches. They target the optimised service composition solutions by some other methods. For example, integer programming, exhaustive search, local search and so on.

\textbf{Integer Linear Programming (ILP)}. ILP methods are utilised for achieving web service composition. Generally, an ILP model is created with three inputs provided: a set of decision variables, an objective function and a set of constraints. On the other hand, the outputs are maximised/minimise objective function and values of decision variables. Therefore, ILP is flexible in taking QoS into account, handling constraints for QoS and optimising the objective function for QoS-aware service composition problem. \cite{gao2005web} define a zero-one IP model for web service composition based on an abstract service workflow, where services may be different in QoS, but classified into the same classes.  \cite{yoo2008web} formulated web service composition problem based on the model introduced in \cite{gao2005web}. Apart from that, compared with the previous work, they simultaneously take both QoS and constraints on QoS into account. However, on the one hand, due to the increase in the number of decision variables, ILP may lead to exponentially increase the complexity and cost in computation \cite{li2016full}. The resulted huge delay is not allowed in the real world scenarios. In addition, if non-linear function is utilised, the scalability is a big problem.
 
\textbf{Dynamic Programming Approach}. Dynamic programming is an effective method for solving problems, where many repetitions of their break-down subproblems and optimal substructures are presented. In  \cite{huang2009effective}, an efficient pruning approach is developed including a forward filtering algorithm for searching task related service candidates, a modified dynamic programming approach for dealing a subproblem of service composition (i.e.,  a  problem on satisfactory of each concept pool of each graph layer ), and a backward-search method for searching optimal composition results. This paper \cite{xu2012towards} forges a problem to solve large-scale service composition efficiently with QoS guarantee, where a dynamic programming algorithm named QDA is developed to for ensuring the optimisation of the composition problem. The problem is optimised by optimising every subproblem based consideration of web service composition with fewest services involved in.  In particularly, best-known QoS are recorded and updated for all added web services, and service parameters by the maximum of the path using the traceback depth- first search to derive an execution plan. However, the global best QoS could be never reached since a trade-off in the efficiency of QoS guarantee in solutions.

\textbf{ER Model-Based Approach (ILP)}. Most of the small and medium business rely on ER database to process information and data. \cite{xu2010semantic} employs ER model to construct domain ontology and semantic web services.  It potentially benefits large groups of organisations. With regard to the ontology construction, it realises the transformation from ER to OWL-DL, which has maximum expressiveness while retaining computational completeness and decidability. Also,  it constructs semantic web service described by OWL-S  from ER.  Therefore, the semantic web service composition problem is transferred to reason composite service based on a link path between entities in the ER model corresponding to the classes.  However, other constructs such as loop and switch constructs can not be effectively expressed in their approach, which demands many further research.

\subsection{Multi-objective, and Many-objective Composition Approaches}\label{multiobjective}
%----------------------chen starts---------------------------------------------------
Maximising or minimising a single objective function is a most commonly used way to handle optimising problems in automated web service composition.  That is a Simple Additive Weighting (SAW) \cite{hwang1981lecture} technique, which presents a utility function for all the individual quality criteria as a whole. This technique optimises and ranks each web service composition using a single value for each solution. However,  the limitation of this technique lies in not handing the conflicting quality criteria.  Those conflicting quality criteria are always presented trade-offs. To overcome this limitation, a set of objectives corresponding to different independent quality criteria are optimised independently. Consequently,  a set of promising solutions that present many quality criteria trade-offs are returned.


\subsubsection{Multi-objective approaches}\label{MultiObjective}

Many multi-objective techniques \cite{liu2005dynamic,zhang2010qos,yu2013efficient,yin2014hybrid,xiang2014qos,chen2014partial} have been investigated to extensively study QoS-aware web service composition problems.  A set of optimised solutions is ranked based on a set of independent objectives, i.e., different QoS attributes. In particular, solutions are compared according to their relationship for domination. Especially, figuring out solutions are clearly dominating the others. For example, given two service composition solutions that are compared based on execution cost $c$ and execution time $t$, solution one, $wsc_1(c=10,t=1)$ and solution two,  $wsc_2(c=13,t=1)$. In our context, $wsc_1$ dominate $wsc_2$ as $wsc_1$ has the same execution time and a lower execution cost. If given $wsc_3(c=10,t=2)$, $wsc_2$ is a \textit{non-dominant} solution in the relation to $wsc_3$ because of its longer execution time and cheaper execution cost. Therefore,  If those non-dominant solutions are globally produced among both the dominant and non-dominant solutions, i.e., they do not dominate themselves. These solutions are called a \textit{Pareto front}, which provide a set of non-dominant solutions for users to choose.


\textbf{Multi-objective techniques with GA} Many approaches to multi-objective Web service employs GA \cite{liu2005dynamic}, but other EC algorithms are also considered. For GA, \cite{liu2005dynamic} employs a service composition model, called  MCOOP (i.e., muti-constraint and multi-objective optimal path) as web service composition solution for only a sequence service composition considered in the paper. In this model, different paths are selected from a service composition graph that includes $N$ service group. In each group,  services present same functionality with different QoS.  Apart from that, GPDSS is proposed to generate the outputs of Pareto optimal composition paths. In particular, two points crossover and mutation are applied to speed up the astringency of this algorithm. The work \cite{wada2012e3} investigates a semi-automated approach to SLA-aware web service composition problem.  Each linear representation proposed here presents three service composition solutions designed for three group users' categories.  The individuals are randomly initialised, evaluated and optimised with objectives from all the possible combinations of throughput, latency, cost and user category.  In this work, two multi-objective genetic algorithms: E-MOGA and Extreme-E are developed. E-MOGA is proposed to search a set of solutions that equally distributed in the searching space by the means of fitness function, where the production of domination value,  Manhattan distance to the worst point and sparsity (i.e, Manhattan distance to the closest neighbour individual)  is assigned to the feasible individual as fitness value, and SLA violation /domination value is assigned to the infeasible solutions. On the other hand, Extrem-E provide extreme solutions by employing fitness functional, where weights use a term 1/exp(p-1), where $p$ is the number of objectives and is assigned to the $p^{th}$ objective.

\textbf{Multi-objective techniques with PSO} The work \cite{yin2014hybrid} combines genetic operators and particle swarm optimisation algorithm together to tackle the multi-objective SLA-aware web service composition problems. The method proposed in the paper  is considered to be more effective in  considering different scare of cases.  It is called as HMDPSO, i.e., hybrid multi-objective discrete particle swarm optimisation. In particular, the updates of particle's velocity and position are achieved by the crossover operator, where both velocity and position of new individual are updated in accordance with positions of \textit{pbest}, \textit{gbest}, and current velocity. On the other hand, mutation strategy is introduced to increase the diversity of particle and is performed on the \textit{gbest} particle if the proposed swarm diversity indicator is below some value. For the evaluation,  the fitness values of individuals are assigned in the same way as the E-MOGA method in \cite{wada2012e3}.

\textbf{Multi-objective techniques with ACO} Generally, ACO simulates foraging behaviours of a group of ants for optimising the traversed foraging path, where the strength of pheromones is taken account for. The work \cite{zhang2010qos} turns the service composition problem into path selection problem for the given abstract workflow with different service candidate set.  It employs a different strategy of "divide and conquer`` for decomposing a given workflow. That is,  two or more abstract execution paths are decomposed from the workflow and have no overlapped abstract services. This decomposing strategy results in a much smaller length of the execution paths compared to those in the works \cite{yu2007efficient}.  Also, a new ACO algorithm is proposed to handle the multi-objective QoS-aware service composition problem. In particular,  the phenomenon is presented as a k-tuple for $k$ objectives, rather than a single value. Apart from that, a different phenomenon updating rule is proposed by considering an employment of a proposed utility function as a global criterion. The paper \cite{wang2014novel} introduces nonfunctional attributes of web services to include trust degree according to the execution log. Also, a novel adaptive ant colony optimisation algorithm is proposed to overcome the slow convergence presented from the traditional ant colony optimisation algorithm. In particular, the pheromone strength coefficient is adjusted dynamically to control both the updating and evaporation of pheromone. The experiment results are analysed in an alternative way. That is, the total Pareto solutions are combined from different compared ACO algorithms, then the accurate rate of each algorithm is calculated based on the compared Pareto solutions identified in the total Pareto solutions. The results also show more Pareto solutions found compared to the traditional ACO methods. However, the experiment is only conducted for the evaluation of a small case study, where only a simple abstract workflow is studied.

\subsubsection{Many-objective approaches}\label{ManyObjective}

Herein, more than three objectives in Multi-objective problems (MOPs) are often considered as many-objective problems. Ishibuchi et al. \cite{ishibuchi2008evolutionary} present an analysis of the multi-objective algorithm for handling optimisation problems with more than 3 objectives. However, they address that the searching ability is deteriorating while the number of objectives is increasing, since the non-dominated solution is very large, which make it harder to move solutions towards the Pareto Front.

The work \cite{de2010many} employs NSGA-II to deal with five different quality criteria (i.e., runtime, price, reputation, availability and reliability) for semi-automated web service composition problem.  To examine the techniques to decrease the deterioration, two preference relations proposed by \cite{bentley1997finding} are applied to NSGA-II: Maximum Ranking (MR) and Average Ranking methods (AR). In particular, MR is the best of all the ranking scores from all the objectives, and AR is a sum of all the ranking scores from all the objectives. Therefore, three algorithms (NSGA-II, NSGA-II with MR and NSGA-II with AR) are evaluated for studying the five different performance metrics ( i.e., hypervolume \cite{zitzler1999evolutionary}, Generational Distance \cite{van2000measuring}, Spread and Coverage \cite{zitzler2000comparison}, and pseudo Pareto front (i.e, a combination of all non-dominated solutions)),  where An empirical evaluation is performed on. The experiment shows NSGA-II with AR outperforms others in both GD and Spread (i.e., more balanced solutions). However,  a certain region of  Pareto Front is generated by NSGA-II,  rather than a wider distribution for the solutions. NSGA-II with MR performs intermediately compared to the other two algorithms. On the whole,  this work, for the first time, takes two preference relations into account for solving many-objective service composition problem, and contribute to finding better solutions with many performance metrics.


\subsection{Dynamic Web Service Composition Approaches}\label{dynamicserivce}
All the previously discussed approaches can be classified into one group that assumes the composition environment is static. The rest approaches can be classified into another group that does not make that closed world assumption. Instead, a real world scenario is taken into account. For example, nonfunctional properties of web services may fluctuate over time or services are failed/ newly registered. A few researcher works on the second group to cope with the dynamic composition environment. To address this problem \cite{nasridinov2012qos},  a suitable mechanism for effectively and efficiently handle this problem raise a significant challenge for the practical value.

\subsubsection{Dynamic Web Service Composition Approaches For Changes in QoS}\label{dynamicQoS}

Service selection is one of the crucial steps when we compose services. In the context of static web service composition, we always selection services based on the QoS advertised by service providers. However, QoS is fluctuating over the time. Herein the execution instances of a service for the run time indicate the dynamic and real QoS of a service. This dynamic QoS is formally modelled as uncertain QoS model.  Based on this model, some studied \cite{wen2014probabilistic} has been addressed recently. In \cite{wen2014probabilistic}, the QoS model describes the probabilities of different dominating relationships in the instances of all services within the same class. To efficient provide a relatively small set of services for selection that is based on the uncertain QoS model, the data structure of R-tree is introduced for spatial query on multidimensional data (i.e., many dimension of QoS attributes) since it can significantly reduce the searching space. This space index technique is one of the key contributions in this paper for efficiently store and retrieve services for service selection. 

Reinforcement learning (RL) is one technique of machine learning for solving sequential decision-making problems to maximise some long-term rewards.  RL is utilised to deal with how actions are taken in an uncertain environment. In our context, this uncertain environment is related to QoS. In \cite{mostafa2015multi}, two approaches are proposed based on multi-objective service composition in uncertain and dynamic environment (MORL) combining the advantages in multi-objective optimisation and reinforcement techniques. In particular, we service composition is modelled based on Partially Observable Markov Decision Process (POMDP), and the solutions to services composition are considered to be a set of decision policies, each of which is considered as a procedure of service selection (i.e., a single workflow). Two approaches are introduced in this paper to learn the optimal selection policy. 

\subsubsection{Dynamic Web Service Composition Approaches For Service Failure}\label{dynamicService}
Traditionally, re-selection of failed service is one of widely used approach to re-ensure the desired execution web service composition. The idea of this traditional approach restores many alternative service candidates for each component service involved in the composition solution. If component service confronts failure, the alternatives are used for the replacement.  A framework, WS-Replication is introduced in \cite{salas2006ws}, which mainly address service failure using the idea of the traditional method. \cite{yin2010qos} further introduce a more complicated model that address attributes of not only QoS but also transactionality for more reliable replacement. However, those approaches \cite{salas2006ws,yin2010qos} have a huge cost in computation due to the re-selection mechanisms \cite{nasridinov2012qos}. To overcome the disadvantage of traditional resection approach, \cite{nasridinov2012qos} proposes a QoS-aware performance prediction for self-healing web service composition system. This system consists of three main phases: monitoring, diagnosis and repair. The monitoring phase is to detect degradation, diagnosis is to identify the source of degradation, and repair is to reselect desired services. To minimising the number of re-selection in the phase of repair, decision tree learning is used to the prediction of the performance based on the QoS. This technique outperforms other classification techniques (i.e., back propagation neural network, support vector machine probabilistic neural network, and group method of data handling and regression tree) by accuracy in \cite{mohanty2010web}.  Those works mainly focus on services failure, and they do not recognise the importance of  QoS changes of service in the repository. \cite{wagner2016robust} proposes a method to cluster services that are back up for service failure. This method determines a set of backup services based on their functional properties for service repairing. By employing functional properties, services and their combinable services are identified. The fragmented clusters are formed based on the subsume relationship among their associated combinable services. Also, to merge these fragmented clusters, unified clusters are created by adding virtual services that subsume the fragmented clusters. Therefore, a set of backup services is provided in both the clusters that the failure services involved in and their sub-clusters, from which we could select suitable service.


\subsection{Semantic Web Service Composition Approaches}\label{Semantic}
Sophisticated real-world applications demand complex requirements in developing service composition. Apart from consideration of both the functional and nonfunctional properties of web services, more complex chaining strategies of semantic services are necessarily arranged to satisfy dependency constraints in the real world. To cope with these complicated requirements, preconditions and postconditions are semantically described using logic formula to enable the complex service composition. any standards have been established for supporting the chaining ability of semantic web services, such as OWL-S. OWL-S accommodates varieties of structural constructs for service composition, namely: Sequence, Unordered, Choice, If-Then-Else, Iterate, Repeat-While, Repeat-Until, Split and ``Split + Join'' \cite{wang2014automated}. For example, ``Split+Join'' calls for the process consisting of the parallel execution of a collection of process components with barrier synchronisation. That is, ``Split+Join'' doesn't completes until all of its components processes have been completed. Additionally, ``Split'' and ``Split+Join'' can be applied for partial synchronisation \cite{wang2014automated}.

Most of the existing service composition approaches do not fully consider services registered in the repository are associated with preconditions and postconditions.  Therefore, it is hard to generate service composition solutions with branch structure in a fully automated way. On the one hand,  a large number of approaches \cite{} only consider inputs and outputs while composing services, the importance of preconditions and postconditions are ignored since it specifies preconditions that need to be satisfied and postconditions that result from service execution,  which causes the effect on the next services. On the other hand, some approaches consider preconditions and postconditions using classic AI planning methods. The limitation of these approaches lies in the linear sequences of actions.


A generalised semantic web service composition is introduced in \cite{bansal2016generalized}, preconditions and postconditions are effectively presented in a conditional directed acyclic graph where conditional nodes created with two outgoing edges representing the satisfied and the unsatisfied case at the run time. They filter the solutions based on the trust rate using Centrality Measure of Social Networks to find web service trusted by the customers. Therefore, a semantic web service composition engine is implemented for automated generated conditional composition and OWL-S description used for execution phrase, which minimises the query execution time by pre-processing and incremental updates to new service added or modified. The main limitation of this work is that the loops structure are not considered in their web service representation, and they leave optimisation problems aside. 

\subsection{Summary and Limitations}\label{summary}


%\section{Summary and Limitations}\label{summary}
%This chapter presented an overview of the recent research conducted on different aspects of the automated Web service composition problem. The first area explored was that of \textbf{single-objective composition}, which aims to create composite solutions with the best possible overall Quality of Service (QoS) by conducting optimisations according to an objective function. Different approaches have been attempted for this, including both traditional approaches such as Integer Linear Programming (ILP) and Evolutionary Computation (EC) approaches such as Genetic Programming (GP), though in certain cases traditional approaches may not scale well as the composition problems grow in complexity. One key limitation of single-objective composition approaches is that they neglect the issue of branching, that is, they do not allow the creation of solutions with multiple alternative outputs that may be produced depending on a condition. To overcome this problem, a new candidate representation that also encodes this form of conditional branching needs to be proposed.

%Another area discussed is that of \textbf{multi-objective, top-K, and many-objective composition}, where each QoS attribute is optimised separately, and a set of solutions presenting trade-offs between these different quality attributes is generated. Multi-objective approaches work best for optimisation problems that involve two or three independent dimensions, while many-objective approaches are capable of handling more than three of them; top-K aims to produce a predetermined number of solution options based on a ranking strategy. The optimisation of multiple independent objectives is complex and provides many possibilities for further exploration. One interesting problem is that of many-objective optimisation for SLA-aware Web service composition, which refers to the constrained optimisation of several independent objectives to reflect the minimum quality requirements of the composition requestor. Accomplishing this is particularly challenging when creating solutions with multiple output possibilities.

%A number of \textbf{AI planning-based composition} approaches are also presented in this chapter. Their fundamental idea is to build solutions step by step, adding one atomic service to the composition at a time and subsequently checking whether the overall desired output has now been produced. These approaches are conducive to ensuring that the generated solutions are feasible and also included conditional branches into the solution's workflow whenever necessary. Despite these advantages, it is difficult to globally optimise the quality of a solution produced through planning, since its components are not modified or improved once they have been selected. A promising strategy to overcome this problem would be to combine planning algorithms with a population-based approach for optimisation, though this avenue has not yet been significantly explored by researchers.

%The \textbf{semantic Web service selection} approaches examined in this chapter propose a more realistic way of selecting the atomic components of a composition. Instead of expecting the return values and parameter types of service operations to match perfectly, the closest possible output-input matches are calculated using semantic distance measurements. The limitation of most works in this area is that they restrict their selection technique to semi-automated composition, where it is assumed that a framework with abstract service slots that are to be filled by concrete atomic services has already been provided. In the case of composition approaches based on AI planning techniques, where the workflow is built as services are connected to the solution, more flexible semantic selection methods are necessary. This motivates further research in this area.

%Finally, this chapter focuses on the issue of \textbf{dynamic Web service composition}. In this type of composition a closed world is abandoned, meaning that the state of services is expected to change throughout time. This brings two main problems: firstly, when the quality of the services in a repository changes, a solution that was once optimised according to the previous quality measures may suddenly transform into a low-standard alternative; secondly, certain services may become unavailable, meaning that solutions which incorporate them are henceforth unusable. The dynamic composition approaches discussed in this chapter use a variety of self-healing and recovery techniques to adjust solutions according to these changes, however they do not rely on EC techniques for doing so. These techniques can quickly re-optimise the QoS of existing solutions and provide composition backups (i.e. other candidates in the population) in case of failure, thus making EC an interesting dynamic alternative.

\chapter{Preliminary Work}\label{C:preliminary}

This chapter exhibits the two initial works for comprehensive quality-aware semantic web service composition approach, which combines a hybridization of various techniques for optimising the quality of semantic matchmaking and quality of service. In particular,  One direct representation and another indirect representation are proposed in two premilitary works using a PSO-based approach and a GP-based approach respectively. Apart from that, composition solutions are represented in a DAG or a tree-like representation,  respectively, in the two approaches. These two works are discussed in the following sections.

\section{Problem Formalisation}\label{problemDes}

We consider a \emph{semantic web service} (\emph{service}, for short) as a tuple $S =(I_{S}, O_{S}, $ $QoS_S)$ where $I_{S}$ is a set of service inputs that are consumed by $S$, $O_{S}$ is a set of service outputs that are produced by $S$, and $QoS_{S}=\{t_S, c_S, r_S, a_S\}$ is a set of non-functional attributes of $S$. The inputs in $I_{S}$ and outputs in $O_{S}$ are parameters modelled through concepts in a domain-specific ontology $\mathcal{O}$. The attributes $t_S, c_S, r_S, a_S$ refer to the response time, cost, reliability, and availability of service $S$, respectively. These four QoS attributes are most commonly used \cite{zeng2003quality}.

A \emph{service repository} $\mathcal{SR}$ is a finite collection of services supported by a common ontology $\mathcal{O}$. A \emph{service request} (also called \emph{composition task}) over $\mathcal{SR}$ is a tuple $T=(I_{T}, O_{T})$ where $I_{T}$ is a set of task inputs, and $O_{T}$ is a set of task outputs. The inputs in $I_{T}$ and outputs in $O_{T}$ are parameters described by concepts in the ontology $\mathcal{O}$.

%A service composition is commonly represented as a \emph{directed acyclic graph} (DAG). Its nodes correspond to the services in the composition. 
\emph{Matchmaking types} are often used to describe the level of a match between outputs and inputs \cite{paolucci2002semantic}: For concepts $a, b$ in $\mathcal{O}$ the \emph{matchmaking} returns $exact$ if $a$ and $b$ are equivalent ($a \equiv b$), $plugin$ if $a$ is a sub-concept of $b$ ($a \sqsubseteq b$), $subsume$ if $a$ is a super-concept of $b$ ($a \sqsupseteq b$), and $fail$ if none of previous matchmaking types is returned. In this paper we are only interested in robust compositions where only $exact$ and $plugin$ matches are considered, see \cite{lecue2009optimizing}. As argued in \cite{lecue2009optimizing} $plugin$ matches are less preferable than $exact$ matches due to the overheads associated with data processing. We suggest to consider the semantic similarity of concepts when comparing different $plugin$ matches.

\emph{Robust causal link} \cite{lecue2008optimizing} is a link between two matched services $S$ and $S'$, noted as $S \rightarrow S'$, if an output $a$ ($a \in {O_S}$) of $S$ serves as the input $b$ ($b \in {O_{S'}}$) of $S'$ satisfying either $a \equiv b$ or $a \sqsubseteq b$.  For concepts $a, b$ in $\mathcal{O}$ the \emph{semantic similarity} $sim(a, b)$ is calculated based on the edge counting method in a taxonomy like WorldNet or Ontology \cite{shet2012new}. This method has the advantages of simple calculation and good performance \cite{shet2012new}. Therefore, the \emph{matchmaking type} and \emph{semantic similarity} of a robust causal link can be defined as follow:
\begin{align}
\label{eq_link}
type_{link} = 
\begin{cases}
	1 & \text{ if $a\equiv b$ ($exact$ match)}\\
	p & \text{ if $a \sqsubseteq b$ ($plugin$ match)}
\end{cases}
,&&
sim_{link} = sim(a,b) = \frac{2N_c}{N_{a}+N_{b}}
\end{align}

\noindent with a suitable parameter $p, 0<p< 1$, and with $N_a$, $N_b$ and $N_c$, which measure the distances from concept $a$, concept $b$, and the closest common ancestor $c$ of $a$ and $b$ to the top concept of the ontology $\mathcal{O}$, respectively. However, if more than one pair of matched output and input exist from service $S$ to service $S'$, $type_e$ and $sim_e$ will take on their average values.

The \emph{semantic matchmaking quality} of the service composition can be obtained by aggregating over all robust causal links as follow:
\begin{align}
MT {=} \prod_{j=1}^{m} type_ {link_{j}}
,&&
SIM {=} \frac{1}{m}\sum_{j=1}^m sim_ {link_{j}}  
\end{align}

We consider two special atomic services $Start = (\emptyset, I_T, \emptyset )$ and $End  = (O_T, \emptyset, \emptyset)$ to account for the input and output requirements given by the composition task $T$, and add them to $\mathcal{SR}$. 

We use formal expressions as in \cite{ma2012formal} to represent service compositions. We use the constructors $\bullet$, $\parallel$, $+$ and $\ast$ to denote sequential composition, parallel composition, choice, and iteration, respectively. The set of \emph{composite service expressions} is the smallest collection $\mathcal{SC}$ that contains all atomic services and that is closed under sequential composition, parallel composition, choice, and iteration. That is, whenever $\cse_0,\cse_1,\ldots,\cse_d$ are in $\mathcal{SC}$ then $\bullet(\cse_1,\ldots,\cse_d)$, $\parallel(\cse_1,\ldots,\cse_d)$, $+(\cse_1,\ldots,\cse_d)$, and $\ast \cse_0$ are in $\mathcal{SC}$, too. Let $\cse$ be a composite service expression. If $\cse$ denotes an atomic service $S$ then its QoS is given by $QoS_S$.  Otherwise the QoS for $\cse$ can be obtained inductively as summarized in Table~\ref{tbl:QoS_Aggre}. Herein, $p_1,\ldots,p_d$ with $\sum\limits^d_{k=1}p_k=1$ denote the probabilities of the different options of the choice $+$, while $\ell$ denotes the average number of iterations.

\begin{table}[htb]
\centering
\caption{QoS calculation for a composite service expression $\cse$}
\begin{tabular}{l|l|l|l|l}
\hline
 $\cse=$       &$r_\cse=$                              &$a_\cse=$                              &$c_\cse=$                            &$t_\cse=$ \\ \hline
 $\bullet(\cse_1,\ldots,\cse_d)$      &$\prod\limits^d_{k=1}r_{\cse_k}$    &$\prod\limits^d_{k=1}a_{\cse_k}$    &$\sum\limits^d_{k=1}c_{\cse_k}$   &$\sum\limits^d_{k=1}t_{\cse_k}$  \\ \hline
 $\parallel(\cse_1,\ldots,\cse_d)$  &$\prod\limits^d_{k=1}r_{\cse_k}$    &$\prod\limits^d_{k=1}a_{\cse_k}$    &$\sum\limits^d_{k=1}c_{\cse_k}$   &$MAX \{ t_{\cse_k} | k \in \{ 1,...,d \} \}$\\ \hline
 $+(\cse_1,\ldots,\cse_d)$     &$\prod\limits^d_{k=1}p_k\cdot r_{\cse_k}$    &$\prod\limits^d_{k=1}p_k\cdot a_{\cse_k}$    &$\sum\limits^d_{k=1}p_k\cdot c_{\cse_k}$   &$\sum\limits^d_{k=1}p_k\cdot t_{\cse_k}$  \\ \hline
 $\ast \cse_0$         &${r_{\cse_0}}^\ell$  &${a_{\cse_0}}^\ell$  &$\ell\cdot c_{\cse_0}$ &$\ell\cdot t_{\cse_0}$ \\ \hline
\end{tabular}
\label{tbl:QoS_Aggre}
\end{table}

When multiple quality criteria are involved in decision making, the fitness of a solution can be defined as a weighted sum of all individual criteria using Eq. (\ref{eq_fitness}), assuming the preference of each quality criterion is provided by users.
\begin{equation}
\label{eq_fitness}
Fitness = w_1 \hat{MT} + w_2 \hat{SIM} + w_3 \hat{A} + w_4 \hat{R} + w_5(1 - \hat{T}) + w_6(1 - \hat{C})
\end{equation}
\noindent with $\sum_{k=1}^{6} w_k= 1$. We call this objective function the \emph{comprehensive quality model} for service composition.
The weights can be adjusted according to users' preferences. $\hat{MT}$, $\hat{SIM}$, $\hat{A}$, $\hat{R}$, $\hat{T}$, and $\hat{C}$ are normalised values calculated within the range from 0 to 1 using Eq. (\ref{eq_normal}). To simplify the presentation we also use the notation $(Q_1,Q_2,Q_3,Q_4,Q_5,Q_6) $ $= (MT,SIM,A,R,T,C)$. $Q_1$ and $Q_2$ have minimum value 0 and maximum value 1. The minimum and maximum value of $Q_3$, $Q_4$, $Q_5$, and $Q_6$ are calculated across all task-related candidates in the service repository $\mathcal{SR}$ using the greedy search in \cite{ma2015hybrid,da2016genetic}.

\begin{equation}
\label{eq_normal}
\hat{Q_k} = 
\begin{cases}
	\frac{Q_k - Q_{k, min}}{Q_{k, max} - Q_{k, min}} & \text{ if $k=1,\ldots,4$ and }Q_{k, max} - Q_{k, min} \neq 0,\\
	\frac{Q_{k,max} - Q_k}{Q_{k, max} - Q_{k, min}} & \text{ if $k=5,6$ and }Q_{k, max} - Q_{k, min} \neq 0,\\
	1 & \text{ otherwise}.
\end{cases}
\end{equation}

\noindent To find best possible solution for a given composition task $T$, our goal is to maximise the objective function in Eq. (\ref{eq_fitness}).


\section{PSO-based Approach to Comprehensive Quality-Aware Automated Semantic Web Service Composition}\label{qswsc_approach}
\subsection{An Overview of our PSO-based Approach}\label{PSO_based_approach}
As PSO has shown promise in solving combinatorial optimisation problems, we propose a PSO-based approach to comprehensive quality-aware automated semantic web service composition. Fig. \ref{overview} shows an overview of our approach consisting of four steps: 
\begin{figure}[h]
\centering
\fbox{\includegraphics[scale=.5]{overview.pdf}}
 \caption{An overview of our PSO-based approach to comprehensive quality-aware automated semantic web service composition.}
 \label{overview}
\end{figure}

Step 1: The composition process is triggered by a composition task, which is clearly defined in Section \ref{problemDes}. 

Step 2: The composition task is used to discover all task-related service candidates using a greedy search algorithm adopted from \cite{ma2015hybrid}, which contributes to a shrunken service repository. This greedy search algorithm keeps adding outputs of the invoked services as available outputs (initialised with $I_{T}$) , and these available outputs are used to discover task-related services from a service repository and updated with the outputs of these discovered services. This operation is repeated until no service is satisfied by the available outputs. During the greedy search, an ontology-based cache ($cache$) is initialised, which stores the concept similarities of matched inputs and outputs of task-related candidates. This $cache$ is also used to discover services by checking whether $null$ is returned by given two output-related and input-related concepts.

Step 3 and Step 4: These two steps follow the standard PSO steps \cite{shi2001particle} except for some differences in particles mapping and decoding processes. In particular, these two differences are related to sorting a created service queue using service-to-index mapping for a particle' position vectors and evaluating the fitness of a particle after decoding this service queue into a $WG$ respectively. Those differences are further addressed in Algorithms \ref{novelSteps} and \ref{graph_building} in Section \ref{PSO-based_algomargin}.
\subsection{The Algorithms for our PSO-based Approach}\label{PSO-based_algomargin}
The overall algorithm investigated here is made up of a PSO-based web service composition technique (Algorithm \ref{novelSteps}) and a $WG$ creating technique from a service queue (Algorithm \ref{graph_building}). In Algorithm \ref{novelSteps}, the  steps $4$, $5$, $6$ and $7$ are different from those of standard PSO: In step 4, the size of task-related service candidates generated by a greedy search determines the size of each particle's position. Each service candidate in a created service candidates queue is mapped to an index of a particle’s position vectors, where each vector has a weight value between 0.0 and 1.0. In step 5, service candidates in the queue are sorted according to their corresponding weight values in descending order. In step 6, this sorted queue is used as one of the inputs of the forward decoding Algorithm \ref{graph_building} to create a $WG$. In step 7, the fitness value of the created $WG$ is the fitness value of the particle calculated by the comprehensive model discussed in Section \ref{problemDes}.
\begin{algorithm}
 %\LinesNumbered
 \SetKwInOut{Input}{Input}\SetKwInOut{Output}{Output}
 \SetKwFunction{generateWeightedGraph}{generateWeightedGraph}
 \SetKwProg{Procedure}{Procedure}{}{}
 \SetNlSty{}{}{:}
 Randomly initialise each particle in the swarm\;
  \While {max. iterations not met}{
     \ForEach{particle in the swarm}{
     Create a service candidates queue and map service candidates to a particle's position vectors\;
     Sort the service queue by position vectors' weights\;
     Use Algorithm \ref{graph_building} to create a $WG$ from the service queue\;
     Calculate the $WG$ fitness value\;
     
      \eIf{fitness value better than pBest}{    
        Assign current fitness as new \emph{pBest}\;
       }{
        Keep previous \emph{pBest}\;
       }	
     }
    Assign best particle's \emph{pBest} value to \emph{gBest}, if better than \emph{gBest}\;
 	Calculate the velocity of each particle\;
  	Update the position of each particle\;
  }
\caption{Steps of PSO-based service composition technique \cite{da2016particle}.}
\label{novelSteps}
\end{algorithm} 

Algorithm  \ref{graph_building} is a forward graph building algorithm extended from \cite{blum1997fast}. This algorithm takes one input, a sorted service queue from step 5 of Algorithm \ref{novelSteps}. Note that different service queues may lead to different $WGs$. In addition. $I_{T}$, $O_{T}$ and $cache$ are also taken as the inputs. Firstly, $Start$ and $End$ are added to $V$ of $WG$ as an initialisation, and $OutputSet$ is also created with $I_{T}$. The following steps are repeated until $O_{T}$ can be satisfied by $Outputset$ or the service queue is $null$. If all the inputs $I_{S}$ of the first popped  $S$ from $queue$ can be satisfied by provided outputs from $OutputSet$, this $S$ is added to $V$ and its outputs are added to $OutputSet$, and $S$ is removed from $queue$. Otherwise, the second popped  $S$ from $queue$ is considered for these operations. Meanwhile, $e$ is created with $type_e$ and $sim_e$ if $S$ is added, and calculated using information provided from $cache$. This forward graph building technique could lead to more services and edges connected to the $WG$, these redundancies should be removed before $WG$ is returned.

\begin{algorithm}
 \SetKwInOut{Input}{Input}\SetKwInOut{Output}{Output}
 \SetKwFunction{createWeightedDAG}{createWeightedDAG}
 \SetKwProg{Procedure}{Procedure}{}{}
 %\LinesNumbered
 \SetNlSty{}{}{:}
 % \Procedure{}{
 \Input{ $I_T$, $O_T$, $queue$, $cache$}
 \Output{WG}
 $WG = (V, E)$\;
 $V \leftarrow$ \{$Start$, $End$ \}\;
 $OutputSet \leftarrow$ \{$I_{T}$\}\;
  \While { $O_{T}$ not satisfied by $OutputSet$}{
     \ForEach{$S$ in $queue$}{
      \uIf{$I_{S}$ satisfied by $OutputSet$}{  
        insert $S$ into $V$\;  
        adjoin $O_{S}$ to $OutputSet$\;
        $queue$.remove $S$\;   
        $e \leftarrow$ calculate $type_e$, $sim_e$ using $cache$\;
        insert $e$ into $E$\;
       }	
     }
  }
 remove $dangling$ $nodes$ and $edges$ from $WG$\; 
 \KwRet $WG$\;
 %}
 \caption{Create a $WG$ from a sorted service queue.}
\label{graph_building}
\end{algorithm} 

\section{Experiment Study for PSO-based Approach}\label{experiment_design}
In this section, we employ a quantitative evaluation approach with a benchmark dataset used in \cite{ma2015hybrid,da2016genetic}, which is an augmented version of Web Service Challenge 2009 (WSC09) including QoS attributes. Two objectives of this evaluation are to: $(1)$ evaluate the effectiveness of our PSO-based approach, see comparison test in Section \ref{comparisonTestWithGP}. $(2)$ evaluate the effectiveness of our proposed comprehensive quality model to achieve a desirable balance on semantic matchmaking quality and QoS, see comparison test in Section \ref{comparisonTest}.

The parameters for the PSO are chosen from the settings from \cite{shi2001particle}, In particular, PSO population size is 30 with 100 generations. We run 30 times independently for each dataset. We configure the weights of fitness function to properly balance semantic matchmaking quality and QoS. Therefore, $w_{1}$ and $w_{2}$ are set equally to 0.25, and $w_{3}$, $w_{4}$, $w_{5}$, $w_{6}$ are all set to 0.125. The $p$ of $type_e$ is set to 0.75 ($plugin$ match) according to \cite{lecue2009optimizing}. In general, weight settings and parameter $p$ are decided according to users' preferences.

\subsection{GP-based vs. PSO-based approach}\label{comparisonTestWithGP}
To evaluate the effectiveness of our proposed PSO-based approach, we compare our PSO-based method with one recent GP-based approach \cite{ma2015hybrid} using our proposed comprehensive quality model. We extend this GP-based approach by measuring the semantic matchmaking quality between parent nodes and children nodes. To make a fair comparison, we use the same number of evaluations (3000 times) for these two approach. We set the parameters of that GP-based approach as 30 individuals and 100 generations, which is considered to be proper settings referring to \cite{da2015gp}.

The first column of Table \ref{meanFitness} shows five tasks from WSC09. The second and third column of Table \ref{meanFitness} show the original service repository size and the shrunk service repository size after the greedy search respectively regarding the five tasks. This greedy search helps reducing the original repository size significantly, which contributes to a reduced searching space. The fourth and fifth column of Table \ref{meanFitness} show the mean fitness values of 30 independent runs accomplished by two methods. We employ independent-samples T tests to test the significant differences in mean fitness value. The results show that the PSO-based approach outperforms the existing GP-based approach in most cases except Task 3. Note that all $p$-values are consistently smaller than 0.01. Using our PSO-based approach, small changes to sorted queues (particles in PSO) could lead to big changes to the composition solutions. This enables the PSO-based approach to escape from local optima more easily than the GP-based approach. 
%the PSO-based approach performs significantly better than the GP-based approach in finding optimal solutions. It may be that the GP-based approach is stuck in local optima in a very large search space due to its evolutionary operators. On the other hand, the decoding process used by the PSO-based approach allows for small changes that more effectively prevent this from happening.
\begin{table}[]
\centering
\caption{Mean fitness values for comparing GP-based approach}
\label{meanFitness}
\begin{tabular}{c|c|c|l|l}
\hline
\multicolumn{1}{c|}{WSC09} &Original $\mathcal{SR}$  &Shrunken $\mathcal{SR}$   &PSO-based approach & GP-based approach  \\ \hline
Task 1                     &572            &80    &0.5592 $\pm$ 0.0128  $\uparrow$  &0.5207 $\pm$ 0.0208           \\ \hline
Task 2                     &4129           &140   &0.4701 $\pm$ 0.0011  $\uparrow$  &0.4597 $\pm$ 0.0029          \\ \hline
Task 3                     &8138           &153   &0.5504 $\pm$ 0.0128              &0.5679 $\pm$ 0.0234 $\uparrow$   \\ \hline
Task 4                     &8301           &330   &0.4690 $\pm$ 0.0017  $\uparrow$  &0.4317 $\pm$ 0.0097            \\ \hline
Task 5                     &15211          &237   &0.4694 $\pm$ 0.0008  $\uparrow$  &0.2452 $\pm$ 0.0369            \\ \hline
\end{tabular}
\end{table}

\subsection{Comprehensive Quality Model vs. QoS Model}\label{comparisonTest}

Recently, a QoS Model, $Fitness = w_1 \hat{A} + w_2 \hat{R} + w_3(1 - \hat{T}) + w_4(1 - \hat{C})$, where $\sum_{i=1}^{4} w_i = 1$, is widely used for QoS-aware web service composition \cite{ma2015hybrid,da2016particle,da2015graphevol}. To show the effectiveness of our proposed comprehensive quality model, we compare the best solutions found by this QoS model and our comprehensive model using our PSO-based approach. We record and compare the mean values of both $SM$ ($SM = 0.5 \hat{MT} + 0.5 \hat{SIM}$) and $QoS$($QoS = 0.25 \hat{A} + 0.25 \hat{R} + 0.25(1 - \hat{T}) + 0.25(1 - \hat{C})$) of best solutions over 30 independent runs. To make the comparison informative, all these recorded values have been normalised from 0 to 1, and compared using independent-samples T tests, see Table \ref{decisionTable}. Note that p-values are consistently smaller than 0.001 in the results indicating significant differences in performance. 

In Table \ref{decisionTable}, the mean values of $QoS$ using QoS model are significantly higher than those using comprehensive quality model for Tasks 2, 3, 4 and 5. However, the mean value of $SM$ using the comprehensive quality model are significantly higher than those using the QoS model, while a slight trade-off in $QoS$ are observed in all tasks. In addition, our comprehensive model achieves a consistently higher comprehensive quality in terms of a combination of $SM$ and $QoS$, which is significantly better in Tasks 1, 2, 3 and 4. 
\begin{table}[]
\footnotesize
\centering
\caption{Mean values of $SM$, $QoS$ and sum of $SM$ and $QoS$ for QoS model and comprehensive quality model using PSO-based approach}
\label{decisionTable}
\begin{tabular}{c|c|l|l}
\hline
\multicolumn{2}{c|}{WSC09}              & \shortstack{QoS \\ Model}         &\shortstack{Comprehensive Quality \\ Model} \\ \hline
\multirow{3}{*}{Task1}  &$SM$      &0.5373 $\pm$ 0.0267               &0.5580 $\pm$ 0.0094 $\uparrow$ \\ \cline{2-4}
                        &$QoS$     &0.5574 $\pm$ 0.0156               &0.5604 $\pm$ 0.0164            \\ \cline{2-4}
                        &$SM+QoS$  &1.0947 $\pm$ 0.0423               &1.1184 $\pm$ 0.0258 $\uparrow$ \\ \hline
\multirow{3}{*}{Task2}  &$SM$      &0.4549 $\pm$ 0.0033               &0.4630 $\pm$ 0.0042 $\uparrow$ \\ \cline{2-4} 
                        &$QoS$     &0.4800 $\pm$ 0.0012 $\uparrow$    &0.4772 $\pm$ 0.0025            \\ \cline{2-4}
                        &$SM+QoS$  &0.9349 $\pm$ 0.0045               &0.9402 $\pm$ 0.0067 $\uparrow$           \\ \hline
\multirow{3}{*}{Task3}  &$SM$      &0.5538 $\pm$ 0.0082               &0.6093 $\pm$ 0.0054 $\uparrow$ \\ \cline{2-4} 
                        &$QoS$     &0.4940 $\pm$ 0.0013 $\uparrow$    &0.4913 $\pm$ 0.0009            \\ \cline{2-4}
                        &$SM+QoS$  &1.0478 $\pm$ 0.0095               &1.1006 $\pm$ 0.0063 $\uparrow$           \\ \hline
\multirow{3}{*}{Task4}  &$SM$      &0.4398 $\pm$ 0.0037               &0.4604 $\pm$ 0.0000 $\uparrow$ \\ \cline{2-4} 
                        &$QoS$     &0.4845 $\pm$ 0.0010 $\uparrow$    &0.4734 $\pm$ 0.0044            \\ \cline{2-4}
                        &$SM+QoS$  &0.9243 $\pm$ 0.0047               &0.9338 $\pm$ 0.0044 $\uparrow$           \\ \hline
\multirow{3}{*}{Task5}  &$SM$      &0.4580 $\pm$ 0.0065               &0.4639 $\pm$ 0.0013 $\uparrow$ \\ \cline{2-4} 
                        &$QoS$     &0.4764 $\pm$ 0.0005 $\uparrow$    &0.4750 $\pm$ 0.0007            \\ \cline{2-4}
                        &$SM+QoS$  &0.9344 $\pm$ 0.0070               &0.9389 $\pm$ 0.0020           \\ \hline
\end{tabular}
\end{table}
\subsection{Further Discussion}\label{discuss1}
To analyse the effectiveness of achieving a good comprehensive quality at the expense of slightly reduced QoS, we demonstrate two best solutions produced using Task 3 as an example. Fig. \ref{comparisontest} $(1)$ and $(2)$ show two weighted DAGs, $WG_1$ and $WG_2$, which have been obtained as the best service compositions solutions based on the QoS model and on the comprehensive quality model, respectively. Both $WGs$ have exactly the same service workflow structure, but some service vertices and edges denoted in red are different. To better understand these differences, we list the overall semantic matchmaking quality $SM$,  overall $QoS$ and semantic matchmaking quality $sm_{e_n}$ associated to these different edges in $WG_1$ and $WG_2$. (Note: $sm_{e_n} = 0.5type_{e_n} + 0.5 sim_{e_n}$), where $\Delta Q$ reveals the gain (positive $\Delta Q$) or a loss (negative $\Delta Q$) of the listed qualities for our comprehensive quality model. Therefore, we achieve a comprehensive quality gain (+0.1433), a result of a gain in semantic matchmaking quality (+0.1467) and a loss in $QoS$ (-0.0034). To understand the improvement of semantic matchmaking quality from these numbers, we pick up $e_4$ that is associated with the smallest $\Delta Q$. The $e_4$ of $WG_1$ and $WG_2$ has two different source nodes, $Ser1640238160$ and $Ser947554374$, and two the same $End$ nodes. $Ser1640238160$ and $Ser947554374$ are services with output parameters $Inst582785907$ and  $Inst795998200$ corresponds to two concepts $Con2037585750$ and $Con103314376$ respectively in the given ontology shown in Fig. \ref{comparisontest} $(4)$. As $Inst658772240$ is a required parameter of $End$, and related to concept $Con2113572083$, $Inst795998200$ is closer to the required output $Inst658772240$ than $Inst582785907$. Therefore,  $Ser947554374$ is selected with a better semantic matchmaking quality compared to $Ser1640238160$.
\begin{figure}[h]
\centering{
\fbox{
\includegraphics[scale=.29]{comparisontest.pdf}}}
 \caption{An example for the comparison of the best solutions obtained based on the QoS model and on the comprehensive quality model for Task 3.}
 \label{comparisontest}
\end{figure}

\section{Summary for PSO-based Approach}\label{summary1}

In PSO-based approach, we propose an effective PSO-based approach to comprehensive quality-aware semantic web service composition, which also has shown promise in achieving a better comprehensive quality in terms of a combination of semantic matchmaking quality and QoS compared to existing works.
%=================================================================================================== GP Approach
\section{GP-based Approach to Comprehensive Quality-Aware Automated Semantic Web Service Composition}\label{GPApproach}

In this section, we first introduce the tree-like representation that will be used in our approach, and then discuss the differences to the most widely used tree-based representations for GP-based service composition in the literature \cite{gupta2015optimization,da2016genetic,yu2013adaptive}. Finally, we present our GP-based approach with newly designed genetic operation methods.
%========================================================================================= Representation
\subsection{A New Tree-like Representation for Web Service Composition}\label{representation} 

Let $\gra=(V,E)$ be a DAG representation of a service composition. Let $S$ be a service in $\gra$, and let $S_1,\ldots,S_d$ be its successors in $\gra$. We define the composite service expression relative to $S$ as follows: 
\begin{equation}
\label{eq_s_expression}
    \cse_S=
    \begin{cases}
      \bullet(S,\parallel(\cse_{S_1},\ldots,\cse_{S_d})), & \text{if $d\ge 2$}, \\
      \bullet(S,\cse_{S_1}), & \text{if $d=1$}, \\
      S, & \text{if $d=0$},
    \end{cases}
\end{equation}
which can be evaluated inductively starting with $Start$ which has no incoming edges in $\gra$. The resulting expression $C_{Start}$ is a composite service expression that is equivalent to $\gra$.
%Note: This needs to be modified slightly if there can be more nodes without incoming edges (other than Start). 

\begin{example}
Consider the composition task $T=(\{a, b, e\},\{ i\})$. Fig.~\ref{fig:ExampleOfRepresentation} shows an example of a composition solution. It involves four atomic services $S_1=(\{a, b \}, \{c, d, j \}, QoS_{S_1})$, $S_2=(\{c \}, \{f, g \}, QoS_{S_2})$, $S_3=(\{d \}, \{h \}, QoS_{S_3})$, and $S_4=(\{f, g, h \}, \{ i \}, QoS_{S_4})$. The two special services $Start=(\emptyset,\{a,b,e\},\emptyset)$ and $End=(\{i\},\emptyset,\emptyset)$ are defined by the given composition task $T$. The corresponding service composition expression is:

\noindent $\cse_{Start}=\bullet(Start,\bullet( S_1,\parallel(\bullet(S_2,\bullet(S_4,End)),\bullet(S_3,\bullet(S_4,End))))).$
\end{example}

Formal expressions can be visualized by expressions trees. For a composite service expression $\cse$ let $\tree$ denote the corresponding expression tree. 
%Let $V_t$ and $V_f$ denote the leaf nodes (also called \emph{terminal nodes}) and the internal nodes (also called \emph{functional nodes}) of $\tree$. 
Every leaf node in $\tree$ is labelled by the corresponding atomic service, while every internal node in $\tree$ is labelled by the corresponding composition constructor. For the sake of brevity we only consider $\bullet$ and $\parallel$ here, but our approach can easily be extended to $+$ and $\ast$, too. If a subtree of $\tree$ (except for $End$) has an isomorphic copy in $\tree$ then we remove it, label its root with a special symbol $q$, and insert an edge to the root of the copy. As a result we obtain a tree-like representation of a service composition. An example is shown in Fig.~\ref{fig:ExampleOfRepresentation}.

\begin{figure}[h!tb]
\centering
%\fbox{\includegraphics[scale=.30]{synthesisRules.pdf}}
\fbox{\includegraphics[width=1.0\textwidth]{Tree(p-symbol).pdf}}
 \caption{Example of a tree-like representation}
 \label{fig:ExampleOfRepresentation}
\end{figure}

%------------------- synthesis rules
Fig.~\ref{fig:ExampleOfRepresentation} shows for every atomic service $S$ its sets of (least required) inputs $\rin_S$ and outputs $\rout_S$. Moreover, the set of available inputs $\prin_S$ is shown which is just the union of the input sets of all (direct and indirect) predecessors of $S$ in the DAG.
%needs to be explained better
This can be easily generalized to composite service expressions. For a parallel composition $\cse=\parallel(\cse_1,\ldots,\cse_d)$ we define $\rin_\cse=\cup_{k=1}^d\rin_{\cse_k}$, and $\rout_\cse=\cup_{k=1}^d\rout_{\cse_k}$, and $\prin_\cse=\cup_{k=1}^d\prin_{\cse_k}$.
%we actually only need binary sequential compositions with a service in the first position
%not sure if this corresponds to what has been computed !!!
For a sequential composition $\cse=\bullet(S,\cse^\prime)$ we define $\rin_\cse=\rin_S\cup(\rin_{\cse^\prime}-\rout_S)$, and $\rout_\cse=\rout_S\cup\rout_{\cse^\prime}$, and $\prin_\cse=\prin_S$.
%more general we have:
%$\cse=\bullet(\cse_1,\ldots,\cse_d)$ we define $\rin_\cse=\rin_{\cse_1}\cup(\rin_{\cse_2}-\rout_{\cse_1})\cup\cdots\cup(\rin_{\cse_d}-\rout_{\cse_{d-1}})$ and $\rout_\cse=\rout_{\cse_1}\cup\cdots\cup \rout_{\cse_d}$, and $\prin_\cse=\prin_{\cse_1}\cup\cdots\cup \prin_{\cse_d}$. 

%------------------- new example
\begin{example}
%\cse here is actually \cse_{S_4}
Consider the sequential composition $\cse_4=\bullet(S_4, End)$ which is shown in the rightmost position in Fig.~\ref{fig:ExampleOfRepresentation}. We obtain $\rin_{\cse_4}=\{f, g, h\}$ which represents the (least required) inputs for this composition, and $\rout_{\cse_4}=\{i\}$ which represents the outputs produced by this composition, and $\prin_{\cse_4}=\{a, b, e, c, d, j, f, g, h\}$ which represents the union of the input sets of all (direct or indirect) predecessors of $S_4$ in the DAG (i.e., $S_1$, $S_2$ and $S_3$).
\end{example}

%\begin{example}
%Consider two atomic services $S_1$ with inputs $I_{S_1}=\{a,b\}$ and outputs $O_{S_1}=\{c,d\}$, and $S_2$ with inputs $I_{S_2}=\{c,d,e\}$ and outputs $O_{S_2}=\{f,g,h\}$. For their sequential composition $\cse=\bullet(S_1,S_2)$ we obtain $\rin_\cse=\{a,b,e\}$ which represents the least required inputs for this composition, $\rout_\cse=\{c,d,f,g,h\}$ which represents the outputs produced by this composition, and $\prin_\cse=\{a,b,e\}$ which represents the inputs that are provided by the predecessors of.
%this is not yet well described
%\end{example}

%\begin{figure}[h!tb]
%\centering
%%\fbox{\includegraphics[scale=.30]{synthesisRules.pdf}}
%\fbox{\includegraphics[width=.8\textwidth]{synthesisRules.pdf}}
% \caption{Example of representation and synthesis rules}
% \label{fig:ExampleOfSynthesisRules}
%\end{figure}

%Note that $\rin$, $\rout$ and $\prin$ can be computed recursively by tree traversal. For computing $\rin$ and $\rout$ we use pre-order depth-first traversal starting with the left-most leaf node of $Tree$, while for $\prin$ we use level-order breath-first traversal starting with the root node of $Tree$. 

%The representation used by our GP-based approach is a tree, $Tree = \{ V | \mathcal{R} \}$ with a set of nodes $V$ stored in a parent-child relationship $\mathcal{R}$ satisfying the following:
%\begin{enumerate}
%\item The node set $V = \{V_f,V_t\}$ consists of a set of functional nodes (internal nodes), $V_f = \{ \bullet, \parallel \}$, and a set of terminal nodes (leaf nodes), $V_t$=$\{Start, S_1, $ $S_2, \ldots, S_n, $ $End\}$. In $V_t$, $Start$ and $End$ are two special services defined as $Start = (\emptyset, I_T, \emptyset )$ and $End  = (O_T, \emptyset, \emptyset)$ that account for the input and output requirements given by the request $T$. 
 
%\item Each node $v \in V$ in $Tree$ is defined as a tuple $(I_{r}, O_{r}, I_{p}, QoS)$. The attributes $QoS$ are aforementioned in Sect. \ref{Problem Description}, while required inputs $I_r$, required outputs $O_r$ and provided inputs $I_p$ of $v$ are discussed here. $I_r$ is a set of inputs that is needed for executing the service composition represented in the subtree for which $v$ is the root node, and $O_r$ is obtained from the execution. $I_p$ of each $v \in V$ is a set of inputs available for $v$, which are obtained from two resources. One is from the task inputs $I_T$, and the other is from the outputs of $v$'s predecessor services in $Tree$, noted as $I_{pre}$.

%\item Some constrains are on $v \in V$ in $Tree$. The sequence node is always $\bullet(v_1 \in V_t, v_2 \in V)$ including two special cases for $Tree$'s $root$ and sequence constructs at the terminal level: $\bullet(Start, v_2 \in V_f)$ and $\bullet(v_1 \in V_t, End)$. The parallel node is always $\parallel(v_1\in V_f,..,v_n \in V_f)$, where $n\geq 2$.
%\end{enumerate}

Our representation supports composition constructs that are available in commonly used composition languages, such as BPEL4WS or OWL-S. Note that our representation is different from the most widely used tree-based representations in \cite{gupta2015optimization,da2016genetic,yu2013adaptive}. These differences are as follows.
\begin{enumerate}
\item $Start$ and $End$ are included in $\tree$, as they are related to measuring the semantic matchmaking qualities regarding $I_T$ and $O_T$.
\item $\rin_\cse$, $\rout_\cse$, $\prin_\cse$, $QoS_\cse$ are attributes, defined as a tuple $(\rin_\cse, \rout_\cse, \prin_\cse, QoS_\cse )$ for  any  $\cse_S$ in $\tree$. These attributes must be updated after population initialisation and genetic operations described in Sect. \ref{GP-Based Algorithm}.
\item $\tree$ preserves all the semantic matchmaking information, which can be easily used for computing robust casual links.
\end{enumerate}

%Based on and by extending the synthesis rules in \cite{fanjiang2014semantic}, we define a set of rules for computing $I_r$, $O_r$, $I_p$ of any $v \in V$ on $Tree$. The overall rules contain bottom-up and top-down methods to traverse $Tree$ recursively as follow: The bottom-up method for computing $I_r$, $O_r$ starts with the leftmost $v \in V_t$ in $Tree$ and continues to the right, then traverses the parent. On the other hand, the top-down method for calculating $I_p$ starts with the root of $Tree$, continues to traverse the children from the leftmost to the rightmost, then traverses the subtrees where each child is a root. During the traversal, $I_{r}$, $O_{r}$ and $I_{p}$ are updated in different ways according to each type of $v \in V_f$ as demonstrated in Fig. \ref{fig:ExampleOfSynthesisRules} $(b)$ and $(c)$. Note, for any $S(I_S, O_S) \in V_t$, $I_{r_{s}} = I_s$ and $O_{r_{s}} = O_s$.

%\begin{enumerate}
%\item If $v$ is $\bullet$, i.e.  $\bullet (S_1,S_2)$, when $S_1(\{a, b\}, \{c, d\}, QoS_1)$ and $S_2(\{c, d, e\}, \{f, g, h\}, $ $QoS_2)$, for $\bullet$, its $I_r = \{a, b, e \}$ is calculated by $I_{r_{1}} \cup (I_{r_{2}} \setminus  O_{r_{1}}) $, which presents the least required inputs for $\bullet$. Meanwhile, for $\bullet$, its $O_r = \{c, d, f, g, h \}$ is calculated by $O_{r_{1}} \cup O_{r_{2}}$, which presents all the output produced by $\bullet$. Suppose, for $\bullet$, its $I_{p}= \{a, b, e\}$ is inherited from the two resources ($I_T \cup I_{pre}$). For $S_1$, its $I_{p_{1}}= \{a, b, e\}$ as $I_{p}$ is passed to $S_1$. For $S_2$, its $I_{p_{2}} = \{a, b, e, c, d \}$ is calculated by $O_{r_{1}} \cup I_{p}$ as the both the inherited inputs produced $I_{p_{1}}$ and $O_{r_{1}}$ are included.

%\item If $v$ is $\parallel$, i.e. $\parallel(S_1,S_2)$, when $S_1(\{a, b\}, \{c, d\}, QoS_1)$ and $S_2(\{c, d, e\}, \{f,$ $ g, h\}, QoS_2)$, for $\parallel$, its $I_r = \{a, b, c, d, e \}$ is calculated by $ I_{r_{1}} \cup I_{r_{2}}$, which presents all the required inputs for $\parallel$. Meanwhile, for $\parallel$, its $O_r = \{c, d, f, g, h \}$ is calculated by $O_{r_{1}} \cup O_{r_{2}}$, which presents all the output produced by $\parallel$. Same as discussed above, for $\parallel$, its $I_{p} = \{a, b, e\}$ is inherited from the two resources ($I_T \cup I_{pre}$), For $S_1$ and $S_2$, $I_{p_{1}} = I_{p_{2}} = \{a, b, e\}$ as $I_{p}$ is passed to its children.
%\end{enumerate}


To compute semantic matchmaking quality, we need to retrieve all the robust causal links on $\tree$. This is performed by retrieving robust  causal links for every  sequential composition $\cse=\bullet(S,\cse^\prime)$. For example,  in Fig \ref{fig:ExampleOfRepresentation}, two robust causal links ($link_2: S_1 \rightarrow S_2$ and $link_3: S_1 \rightarrow S_3$ ) are retrieved from $\cse_1 = \bullet (S_1, \cse_{\parallel})$, because outputs $O_{S_1}=\{ c, d, j \}$ match inputs $I_{\cse_{\parallel}}=\{c, h, d, f, g\}$.


%\begin{algorithm}
% \setlength\hsize{0.9\linewidth}
% \SetKwInOut{Input}{Input}\SetKwInOut{Output}{Output}
% \SetKwFunction{evaluateSemanticLink}{evaluateSemanticLink}
% \SetKwProg{Procedure}{Procedure}{}{}
% \LinesNumbered
% \SetNlSty{}{}{:}
%  \Procedure{evaluateSemanticLink( ){}}{
% \Input{Tree $T$, IndexCache $cache$}
% \Output{Match type quality $MT$, Concept similarity $S$}
% \{ $ServNode \} \leftarrow  T.getAllServNode$\;
%     \ForEach{$ServNode$ in  \{ $ServNode$ \} }{
%      $NeighbourNode \leftarrow  ServNode.getNeighbourNode$\;
%    	 \ForEach{$O$ in $ServNode.outputs$ and $I$ in $NeighbourNode.requiredInputs$}{
%	  $mt_{P}, s_{P} \leftarrow$ query $cache(O,I)$\;
%	  $mt_{L} \leftarrow$ aggregation( $mt_{p}$ )\;
%	  $s_{L} \leftarrow$ aggregation( $s_{p}$ )\;
%	  \{$mt_{L}$ \} add $mt_{L}$\;
%	  \{$s_{L}$ \} add $s_{L}$\;
%        }
%       $MT \leftarrow$ aggregation( \{ $mt_{L}$\} )\;
%       $S \leftarrow$ aggregation( \{ $s_{L}$ \} )\;
%       }
% \KwRet $MT, S$\;
% }
% \caption{Evaluate semantic link in the tree.}
%\label{evaluateSemanticLink}
%\end{algorithm} 

%========================================================================================= Algorithm
\subsection{GP-Based Algorithm}\label{GP-Based Algorithm}

Now we present our GP-based approach for service composition, see Algorithm~\ref{GP-based algorithm}. To begin with the algorithm, we generate the initial population $P_0$, which is then evaluated using our comprehensive quality model. The iterative part of the algorithm comprises lines 3 to 7,  which will be repeated until the maximum number of generations is reached or the best solution is found. During each iteration, we use tournament selection to select individuals, on which crossover and/or mutation are performed to evolve the polulation. These steps correspond to the standard GP steps \cite{koza1992genetic} except for some particularities that will be discussed below.

%How do we know that we have found the best solution? In the algorithm below: where do we get max.fitness from?

\begin{algorithm}[h!tb]
 %\LinesNumbered
 \SetKwInOut{Input}{Input}\SetKwInOut{Output}{Output}
 \SetKwFunction{generateWeightedGraph}{generateWeightedGraph}
 \SetKwProg{Procedure}{Procedure}{}{}
 \SetNlSty{}{}{:}
 \Input{ $T$, $\mathcal{SR}$, $\mathcal{O}$}
 \Output{an optimal composition solution}
 Initialise population $P_0$ (using a 3-step method)\;
 Evaluate each individual in population $P_0$ (using our comprehensive quality model)\;
  \While {max.populations or max.fitness not yet met}{
     Select the fittest individuals for evolution\;
     Apply crossover and mutation to the selected individuals\;
%     Generate new individuals\;
     Evaluate each new individual\;
     Replace the individuals with the smallest fitness in the population by the new individuals\;
  }
 Find the individual with the highest fitness in the final population\;
\caption{GP-based algorithm for service composition.}
\label{GP-based algorithm}
\end{algorithm} 

\textbf{Population initialisation.}
The initial population is created by generating a set of service compositions in form of DAGs, and then transforming them into their tree-like representations (\emph{the individuals}). The initialisation is performed as follows: 

%Should we explain when a DAG is valid service composition solution?
\textsc{Step 1}. Greedy search is performed to randomly generate a set of DAGs, each representing a (valid) service composition for the given composition task $T$. For this, a simple forward graph building algorithm is applied starting with the node $Start$ and the inputs $I_T$ of the composition task $T$. Details of this algorithm can be found in \cite{ma2015hybrid}. An example of a generated DAG is shown in Fig.~\ref{fig:DAG} with seven robust casual links marked on.

\textsc{Step 2}. The DAGs can be simplified by removing some redundant edges and service nodes. 
%We formalise the criteria for identifying such edges: The outgoing edges from an atomic service $S$ to its direct successors are considered for removal if the direct successors overlap with their successors. This criterion is checked from $Start$ to $End$. 
While this step is not compulsory, it can help to notably reduce the size of the DAG and, consequently, the corresponding tree-like representation. 

\textsc{Step 3}. 
%For each DAG we initialize an auxiliary data structure to record the available inputs for every service node in the DAG. 
We transform each DAG into its tree-like representation using an algorithm modified from \cite{da2016genetic} to satisfy the particular requirements of our proposed approach. For example, Fig.~\ref{fig:ExampleOfRepresentation} shows an example of a tree-like individual corresponding to the DAG shown in Fig.~\ref{fig:DAG}.

%\begin{example}
%Fig.~\ref{DAG} illustrates the generation and simplification of a DAG.  The  corresponding tree shown in Fig.~\ref{fig:ExampleOfRepresentation} is transferred from this DAG. In the initial DAG, service $S_4$ requires the inputs produced by $S_1$, $S_2$ and $S_3$. However, $S_4$ can not be invoked until both $S_2$ and $S_3$ are executed. Therefore, we can simplify the DAG by removing the edge between $S_1$ and $S_4$ (noted as $link_7$) without impacting the execution of a service composition. To transform the DAG into a tree, we traverse the DAG from $Start$ to $End$ with a cursor to mark the current position while creating a node in the tree according to the cursor's position and the number of outgoing edges. In addition, seven robust casual links marked in this DAG could be retrieved from transferred tree in Fig.~\ref{fig:ExampleOfRepresentation} .
%\end{example}

\begin{figure}[htb]
\centering
%\fbox{\includegraphics[scale=.30]{transformationExample.pdf}}
\fbox{\includegraphics[width=.7\textwidth]{DAG(p-symbol).pdf}}
 \caption{Example of a DAG used for transferring it into tree-like representation}
 \label{fig:DAG}
\end{figure}


%\begin{figure}[htb]
%\centering
%%\fbox{\includegraphics[scale=.30]{transformationExample.pdf}}
%\fbox{\includegraphics[width=.8\textwidth]{transformationExample.pdf}}
% \caption{Example for the transformation of the DAG for a service composition into its tree-like representation}
% \label{transformationExample}
%\end{figure}

%This section still needs to be modified. I can describe crossover and mutation similar to the GraphEvol approach (though these are a bit more restrictive than the crossover and mutation considered so far in this paper). Or I can describe it a bit more general or even leave it somewhat open.
%Later we need an example. The solution of Dataset 3 is a possible example, but a bit small. I would like to see the solution of Dataset 4 before I can say what a good example would be.
\textbf{Crossover and Mutation.} 
During the evolutionary process, the correctness of the representation is maintained by crossover and mutation. 

A crossover operation exchanges a subtree of a selected individual (its attributes noted as $\cse_1 (\rin_{\cse_1}, \rout_{\cse_1}, \prin_{\cse_1}, QoS_{\cse_1})$) with the subtree of another selected individual (its attributes noted as $\cse_2 (\rin_{\cse_2}, \rout_{\cse_2}, \prin_{\cse_2}, QoS_{\cse_2})$) 
if they represent the same functionality (i.e. $\rin_{\cse_1} = \rin_{\cse_2}$ and $\rout_{\cse_1} = \rout_{\cse_2}$). That is, at the root nodes of both subtrees, we have identical inputs and identical outputs. A crossover operation is performed in two cases: crossover on two functional nodes or on two terminal nodes. We never exchange a functional node with an terminal node, since the two associated subtrees cannot be equivalent in this case. For example, $End$ must appear in the subtree associated with any functional node, but not for any selected terminal node (atomic services). 
%\begin{figure}[h]
%\centering
%\fbox{\includegraphics[scale=.26]{crossoverExample.pdf}}
% \caption{Example of a crossover}
% \label{crossoverExample}
%\end{figure}

A mutation operation replaces a subtree of the selected individual (its attributes noted as $\cse_1 (\rin_{\cse_1}, \rout_{\cse_1}, \prin_{\cse_1}, QoS_{\cse_1})$) with a newly generated subtree satisfying the least required functionality. To do this, a subtree $\cse_1$ must be selected from the selected individual, and a new composition task $T=(\{\prin_{\cse_1}\},\{\rout_{\cse_1} \cap O_T\})$ or $T'=(\{\prin_{\cse_1}\},\{\rout_{\cse_1}\})$ is used to generate a tree in the same way as the 3-step method performed during the population initialisation. We utilise the available inputs and least required outputs for mutation, because it potentially bring more possibilities in generating more varieties of subtrees. The mutation is performed in two cases: mutation on a functional node with $T$ or on a service node with $T'$, two examples shown in Fig \ref{mutationExample} $(a)$ and $(b)$. In Fig \ref{mutationExample} $(a)$, a functional node $\cse_{\parallel}$ is selected for mutation, the whole subtree is replaced with the generated subtree excluding its head (i.e., $Start$ and its parent node $\bullet$). In Fig \ref{mutationExample} $(b)$, a atomic service $S_1$ is selected for mutation, the branch of the selected node (i.e., $S_1$ and its parent node $\bullet$) is replaced with the generated subtree excluding both its head (i.e., $Start$ and its parent node $\bullet$) and its tail (i.e., $End$).

\begin{figure}[h!tb]
\centering
%\fbox{\includegraphics[scale=.23]{mutationExample.pdf}}
\fbox{\includegraphics[width=\textwidth]{mutationExample.pdf}}
 \caption{Examples of two mutations on terminal and functional nodes}
 \label{mutationExample}
\end{figure}

Note: The set of available nodes considered for crossover  and mutation do not include $Start$ and $End$, and their parent nodes, because these nodes remain the same for all individuals.  In addition, the nodes selected for crossover and mutation must not break the functionality of $q$ symbols. For example, in Fig. \ref{fig:ExampleOfRepresentation}, both sequential composition $\cse_2$ and $\cse_3 $ are not considered for crossover and mutation as they break the edge of $q$ symbol, but the parallel composition $\cse_{\parallel}$ can be considered for genetic operations, as it may bring a new fully functional $q$ symbol or a subtree without $q$ symbol involved. The pointed subtree $\cse_4$ could also be selected for genetic operations. 
%=================================================================================================== Experiments
\section{Experiment Study for GP-based Approach}\label{experiment_study}

We have conducted experiments to evaluate our proposed approach. 
For our experiments we have used the benchmark datasets originating from OWLS-TC \cite{kuster2008opossum} , which have been extended with real-world QoS attributes and five composition tasks \cite{ma2015hybrid}. To explore the effectiveness and efficiency of our proposed GP-based approach, we compare it against one recent GP-based approach \cite{ma2015hybrid}. For that we have extended the later approach by our proposed comprehensive quality model, so that semantic matchmaking quality can be computed based on the parent-child relationship in the underlying tree representations. 

%Describe the computer environment that you have used for your experiments!

To assure a fair comparison we have used exactly the same parameter settings as in \cite{ma2015hybrid}. In particular, the GP population size has been set to 200, the number of generations to 30, the reproduction rate to 0.1, the crossover rate to 0.8, and the mutation rate to 0.1. We have run every experiment with 30 independent repetitions. Without considering any users' true service composition preferences, the weights in fitness function in Eq. (\ref{eq_fitness}) have been are configured simply to balance semantic matchmaking quality and QoS. Particularly, $w_{1}$ and $w_{2}$ are both set to 0.25, while $w_{3}$, $w_{4}$, $w_{5}$, and $w_{6}$ are all set to 0.125. The parameter $p$ of $type_{link}$ is set to 0.75 ($plugin$ match) in accordance with the recommendation in \cite{lecue2009optimizing}. 

%Why is it fair to use the same parameter settings as in \cite{ma2015hybrid}? Wouldn't it be fair to choose parameter settings that we like and then run experiments with both approaches using the same parameter settings?

Our experiments indicate that our method can work consistently well under valid weight settings and parameter $p$ to be decided by users' preferences in practice.

%========================================================================================= Comparison
\subsection{Comparison against a previous GP-based approach}\label{comparisonTest2}

Table~\ref{meanFitness_GP} shows the fitness values obtained by the two GP-based approaches. To compare the results, an independent-samples T-test over 30 runs has been conducted. The results show that our GP-based approach outperforms the previous GP-based approach \cite{ma2015hybrid} in finding more optimized composition solutions for Tasks 3 and 4. (Note: the P-values are lower than 0.0001). For Tasks 1, 2 and 5, both approaches achieve the same fitness. Therefore, the overall effectiveness of our proposed approach is considered to be better. 

%Why have we conducted a independent-samples T-test rather than a Wilcoxon test?
%We should explain better what we want to emphasize by saying "note: the P-values are lower than 0.0001"
\begin{table}[h!tb]
\footnotesize
\centering
\caption{Mean fitness values for our approach in comparison to \cite{ma2015hybrid}\\ (Note: the higher the fitness the better)}
\label{meanFitness_GP}
\begin{tabular}{l|l|l}
\hline
\multicolumn{1}{c|}{Task} & Our GP-based approach                  & Ma et al. approach \cite{ma2015hybrid}  \\ \hline
OWL-S TC1                     & 0.923793 $\pm$ 0.000000                & 0.923793 $\pm$ 0.000000                   \\ \hline
OWL-S TC2                     & 0.933026 $\pm$ 0.000000                & 0.933026 $\pm$ 0.000000                   \\ \hline
OWL-S TC3                     & 0.870251 $\pm$ 0.000000 $\uparrow$     & 0.832306 $\pm$ 0.008241                   \\ \hline
OWL-S TC4                     & 0.798137 $\pm$ 0.007412 $\uparrow$     & 0.760146 $\pm$ 0.005044                   \\ \hline
OWL-S TC5                     & 0.832998 $\pm$ 0.000000                & 0.832998 $\pm$ 0.000000                   \\ \hline
\end{tabular}
\end{table}

Table~\ref{meanTime} shows the execution times observed for the two GP-based approaches. Again an independent-samples T-test over 30 runs has been conducted. For Tasks 1 and 2 both approaches need about the same time, while for Tasks 3,4, and 5 our approach needs slightly more time. (Note: the P-values are lower than 0.0001). However, even in the worst case it exceeds \cite{ma2015hybrid} by no more than 1 second, which is acceptable for most real-word scenarios. Hence, in terms of efficiency our approach is comparable to \cite{ma2015hybrid}.
\begin{table}[h!tb]
\footnotesize
\centering
\caption{Mean execution time (in ms) for our approach in comparison to \cite{ma2015hybrid}\\ (Note: the shorter the time the better)}
\label{meanTime}
\begin{tabular}{l|l|l}
\hline
\multicolumn{1}{c|}{Task} & Our GP-based approach            & Ma et al. approach \cite{ma2015hybrid}     \\ \hline
OWL-S TC1                     & 7396.366667 $\pm$  772.408168    &  7310.866667  $\pm$952.701775                       \\ \hline
OWL-S TC2                     & 2956.133333  $\pm$ 761.350965    &  3036.966667 $\pm$ 777.121101                       \\ \hline
OWL-S TC3                     & 1057.266667  $\pm$ 174.405183    &  763.800000 $\pm$ 221.241232  $\downarrow$          \\ \hline
OWL-S TC4                     & 4479.466667  $\pm$ 519.767172    &  3068.800000 $\pm$ 472.013106  $\downarrow$        \\ \hline
OWL-S TC5                     & 6276.533333  $\pm$ 1075.102328   &  5030.200000 $\pm$ 991.863812 $\downarrow$         \\ \hline
\end{tabular}
\end{table}

The experiments confirm that there is a trade-off between fitness and execution time in GP-based service composition. It can be argued that our proposed approach achieves a better balance as the computed solutions observe a significantly higher fitness while there is a moderate increase in execution time compared to \cite{ma2015hybrid}. 

%========================================================================================= Discussion
\subsection{Further Discussion}
For Tasks 3 and 4, the optimized composition solutions obtained by the two approaches are shown in Fig.~\ref{example1and2}(a) and Fig.~\ref{example1and2}(b), respectively. The functional and nonfunctional descriptions of all services involved in these solutions are listed in Fig.~\ref{example1and2}(c). 

%For Datasets 3 and 5, the optimized composition solutions obtained by the two approaches are shown in Fig.~\ref{example1and2}(a) and Fig.~\ref{example1and2}(b), respectively. The functional descriptions of all services involved in these solutions are listed in Fig.~\ref{example1and2}(c). For Dataset 5 the composition task is $T=( \{ duration, city, country \}, \{ weatherseason, map, hotel\})$. The optimized composition solutions obtained by the two approaches are identical, even though different representations have been used during GP.  

%change "Task" to "Dataset" in the figure,
% there is only one dataset with 5 different tasks
%Should we move the composition task into the figure so that readers don't need to look for it in the text?

\begin{figure}[h!tb]
\centering
%\fbox{\includegraphics[scale=.30]{example1and2.pdf}}
\fbox{\includegraphics[width=.9\textwidth]{experimentExample(p-symbol).pdf}}
 \caption{Example of best solutions using the two GP-based approaches}
 \label{example1and2}
\end{figure}
\noindent For Task 3 the composition task is $T_3=(\{academic$-$item$-$number\},\{ book, maxprice\})$. The best composition solutions obtained by the two approaches are different, see Fig.~\ref{example1and2}. Both solutions have the same semantic matchmaking quality as the matchmaking type of all links is $exact$ match. However, both solutions differ in their QoS. This is due to the different services that are involved: $S_2$ versus $S_3$. The QoS of $S_2$ is much better that of $S_3$. Consequently, the best composition solution obtained by our approach has higher fitness according to our quality model.  It is interesting to observe that the best composition solution obtained by our approach for Task 3 can be evolved from the best composition solution in  \cite{ma2015hybrid} just by a single mutation on $S_3$ using available inputs.


For Task 4 the composition task is $T_4=(\{academic$-$item$-$number\},\{ maxprice, book-type,recommendedpriceindollar\})$. The best solutions obtained by the two approaches are also different, see Fig.~\ref{example1and2}. Note that solution generated by our approach is composed of four atomic services ($S_4$, $S_5$, $S_6$ and $S_2$) while the solution generated by approach \cite{ma2015hybrid} is composed of  five atomic services ($S_4$, $S_5$, $S_6$, $S_1$ and $S_2$). Both solutions have the same semantic matchmaking quality as the matchmaking type of all links is $exact$ match. However, the overall QoS of our approach is better. This is due to the additional $S_1$ in their approach, which has a significant negative impact on QoS.

We observe from above examples that our approach is able to produce better solutions because our proposed representation keeps available inputs and least required outputs of each node on the tree which unlocks more opportunities for 
mutation and crossover rather than restricting them to the previously used inputs and ouputs only.

%\begin{figure}[h!tb]
%\centering
%\fbox{\includegraphics[scale=.30]{example1and2.pdf}}
%\fbox{\includegraphics[width=.9\textwidth]{example1and2.pdf}}
% \caption{Example of best solutions using the two GP-based approaches}
% \label{example1and2}
%\end{figure}

%It may due to the structure of the introduced representation that brings more possibilities of different combined functionality via functional nodes and more flexible mutation operator that utilises available inputs to generate more varieties of subtrees.

%How about the best solutions obtained with both approaches for Dataset 4? Can we look at them?
%=================================================================================================== Conclusion
\section{Summary for GP-based Approach}\label{summary2}

In this work, we introduces a novel GP-based approach to comprehensive quality-aware semantic web service composition. In particular, a tree-like representation is proposed to direct cope with the evaluation of semantic matchmaking quality. Meanwhile, crossover and mutation methods are proposed to maintain the correctness of individuals. The experiment  shows that our proposed approach could effectively produce better solutions in both semantic matchmaking quality and QoS than the existing approach.

\section{Conclusion}\label{summary2}
In the preliminary work, we propose and formalize a novel and effective comprehensive quality model for simultaneously optimize QoS and quality of semantic matchmaking. Apart from that, two novel and effective EC-based methods are proposed for comprehensive quality-aware semantic web service composition. In particular, indirect and direct representations are designed for studying their effectiveness and efficiency along with novel EC-based methods. Overall experimental studies show both two methods reach good performances comparing to existing works. We also find that our PSO-based approach utilizing an indirect representation outperforms our GP-based approach utilizing a direct representation for finding more effective solutions. These findings motivate us to work on a more effective indirect presentation along with PSO-based memetic method.
%=================================================================================================== Bibliography
%The bibliography can be improved. Remove unnecessary bibliographic information from some of the items. Make sure that QoS is not shown as qos.

\chapter{Proposed Contributions and Project Plan}\label{C:plan}
In the previous chapter, some preliminary works have been done to investigate the performances of direct representation and indirect representation on the proposed PSO-based approach and GP-based approach respectively. We have shown that those two approaches outperform some recent existing EC-based approaches using our proposed comprehensive quality evaluation model. To conduct a further research on the remaining subjectives for objective one and the rest three objectives outlined in the proposal, we present the proposed contribution, project plan, timeline and thesis outlines in this chapter.

\section{Proposed Contributions}
This thesis will contribute to the field of semantic web service composition by considering several key composition problems simultaneously, and to the field of Evolutionary Computation techniques by proposing more novel representation and genetic operators. The proposed contributions of this project are listed below:

\begin{enumerate}
 \item Develop a new fully automated semantic web service composition approach for simultaneously optimizing the quality of semantic matchmaking and QoS. We expect that more effective and efficient methods for handling a novel combinatorial optimization problem for semantic web service composition, which is done by relying on effective representations, a comprehensive quality model and a motivated EC-based method with local search or AI planning techniques.

\item Develop EMO-based approach to effectively and efficiently explore the Pareto front of comprehensive quality-aware service composition. Meanwhile, conditional constraints on SLA and customized matchmaking level is expected to effectively and efficiently handled. Apart from that, posteriori preference articulation techniques for user preference on comprehensive quality is also expected to reach the most preferred Pareto solutions. Each of these achievements is not considered in any past research.

\item Develop an EC-based approaches for dynamic web service composition problem. We expect these approaches effectively and efficiently handle composition environment changes. In particular, the changes of QoS, Ontology and service repository (i.e., service failure and new service registration). These changes are not fully studied in the past, and this problem be firstly solved using motivated EC-based techniques.

\item Develop EC-based approaches to a more complex and realistic semantic web service composition based on not only inputs and outputs, but preconditions and postconditions. We expected a general mechanism is developed for supporting all the service composition constructs considering preconditions and effects. Those occasionally considered precondition and effects are neither fully studied for supporting all the composition constructs, nor employed in a fully automated approach.
\end{enumerate}

\section{Overview of Project Plan}

Six milestones for PhD project are defined in the initial research plan shown in Table \ref{tab:projectOverview}. The first phase of this plan has been completed, which comprising of literature view, two initial works on comprehensive quality-aware semantic web service composition and a research proposal writing. The second phase is related to the first objective of this proposal and is currently in progress and covered in Chapter \ref{C:preliminary} of the proposal. The remaining phases are expected to be completed as planned.

\begin{table}
\small
\centering
\caption{The milestones of PhD project plan.}
\vspace{0.2cm}
\begin{tabular}{|c|p{100mm}|l|}
\hline
Phase & Description & Duration (months) \\ \hline
1 & Reviewing literature, developing initial EC-based composition approach, and writing the proposal & 9 (Complete)  \\
2 & Develop hybrid approaches to comprehensive quality-aware automated web service composition & 6 (In progress) \\
3 & Develop multi-objective approaches to optimize the comprehensive quality for fully automated service composition & 7 \\
4 & Develop hybrid techniques to support dynamic semantic web service composition effectively & 8 \\
5 & Develop hybrid approaches for semantic web service compositions based on preconditions and effects & 3 \\
6 & Writing the thesis & 6 \\ \hline
\end{tabular}
\label{tab:projectOverview}
\end{table}

\section{Project Timeline}

Table \ref{tab:projectTimeline} provide an estimated timeline comprising of the minor goals and milestones, which is expected to serve as a guide and be completed through the PhD project.

\begin{table}
\small
\centering
\caption{Project timeline for the remaining 24 months.}
\vspace{0.2cm}
\begin{tabular}{|c|p{50mm}||ccc|ccc|ccc|ccc|}
\hline
& & \multicolumn{12}{c|}{Time in Months} \\ \hline
Phase & Task                                                & 2 & 4 & 6  & 8 & 10 & 12  & 14 & 16 & 18  & 20 & 22 & 24 \\ \hline
n/a & Updating the literature review
                                                            & x & x & x & x & x & x & x & x & x & x & x & x \\ 
2 & Developing indirect representation utilize in a hyper-heuristics methods
                                                            & x & x &    &   &   &    &   &   &    &   &   &   \\
3 & Investigating unconstrained MO optimization for semantic quality-aware service composition
                                                            &   & x & x  &   &   &    &   &   &    &   &   &   \\
3 & Improving performance of MO approach by integrating other techniques  
                                                            &   &   &  x & x &   &    &   &   &    &   &   &   \\
3 & Extending MO approach to handle constraints on SLA and semantic matchmaking level
                                                            &   &   &    & x & x &    &   &   &    &   &   &   \\
3 & Extending MO approach to integrate user preferences
                                                            &   &   &    &   & x & x   &   &   &    &   &   &   \\
4 & Develop EC-based approach to handle changes in QoS and Ontology
                                                            &   &   &    &   &   & x  & x &   &    &   &   &   \\
4 & Develop EC-based approach to handle service failure and new service registration
                                                            &   &   &    &   &   &    & x & x &    &   &   &   \\
5 & Develop EC-based approach to handle preconditions and effects.
                                                            &   &   &    &   &   &    &   & x & x  &   &   &   \\
6 & Writing the first thesis draft  
                                                            &   &   &    &   &   &    &   &   &    & x & x &   \\
6 & Editing the final draft
                                                            &   &   &    &   &   &    &   &   &    &   & x & x \\
\hline
\end{tabular}
\label{tab:projectTimeline}
\end{table}

\section{Thesis Outline}

The following is an outline of the PhD thesis,  in which Chapter 6 may be replaced by Chapter 7 since it is an optimal objective.

\begin{itemize}
 \item \textit{Chapter 1: Introduction}\\
 This chapter covers a problem statement, motivations, research goals, contributions, and organization of the thesis.
 \item \textit{Chapter 2: Literature Review}\\
This chapter initially provides some fundamental concepts for demonstrating the background of service composition. Followed, a comprehensive understanding and analyzing of existing work on the web service composition.  In particular, four research directions for web service composition are investigated: single-objective approach, multi-objective approach, dynamic service composition and semantic web service composition.  However, these techniques are mainly focused on EC-based approaches
 \item \textit{Chapter 3: EC-based Approaches to Comprehensive Quality-Aware Web Service Composition}\\
This chapter will introduce new and effective approaches that combine EC-based techniques and other searching methods. These approaches are developed to effective and efficient handle comprehensive quality-aware fully automated semantic service composition problems. Apart from that, the performance of different direct/indirect representations is also investigated here.
 \item \textit{Chapter 4: EMO-based Approaches to Comprehensive Quality-Aware Semantic Service Composition}\\
This chapter will demonstrate our fully automated multi-objective approaches, which optimizes every quality criteria of comprehensive quality for semantic web service composition. Constraints on SLA and customized matchmaking levels are also considered in the multi-objective approaches. Apart from that, user preferences are integrated into multi-Objective approaches to effectively and efficiently reach the most preferred Pareto solutions.
 \item \textit{Chapter 5: EC-based Approaches to Support Dynamic Semantic Web Service Composition}\\
This chapter will discuss effective and efficient EC-based methods for handling dynamic service composition problems regarding the changes in QoS and Ontology and service repository (i.e., service failure new service registration). Those approaches are compared with existing dynamic service composition approaches, which do not utilize EC-based techniques.
 \item \textit{Chapter 6: EC-based Approaches for Semantic Web Service Compositions Based on Preconditions and Effects}\\
This chapter will discuss semantic service composition problems. We firstly introduce a proposed matchmaking mechanism for preconditions and effects, which fully support different composition constructs, such as sequence, parallel, loop and choice. These problems will be fully automated solved by EC-based methods focusing on improving their effectiveness and efficiency.
 \item \textit{Chapter 7: Conclusions and Future Work}\\
This chapter concludes all the contributions made by our proposed approaches in each objective. In addition, the limitations are also pointed along with the future research directions.
\end{itemize}


\section{Resources Required}

\subsection{Computing Resources}
This research mainly utilises an experimental approach. Due to the high computation of the experiment execution, ECS Grid computing facilities is required to complete these experiments.

\subsection{Library Resources}
The related literature of this research can be found online using the resources provided by Victoria University of Wellington. Apart from that, useful textbook and lecture notes can be also found in university's library.

\subsection{Conference Travel Grants}
Publications to relevant venues in this field are expected throughout this project, therefore travel grants from Victoria University of Wellington are required for key conferences.

%%%%%%%%%%%%%%%%%%%%%%%%%%%%%%%%%%%%%%%%%%%%%%%%%%%%%%%

\backmatter

%%%%%%%%%%%%%%%%%%%%%%%%%%%%%%%%%%%%%%%%%%%%%%%%%%%%%%%


\bibliographystyle{acm}
\bibliography{bibliography}


\end{document}
