\chapter{Proposed Contributions and Project Plan}\label{C:plan}
In the previous chapter, some preliminary works have been done to investigate the performances of direct representation and indirect representation on the proposed PSO-based approach and GP-based approach respectively. We have shown that those two approaches outperform some recent existing EC-based approaches using our proposed comprehensive quality evaluation model. To conduct a further research on the remaining subjectives for objective one and the rest three objectives outlined in the proposal, we present the proposed contribution, project plan, timeline and thesis outlines in this chapter.

\section{Proposed Contributions}
This thesis will contribute to the field of semantic web service composition by considering several key composition problems simultaneously, and to the field of Evolutionary Computation techniques by proposing more novel representation and genetic operators. The proposed contributions of this project are listed below:

\begin{enumerate}
 \item Develop a new fully automated semantic web service composition approach for simultaneously optimizing the quality of semantic matchmaking and QoS. We expect that more effective and efficient methods for handling a novel combinatorial optimization problem for semantic web service composition, which is done by relying on effective representations, a comprehensive quality model and a motivated EC-based method with local search or AI planning techniques.

\item Develop EMO-based approach to effectively and efficiently explore the Pareto front of comprehensive quality-aware service composition. Meanwhile, conditional constraints on SLA and customized matchmaking level is expected to effectively and efficiently handled. Apart from that, posteriori preference articulation techniques for user preference on comprehensive quality is also expected to reach the most preferred Pareto solutions. Each of these achievements is not considered in any past research.

\item Develop an EC-based approaches for dynamic web service composition problem. We expect these approaches effectively and efficiently handle composition environment changes. In particular, the changes of QoS, Ontology and service repository (i.e., service failure and new service registration). These changes are not fully studied in the past, and this problem be firstly solved using motivated EC-based techniques.

\item Develop EC-based approaches to a more complex and realistic semantic web service composition based on not only inputs and outputs, but preconditions and postconditions. We expected a general mechanism is developed for supporting all the service composition constructs considering preconditions and effects. Those occasionally considered precondition and effects are neither fully studied for supporting all the composition constructs, nor employed in a fully automated approach.
\end{enumerate}

\section{Overview of Project Plan}

Six milestones for PhD project are defined in the initial research plan shown in Table \ref{tab:projectOverview}. The first phase of this plan has been completed, which comprising of literature view, two initial works on comprehensive quality-aware semantic web service composition and a research proposal writing. The second phase is related to the first objective of this proposal and is currently in progress and covered in Chapter \ref{C:preliminary} of the proposal. The remaining phases are expected to be completed as planned.

\begin{table}
\small
\centering
\caption{The milestones of PhD project plan.}
\vspace{0.2cm}
\begin{tabular}{|c|p{100mm}|l|}
\hline
Phase & Description & Duration (months) \\ \hline
1 & Reviewing literature, developing initial EC-based composition approach, and writing the proposal & 9 (Complete)  \\
2 & Develop hybrid approaches to comprehensive quality-aware automated web service composition & 6 (In progress) \\
3 & Develop multi-objective approaches to optimize the comprehensive quality for fully automated service composition & 7 \\
4 & Develop hybrid techniques to support dynamic semantic web service composition effectively & 8 \\
5 & Develop hybrid approaches for semantic web service compositions based on preconditions and effects & 3 \\
6 & Writing the thesis & 6 \\ \hline
\end{tabular}
\label{tab:projectOverview}
\end{table}

\section{Project Timeline}

Table \ref{tab:projectTimeline} provide an estimated timeline comprising of the minor goals and milestones, which is expected to serve as a guide and be completed through the PhD project.

\begin{table}
\small
\centering
\caption{Project timeline for the remaining 24 months.}
\vspace{0.2cm}
\begin{tabular}{|c|p{50mm}||ccc|ccc|ccc|ccc|}
\hline
& & \multicolumn{12}{c|}{Time in Months} \\ \hline
Phase & Task                                                & 2 & 4 & 6  & 8 & 10 & 12  & 14 & 16 & 18  & 20 & 22 & 24 \\ \hline
n/a & Updating the literature review
                                                            & x & x & x & x & x & x & x & x & x & x & x & x \\ 
2 & Developing indirect representation utilize in a hyper-heuristics methods
                                                            & x & x &    &   &   &    &   &   &    &   &   &   \\
3 & Investigating unconstrained MO optimization for semantic quality-aware service composition
                                                            &   & x & x  &   &   &    &   &   &    &   &   &   \\
3 & Improving performance of MO approach by integrating other techniques  
                                                            &   &   &  x & x &   &    &   &   &    &   &   &   \\
3 & Extending MO approach to handle constraints on SLA and semantic matchmaking level
                                                            &   &   &    & x & x &    &   &   &    &   &   &   \\
3 & Extending MO approach to integrate user preferences
                                                            &   &   &    &   & x & x   &   &   &    &   &   &   \\
4 & Develop EC-based approach to handle changes in QoS and Ontology
                                                            &   &   &    &   &   & x  & x &   &    &   &   &   \\
4 & Develop EC-based approach to handle service failure and new service registration
                                                            &   &   &    &   &   &    & x & x &    &   &   &   \\
5 & Develop EC-based approach to handle preconditions and effects.
                                                            &   &   &    &   &   &    &   & x & x  &   &   &   \\
6 & Writing the first thesis draft  
                                                            &   &   &    &   &   &    &   &   &    & x & x &   \\
6 & Editing the final draft
                                                            &   &   &    &   &   &    &   &   &    &   & x & x \\
\hline
\end{tabular}
\label{tab:projectTimeline}
\end{table}

\section{Thesis Outline}

The following is an outline of the PhD thesis,  in which Chapter 6 may be replaced by Chapter 7 since it is an optimal objective.

\begin{itemize}
 \item \textit{Chapter 1: Introduction}\\
 This chapter covers a problem statement, motivations, research goals, contributions, and organization of the thesis.
 \item \textit{Chapter 2: Literature Review}\\
This chapter initially provides some fundamental concepts for demonstrating the background of service composition. Followed, a comprehensive understanding and analyzing of existing work on the web service composition.  In particular, four research directions for web service composition are investigated: single-objective approach, multi-objective approach, dynamic service composition and semantic web service composition.  However, these techniques are mainly focused on EC-based approaches
 \item \textit{Chapter 3: EC-based Approaches to Comprehensive Quality-Aware Web Service Composition}\\
This chapter will introduce new and effective approaches that combine EC-based techniques and other searching methods. These approaches are developed to effective and efficient handle comprehensive quality-aware fully automated semantic service composition problems. Apart from that, the performance of different direct/indirect representations is also investigated here.
 \item \textit{Chapter 4: EMO-based Approaches to Comprehensive Quality-Aware Semantic Service Composition}\\
This chapter will demonstrate our fully automated multi-objective approaches, which optimizes every quality criteria of comprehensive quality for semantic web service composition. Constraints on SLA and customized matchmaking levels are also considered in the multi-objective approaches. Apart from that, user preferences are integrated into multi-Objective approaches to effectively and efficiently reach the most preferred Pareto solutions.
 \item \textit{Chapter 5: EC-based Approaches to Support Dynamic Semantic Web Service Composition}\\
This chapter will discuss effective and efficient EC-based methods for handling dynamic service composition problems regarding the changes in QoS and Ontology and service repository (i.e., service failure new service registration). Those approaches are compared with existing dynamic service composition approaches, which do not utilize EC-based techniques.
 \item \textit{Chapter 6: EC-based Approaches for Semantic Web Service Compositions Based on Preconditions and Effects}\\
This chapter will discuss semantic service composition problems. We firstly introduce a proposed matchmaking mechanism for preconditions and effects, which fully support different composition constructs, such as sequence, parallel, loop and choice. These problems will be fully automated solved by EC-based methods focusing on improving their effectiveness and efficiency.
 \item \textit{Chapter 7: Conclusions and Future Work}\\
This chapter concludes all the contributions made by our proposed approaches in each objective. In addition, the limitations are also pointed along with the future research directions.
\end{itemize}


\section{Resources Required}

\subsection{Computing Resources}
This research mainly utilises an experimental approach. Due to the high computation of the experiment execution, ECS Grid computing facilities is required to complete these experiments.

\subsection{Library Resources}
The related literature of this research can be found online using the resources provided by Victoria University of Wellington. Apart from that, useful textbook and lecture notes can be also found in university's library.

\subsection{Conference Travel Grants}
Publications to relevant venues in this field are expected throughout this project, therefore travel grants from Victoria University of Wellington are required for key conferences.