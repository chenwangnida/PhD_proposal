\chapter{Proposed Contributions and Project Plan}\label{C:plan}
In the previous chapter, some preliminary works have been done to investigate the performances of direct representation and indirect representation in the proposed PSO-based approach and GP-based approach respectively. We have shown that those two approaches outperform some recent existing EC-based approaches using our proposed comprehensive quality evaluation model. To conduct a further research on the remaining objectives outlined in the proposal, we present the proposed contribution, project plan, timeline and thesis outlines in this chapter.

\section{Proposed Contributions}
This thesis will contribute to the field of semantic web service composition by considering several key composition problems simultaneously, and to the field of Evolutionary Computation techniques by proposing more novel representations and genetic operators. The proposed contributions of this project are listed below:

\begin{enumerate}
 \item Develop fully automated semantic web service composition approaches for simultaneously optimizing the quality of semantic matchmaking and QoS. We expect more effective and efficient methods to be developed for handling this novel combinatorial optimization problem, which is done by developing effective representations, a comprehensive quality model and EC-based methods with local search or AI planning techniques.

\item Develop fully automated EMO-based approaches to effectively and efficiently explore the Pareto Front to comprehensive quality-aware service composition. Meanwhile, conditional constraints on SLA and customized matchmaking level are expected to be effectively and efficiently handled. Apart from that, posteriori preference articulation techniques for user preference on comprehensive quality are also to be developed for reaching the most preferred Pareto Front. Each of these achievements is not completed in any past research.

\item Develop fully automated EC-based approaches for dynamic web service composition. These approaches is expected to effectively and efficiently handle composition environment changes. That is, changes in QoS, ontology and service repository (i.e., service failure and new service registration). These dynamic problems are not fully studied in the past research, and will be firstly solved using EC-based techniques.

\item Develop fully automated EC-based approaches to comprehensive quality-aware semantic web service composition supporting preconditions and postconditions. We expected a general mechanism to be developed for satisfactions on preconditions and effects, and for supporting loop and choice composition constructs. Various composition constructs, preconditions and effects support and comprehensive quality optimization are not simultaneously handled in any existing research.
\end{enumerate}

\section{Overview of Project Plan}

Six milestones for PhD project are defined in the initial research plan shown in Table \ref{tab:projectOverview}. The first phase of this plan has been completed, which comprising of literature view, two initial works on comprehensive quality-aware semantic web service composition and a research proposal writing. The second phase is related to the first objective of this proposal that is currently in progress and covered in Chapter \ref{C:preliminary} of the proposal. The remaining phases are expected to be completed as planned.

\begin{table}
\small
\centering
\caption{The milestones of PhD project plan.}
\vspace{0.2cm}
\begin{tabular}{|c|p{100mm}|l|}
\hline
Phase & Description & Duration (months) \\ \hline
1 & Reviewing literature, developing initial EC-based composition approach, and writing the proposal & 9 (Complete)  \\
2 & Develop EC-based approaches to comprehensive quality-aware automated semantic web service composition & 6 (In progress) \\
3 & Develop EMO-based approaches to comprehensive quality-aware automated semantic web service composition & 7 \\
4 & Develop EC-based techniques to support dynamic semantic web service composition & 8 \\
5 & Develop EC-based approaches to comprehensive quality-aware automated semantic web service composition supporting preconditions and effects & 3 \\
6 & Writing the thesis & 6 \\ \hline
\end{tabular}
\label{tab:projectOverview}
\end{table}

\section{Project Timeline}

Table \ref{tab:projectTimeline} provide an estimated timeline comprising of the minor goals and milestones, which is expected to serve as a guide and be completed through the PhD project.

\begin{table}
\small
\centering
\caption{Project timeline for the remaining 24 months.}
\vspace{0.2cm}
\begin{tabular}{|c|p{50mm}||ccc|ccc|ccc|ccc|}
\hline
& & \multicolumn{12}{c|}{Time in Months} \\ \hline
Phase & Task                                                & 2 & 4 & 6  & 8 & 10 & 12  & 14 & 16 & 18  & 20 & 22 & 24 \\ \hline
n/a & Updating the literature review
                                                            & x & x & x & x & x & x & x & x & x & x & x & x \\ 
2 & Developing indirect representations utilize in a hyper-heuristics methods
                                                            & x & x &    &   &   &    &   &   &    &   &   &   \\
3 & Investigating unconstrained EMO-based approaches for comprehensive quality-aware service composition
                                                            &   & x & x  &   &   &    &   &   &    &   &   &   \\
3 & Improving performance of EMO-based approaches by integrating other techniques  
                                                            &   &   &  x & x &   &    &   &   &    &   &   &   \\
3 & Extending EMO-based approaches to handle constraints on SLA and semantic matchmaking level
                                                            &   &   &    & x & x &    &   &   &    &   &   &   \\
3 & Extending EMO-based approaches to integrate preference articulation techniques
                                                            &   &   &    &   & x & x   &   &   &    &   &   &   \\
4 & Develop EC-based approaches to handle changes in QoS and ontology
                                                            &   &   &    &   &   & x  & x &   &    &   &   &   \\
4 & Develop EC-based approaches to handle service failures and new service registration
                                                            &   &   &    &   &   &    & x & x &    &   &   &   \\
5 & Develop EC-based approaches to handle preconditions and effects.
                                                            &   &   &    &   &   &    &   & x & x  &   &   &   \\
6 & Writing the first thesis draft  
                                                            &   &   &    &   &   &    &   &   &    & x & x &   \\
6 & Editing the final draft
                                                            &   &   &    &   &   &    &   &   &    &   & x & x \\
\hline
\end{tabular}
\label{tab:projectTimeline}
\end{table}

\section{Thesis Outline}

The following is an outline of the PhD thesis,  in which Chapter 6 might be replaced by Chapter 7 since it is an optionally objective.

\begin{itemize}
 \item \textit{Chapter 1: Introduction}\\
 This chapter covers a problem statement, motivations, research goals, contributions, and organization of the thesis.
 \item \textit{Chapter 2: Literature Review}\\
This chapter initially provides some fundamental concepts for demonstrating the background of service composition. Followed, a comprehensive understanding and analyzing of existing work on the web service composition.  In particular, four research directions for web service composition are investigated: single-objective approaches, multi-objective approaches, dynamic service composition and web service composition supporting preconditions and effects.
 \item \textit{Chapter 3: EC-based Approaches to Comprehensive Quality-Aware Web Service Composition}\\
This chapter will introduce new EC-based approaches that combine local search and/or AI planing methods. These approaches are developed to effective and efficient handle comprehensive quality-aware fully automated semantic service composition problems. Apart from that, the effectiveness of different direct/indirect representations are also investigated here.
 \item \textit{Chapter 4: EMO-based Approaches to Comprehensive Quality-Aware Semantic Service Composition}\\
This chapter will demonstrate our fully automated EMO-based approaches, optionally combining local search and/or AI planing methods, which optimizes different quality criteria within the comprehensive quality. Constraints on SLA and customized matchmaking levels are also handled in the multi-objective approaches. Apart from that, preference articulation techniques for multi-objective comprehensive quality-aware semantic web service composition are also demonstrated here.
 \item \textit{Chapter 5: EC-based Approaches to Support Dynamic Semantic Web Service Composition}\\
This chapter will discuss effective and efficient EC-based methods for handling dynamic service composition problems regarding the changes in QoS and Ontology and service repository (i.e., service failure new service registration). Those approaches are compared with existing dynamic service composition approaches, which do not utilize EC-based techniques.
 \item \textit{Chapter 6: EC-based Approaches for Semantic Web Service Compositions Supporting Preconditions and Effects}\\
This chapter will discuss service composition supporting preconditions and effects. We firstly introduce a proposed matchmaking mechanism for preconditions and effects, which fully support different composition constructs, such as sequence, parallel, loop and choice. New and effective representations are introduced here to cope with preconditions and effects and utilized in our EC-based approaches, which are optionally combined with local search or AI planning to simultaneously handles various composition constructs and comprehensive quality optimization. 
\item \textit{Chapter 7: Conclusions and Future Work}\\
This chapter concludes all the contributions made by our works completed for each objective. In addition, limitations are also pointed out along with the future research directions.
\end{itemize}


\section{Resources Required}

\subsection{Computing Resources}
This research mainly utilizes an experimental approach. Due to the high computation of the experiment execution, ECS Grid computing facilities are required to complete these experiments.

\subsection{Library Resources}
The related literature of this research can be found online using the resources provided by Victoria University of Wellington. Apart from that, useful textbook and lecture notes can be also found in university's library.

\subsection{Conference Travel Grants}
Publications to relevant venues in this field are expected throughout this project, therefore travel grants from Victoria University of Wellington are required for key conferences.