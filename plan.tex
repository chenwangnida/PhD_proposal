\chapter{Proposed Contributions and Project Plan}\label{C:plan}

\section{Proposed Contributions}

This thesis will contribute to the field of Evolutionary Computation by proposing novel adaptations to candidate representations in an evolutionary algorithms, and to the field of Web service composition by considering multiple composition facets simultaneously. The proposed contributions of this project are listed below:

\begin{enumerate}
 \item Present a new fully automated Web service composition approach that relies on a combination of planning and EC techniques. This is expected to handle composition tasks that require conditional constraints and quality optimisation simultaneously, a problem that has not yet been investigated by researchers in the area.

 \item Extend a many-objective Web service composition approach to handle solutions with conditional constraints, particularly in the case of SLA-aware Web service composition. This extension is expected to enable the optimisation of distinct execution paths independently, meaning that SLA constraints can be verified for each possible execution path. This extension will contribute to an under-explored area in the field of Web service composition.

 \item Present an EC-based approach for the semantic selection of Web services in fully automated composition scenarios, as opposed to relying on the typical semi-automated usage where an abstract workflow has already been provided. This novel approach is expected to improve the selection performance of fully automated composition techniques, which currently rely on exhaustive similarity calculations.

 \item Introduce the use of Evolutionary Computation techniques in dynamic Web service composition scenarios, leveraging the population diversity intrinsic to these approaches.
 As the composition environment changes and solutions are re-optimised and adjusted, the use of EC techniques is expected to yield solutions with better quality levels than those produced by approaches that do not perform re-optimisation.
\end{enumerate}

\section{Overview of Project Plan}

Six overall phases have been defined in the initial research plan for this PhD project, as shown in Table \ref{tab:projectOverview}. The first phase, which comprises reviewing the relevant literature, investigating an initial Web service composition approach, and producing the proposal, has been completed. The second phase, which corresponds to the first objective of the thesis, is currently in progress and is expected to be finished on time, thus allowing the remaining phases to also be carried out as planned.

\begin{table}
\small
\centering
\caption{Overall phases of project plan.}
\vspace{0.2cm}
\begin{tabular}{|c|p{100mm}|l|}
\hline
Phase & Description & Duration (months) \\ \hline
1 & Reviewing literature, developing initial planning/EC composition approach, and writing the proposal & 10 (Complete)  \\
2 & Developing direct, indirect, and hybrid planning EC representations & 5 (In progress) \\
3 & Investigating many-objective optimisation techniques for the planning/EC approach & 6 \\
4 & Developing a semantic selection method suitable to planning-based composition approaches & 4 \\
5 & Adapting the composition approach created so far to function in a dynamic environment & 5 \\
6 & Writing the thesis & 6 \\ \hline
\end{tabular}
\label{tab:projectOverview}
\end{table}

\section{Project Timeline}

The phases included in the plan above are estimated to be completed following the timeline shown in Table \ref{tab:projectTimeline}, which will serve as a guide throughout this project. Note that part of the first phase has already been done, therefore the timeline only shows the estimated remaining time for its full completion.

\begin{table}
\small
\centering
\caption{Project timeline for the remaining 24 months.}
\vspace{0.2cm}
\begin{tabular}{|c|p{50mm}||ccc|ccc|ccc|ccc|}
\hline
& & \multicolumn{12}{c|}{Time in Months} \\ \hline
Phase & Task  & 2 & 4 & 6  & 8 & 10 & 12  & 14 & 16 & 18  & 20 & 22 & 24 \\ \hline
n/a & Updating the literature review
& x & x & x & x & x & x & x & x & x & x & x & x \\ 
2 & Developing direct solution representation
& x &   &    &   &   &    &   &   &    &   &   &   \\
2 & Developing indirect and hybrid solution representations
& x & x &    &   &   &    &   &   &    &   &   &   \\
3 & Investigating unconstrained MO optimisation
&   & x &  x & x &   &    &   &   &    &   &   &   \\
3 & Extending MO approach to handle SLA constraints
&   &   &    &   & x &    &   &   &    &   &   &   \\
4 & Designing and implementing semantic selection technique
&   &   &    &   & x & x  & x &   &    &   &   &   \\
5 & Extending planning/EC approach to handle QoS changes
&   &   &    &   &   &    & x & x &    &   &   &   \\
5 & Creating a strategy for handling service failure
&   &   &    &   &   &    &   & x & x  &   &   &   \\
6 & Writing the first draft of the thesis 
&   &   &    &   &   &    &   &   &    & x & x &   \\
6 & Editing the final draft
&   &   &    &   &   &    &   &   &    &   & x & x \\
\hline
\end{tabular}
\label{tab:projectTimeline}
\end{table}

\section{Thesis Outline}

The completed thesis will be organised into the following chapters:

\begin{itemize}
 \item \textit{Chapter 1: Introduction}\\
 This chapter will introduce the thesis, providing a problem statement and motivations, defining research goals and contributions, and outlining the structure of this dissertation.
 \item \textit{Chapter 2: Literature Review}\\
 The literature review will examine the existing work on Web service composition, discussing the fundamental concepts in this field in order to provide the reader with the necessary background. Multiple sections will then follow, each of them analysing the problem from a different angle, considering issues such as QoS-aware composition, semantic selection, and composition methods for dynamic environments. The focus of this review is on investigating composition techniques that are based on Evolutionary Computation and on AI planning.
 \item \textit{Chapter 3: A Hybrid Planning/EC Approach to Web Service Composition}\\
 This chapter will introduce a new approach to Web service composition that combines elements of AI planning with Evolutionary Computation techniques. One of the critical aspects of this new approach is the representation of composition candidates, therefore multiple representation possibilities will be analysed and compared to identify the most suitable model. 
 \item \textit{Chapter 4: Many-Objective Optimisation of Compositions with Multiple Paths}\\
 In this chapter, many-objective optimisation techniques will be employed to allow each quality dimension of compositions to be improved independently. After investigating these techniques in an unconstrained manner, SLA constraints will also be considered. While these constraints add to the complexity of the problem, they may also prove useful when filtering potential solutions, thus experiments comparing the constrained and unconstrained approaches will be conducted.
 \item \textit{Chapter 5: Semantic Web Service Selection in a Planning-Based Composition Technique}\\
 A novel semantic approach for selecting which services are to be included in the composition will be proposed in this chapter. This approach focuses on selecting candidate services to be used in a planning-based composition technique without relying on precalculated semantic distances, as existing approaches have done. A comparison will be performed between this novel approach and the those that rely on precalculations.
 \item \textit{Chapter 6: EC-Based Compositions in a Dynamic Environment}\\
 This chapter will extend the composition technique discussed throughout this thesis to work in dynamic environments. Firstly, a way of re-optimising solutions as quality values fluctuate will be proposed, and then an approach for responding to service faults will be discussed. The extended technique will be compared against the existing dynamic composition approaches, which do not use EC techniques, to establish whether there are quality gains in the solutions produced.
 \item \textit{Chapter 7: Conclusions and Future Work}\\
 In this chapter, conclusions will be drawn from the analysis and experiments conducted in the different phases of this research, and the main findings for each one of them will be summarised. Additionally, future research directions will be discussed.
\end{itemize}


\section{Resources Required}

\subsection{Computing Resources}
An experimental approach will be adopted in this research, entailing the execution of experiments that are likely to be computationally expensive. The ECS Grid computing facilities can be used to complete these experiments within reasonable time frames, thus meeting this requirement.

\subsection{Library Resources}
The majority of the material relevant to this research can be found online, using the university's electronic resources. Other works may either be acquired at the university's library, or by soliciting assistance from the Subject Librarian for the fields of engineering and computer science.

\subsection{Conference Travel Grants}
Publications to relevant venues in this field are expected throughout this project, therefore travel grants from the university are required for key conferences.
