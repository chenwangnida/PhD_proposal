\chapter{Introduction}\label{C:intro}

\section{Problem Statement}
\emph{Service-oriented computing} (SOC) is a novel computing paradigm that employs services as fundamental elements to achieve the agile development of cost-efficient and integrable enterprise applications in heterogeneous environments \cite{papazoglou2003service, papazoglou2006p}. One of the primary purposes of SOC is to overcome conflicts due to diverse platforms and programming languages to make integrable and seamless communication among those existing or newly built independent services. \emph{Service Oriented Architecture} (SOA)  could abstractly realise service-oriented paradigm of computing. This accomplishment has been contributing to reuse of software components, from the concept of functions to units and from units to services during the evolution of development in SOA \cite{booth2004web, overdick2007resource}. One of the most typical implementation of SOA is \emph{web service}, which is designated as ``modular, self-describing, self-contained applications that are available on the Internet" \cite{curbera2001web}. Several standards play a significant role in registering, enquiring and grounding web services across the web, such as UDDI \cite{curbera2002unraveling}, WSDL \cite{lausen2007semantic} and SOAP \cite{fensel2011semantic}. \emph{Web service composition} aims to loosely couple a set of web services to provide a value-added composite service that accommodates complex functional and non-functional requirements of service users. 

Two most notable challenges for web service composition are ensuring interoperability of services and achieving Quality of Service (QoS) optimisation \cite{fensel2011semantic}. \emph{Interoperability} of web services presents challenges in syntactic and semantic dimensions. On the one hand, the syntactic dimension is covered by the XML-based technologies \cite{yu2008deploying}, such as $WSDL$ and $SOAP$. In this dimension, most services are composed together merely based on the matching of input-output parameters. On the other hand, the semantic dimension enables a better collaboration through ontology-based semantics \cite{o2005review}, in which many standards have been established, such as OWL-S \cite{martin2004owl}, Web Service Modeling Ontology (WSMO) \cite{lausen2005w3c}, SAWSDL \cite{kopecky2007sawsdl}, and Semantic Web Services Ontology (SWSO) \cite{petrie2016web}. This dimension brings around some other underlying functionality of services (i.e., precondition and postcondition) that could effect the execution of web services and their composition. The interoperability challenge gives birth to \emph{Semantic web services composition}, as a different technique from traditional service composition methods (i.e., only syntactic dimension is presented in web services). The interoperability of services is presented as a semantic matchmaking relationship between provided information (i.e one service output) and required information (i.e., another service input), which do not often perfectly match with each other based on their semantic descriptions. Therefore, the quality of matchmaking becomes an aspect of the quality concern for users, which also raise researchers' interests. Another challenge is related to QoS optimisation, where QoS represents non-functional attributes of service composition (e.g, cost, time, reliability and availability). Often, a global search method is employed to minimise the cost and maximise the reliability of composite services. This challenge gives birth to \emph{QoS-aware service composition} that aims to find composition solutions with optimised QoS. Furthermore, QoS-aware service composition problem is facilitated by \emph{Service Level Agreement (SLA)} \cite{sahai2002automated}, i.e., binding constraints on QoS. This results in the \emph{SLA-aware web service composition}, e.g., constraints on cost, execution time, availability and reliability are separately specified with both lower and upper bounds.

Apart from the two notable challenges discussed above, the environment of service composition is changing in the real world, rather than \emph{static}. For example, QoS values of services being composed of are fluctuating over time. Services chosen at the planning stage may not available to be invoked at the runtime, it may due to imprecise interface description \cite{ishikawa2011bridging} or hardware failures \cite{guinard2009discovery}. New services can also be registered in the service repository from time to time. Existing techniques for \emph{static web service composition} can not support such a changing environment. Therefore, \emph{dynamic web service composition} become a very demanding research field with a growing interest and a practical value. Particularly, some mechanisms are required to automatically detect the changes or recover from the faults \cite{chan2009fault}. Additionally, in the context of semantic web service composition, semantics of web services can make the problem of dynamic web service composition more complicated due to the changes in the ontology.


Different service composition approaches \cite{da2016genetic,da2016particle,gupta2015optimization,lecue2009optimizing,ma2015hybrid,qi2010combining,rodriguez2010composition,yu2013adaptive,wang2014automated} have been proposed to cope with those composition challenges discussed above and they can be classified into two main categories: \emph{semi-automated web service composition} and \emph{fully automated web service composition}. The first category of composition approach requires human beings to manually create abstract service workflows. Generally, researchers assume a pre-defined abstract workflow is given and provided. The optimisation problem in this approach turns to selecting concrete services with the best possible quality to each abstract service slot in a given workflow. Due to a tremendous growth in industries and enterprise applications, the number of web services has increased dramatically and unprecedentedly. The process of designing abstract workflows  manually is fraught with difficulties in effectively and efficiently solving composition problems, such as QoS-aware service composition problem. Therefore, full automation of composition process is introduced in web service composition for less human intervention, less time consumption, and high productivity \cite{rao2004survey}. The differences in fully automated approach is that an abstract workflow is not provided, but generated while service are being selected. 


Generating composition plans automatically in discovering and selecting suitable web services is an NP-hard problem \cite{moghaddam2014service}, which means the near-optimised composition solution is not likely to be found in a reasonable computation time. \emph{Artificial Intelligence (AI) planning-based approaches}, \emph{Evolutionary Computation (EC) techniques} and hybrid techniques have been introduced. AI planning is utilised to solve the automated web service composition problems as a plan making process, from initial states to a set of actions to desired goal states-composite web services, where services are considered as actions triggered by one state (i.e., inputs) and resulted in another state (i.e., outputs). In the second approach, heuristics have been employed to generate near-optimal solutions using a variety of EC techniques have been used in this context, e.g., Genetic Algorithms (GA), Genetic Programming (GP) and Particle Swarm Optimisation (PSO). EC-based techniques have been effectively developed to solve the \emph{QoS-aware web service composition} problem. Many representations have been carefully investigated in QoS-aware web service composition problems, since they could significantly affect the performance of fully automated service composition approaches. In the third approach, a hybrid of AI planning-based approaches and EC-based approaches \cite{da2016genetic,ma2015hybrid} have been proposed to ensure the correctness in constructing workflows and to optimise the quality of composition solutions (e.g., QoS) at the same time. From the literature, hybrid approaches generally outperform EC-based methods in finding optimal solutions in the domain of automated QoS-aware web service composition. It becomes a very promising direction to investigate. As summarised here, a few previously addressed challenges for fully automated service compositions lies in jointly optimising QoS and semantic matchmaking quality, dealing with multi-objective optimisation, handling environment changes in dynamic service composition, and composing semantic web services. Those challenges are addressed with key limitations in Section \ref{C:motivation}. 

The overall goal of this thesis is to propose effect and efficient approaches to comprehensive-quality aware automated web service composition. This comprehensive quality aims to jointly optimise semantic matchmaking quality and QoS. Meanwhile, this new approach also tackles several service composition problems, such as multi-objective optimisation, dynamic web service composition and semantic web service composition based on preconditions and postconditions.

\section{Motivations}\label{C:motivation}
The motivations of this proposed research lies in the requirements from five key aspects that simultaneously account for. 
\begin{enumerate*}
 \item \emph{Various techniques of hybridisation}.
 \item \emph{Hybridisation of quality of semantic matchmaking and quality of service}.
 \item \emph{Muti-objective optimisation}.
 \item \emph{Dynamic semantic service composition}.
 \item \emph{Service composition based on preconditions and effects}.
\end{enumerate*}
Herein these requirements are explicitly discussed below. 
\subsubsection{Various Techniques of Hybridisation}

Various techniques have been utilised to solve service composition problems, such as AI planning, local searching and EC-based techniques \cite{feng2013dynamic,parejo2008qos,qi2010combining,wang2014automated}. AI planning is a prominent technique for handling web service composition problems while always ensuring the correctness of the solution (i.e., all the inputs of involved services are satisfied) \cite{wang2014automated}.  Local search is a exhaustive search technique for solving optimisation problems. In local search, solutions keep moving to the neighbour solutions, driven by some local maximisation criterion until near-optimal solution found \cite{parejo2008qos}. However, this technique has the shortage of being trapped in local optimal. On the other hand, EC-based techniques are outstanding at solving combinatorial optimisation problems for finding the globally optimised solutions in large searching spaces. To take the benefits from various techniques,  various techniques of hybridisation allows escaping local optima easily and improving the rate of convergence rate \cite{renders1996hybrid}.

Traditionally, various techniques are employed to handle service composition problem independently in existing works. Many researchers investigated AI planning techniques for service composition problems using classical planning algorithm, where inputs, outputs, preconditions and effects are well defined along with the actions (i.e, services) \cite{markou2015non,peer2005web}. On the one hand, AI planning ensures both the correctness of functionality and satisfactory of constrains, but it is always considered to be less efficient and less scalable, incapable of solving complicated optimisation problems \cite{parejo2008qos}. On the other hand, some researchers combined AI planning and local search to handle optimisation problems, e.g., a combination of Graphplan \cite{blum1997fast} and Dijkstra’s algorithm is proposed by \cite{feng2013dynamic} to achieve a correct solution with optimised QoS. Yet, many EC techniques have been utilised to handle service composition problems for global optimal solutions. A few researchers also combine both local search and EC-based techniques for efficiently finding composition solution with optimised QoS \cite{parejo2008qos}. From the techniques discussed above, they are problem-specific for either optimising QoS or number of services. In this thesis, more complicated and realistic service composition problems are addressed. To cope with this composition problems featured in all the motivations discussed below (i.e comprehensive quality-aware service composition, multi-objective optimisation, dynamic service composition, and composing semantic web services based on preconditions and postconditions ), new hybrid methods are proposed to adapt in the problems specified in the thesis. For example, one heuristic in a EC-based technique is initially utilised for a purpose of global search, then another heuristic is additionally designed for a purpose of a local search, which determining neighbourhood of composition solutions. This hybrid methods may lead to more effective performance comparing those using only one individual technique. 

\subsubsection{Comprehensive Quality of Semantic Web Service Composition}\label{SC:hybridisation}
Web service compositions are optimised by the well known non-functional attributes (i.e, QoS), when services are composed together based on outputs provided by one service and inputs required by another service. In the domain of semantic web services, often, the provided information (i.e., outputs) does not perfectly match the required information (i.e., inputs) according to their semantic descriptions \cite{lecue2008optimizing}. The quality of the matches (i.e., quality of semantic matchmaking) are one part of the goal for achieving service compositions \cite{lecue2009optimizing}. Therefore, a comprehensive quality is proposed for simultaneously considering both QoS and semantic matchmaking quality regarding a combinatorial optimisation problem in web service composition. One motivated example from a practical perspective is explained here: many different service compositions can meet a user request but differ significantly in terms of QoS and semantic matchmaking quality. For example, in the classical travel planning context, some component service must be employed to obtain a travel map. Suppose that two services can be considered for this purpose. One service $S$ can provide a street map at a price of 6.72. The other service $S'$ can provide a tourist map at a price of 16.87. Because in our context a tourist map is more desirable than a street map, $S'$ clearly enjoys better semantic matchmaking quality than $S$ but will have negative impact on the QoS of the service composition (i.e., the price is much higher). One can easily imagine that similar challenges frequently occur when looking for service compositions. Hence, a good balance between QoS and semantic matchmaking quality is called for.

Existing works on service composition focus mainly on addressing only one quality aspect discussed above. For the semantic matchmaking quality, it is mainly addressed in works that focus on the discovery of atomic services, i.e., one-to-one matching of user requirements to a single service. Some works \cite{bansal2016generalized,boustil2014semantic,mier2015integrated} on semantic service composition utilise semantic descriptions of web services (e.g., description logic) to ensure the interoperability of web services, but the goal of the composition is often to minimise the number of services or the size of a graph representation for a web service composition. These approaches do not guarantee an optimised QoS of service compositions. On the other hand, huge efforts have been devoted to studying QoS-aware web service composition \cite{da2015graphevol,da2016particle,gupta2015optimization,ma2015hybrid,qi2010combining,yu2013adaptive}. Some of these works do consider different semantic matchmaking types (e.g., Exact and Plugin matches \cite{paolucci2002semantic}) when they compose services, but do not recognise the importance of semantic matchmaking quality regarding different matchmaking types. However, it is not sufficient to only consider one quality aspect for optimising service composition. For these reasons, there is a need to device an comprehensive quality model for jointly optimising the two quality aspects. Apart from that, existing approaches utilise representations that may not effectively and efficiently handle both QoS and quality of semantic matchmaking. Therefore, new approaches need to be proposed to encode the required information of the two quality aspects, QoS and quality of semantic matchmaking quality. 


\subsubsection{Multi-Objective Composition Optimisation}
Existing approaches for handling web service composition problems fall into two groups, depending on the different goals of optimisation for either a single objective or multiple objectives. In single-objective service compositions, one composition solution is always returned by a composition task, where the preferences of each quality component within the single objective (e.g., a weighted sum of different quality criteria) is known by users. However, users do not always have clear preferences when many quality criteria are presented. Therefore, multi-objective is a natural requirement from users to provide a set of trade-off solutions that concern about the conflicting and independent quality criteria. For example, premium users do not care cost as much as price-sensitive users do, so premium users may prefer a composition solution with lowest execution time with no limit on the budget,  rather than one with a relatively lower execution time without exceeding a budget. Therefore, a multi-objective  fully automated service composition approach can be used to generate a set of solutions due to the following two reasons: first, the preferences of different quality is not clear and hard to determine in advance; second, single-objective optimisation using weighted sum method can not reach solutions in the non-convex regions \cite{kim2006adaptive}.  

Existing research on the automated web service composition mainly concentrates on single objective problems for QoS-aware web service compositions. For example, there is only one solution promoted by a unified QoS ranking score to the users. However, to our best knowledge, existing works on multi-objective service composition \cite{liu2005dynamic,wada2012e3,yao2009qos,yin2014hybrid} are only approached by semi-automated methods to handle the conflicting QoS attributes independently, where the workflow structure is assumed to be pre-existing. On the one hand, simultaneously constructing workflows and selecting proper services for optimising multi-objectives is a very challenge work to complete. On the other hand, some constraints on SLA are also employed to some of these approaches to reach the solutions with a desirable level. These constrains raised the complexity of absolute Preto priority relation \cite{garey1979guide}. From above discussion, these is a lack of fully automated approaches to multi-objective web service composition problems for QoS-aware web service composition abiding by constrains on SLA. Moreover, the insufficiency of handling only non-functional attributes (i.e., QoS) has given rise to adding semantic matchmaking quality into simultaneous consideration.

\subsubsection{Dynamic Semantic Web Service Composition}
In a dynamic environment, QoS of atomic services in a service repository is fluctuating over time. Static service composition solution is no longer enough, and requisite actions must be taken if the original composition solution changes in QoS or is not be executable due to any service involved goes offline. Apart from that, newly registered services could also could alter the composition plan as it could significantly contribute to the overall QoS or quality of semantic matchmaking. Therefore, dynamic web service composition is proposed to effectively and efficiently update composition solutions when they are not presented as optimal and/or executable solutions \cite{li2014fault}. 


Most approaches work on effective and efficient methods for service re-selection for each services employed, which do not allow the changes of service composition structure. Apart from that, the cost of initial planning is ignored and separated from the adaption of dynamic environment. Some techniques \cite{andrews2003business,baresi2011self,koning2009vxbpel} endeavoured to update outdated or incorrect compositions, and they allow for dynamic adaptation of the solutions based on implementation of variability constructs at the language level. For example, a composition language extending a typical WS-BPEL \cite{andrews2003business} is proposed for supporting the dynamic adaption using ECA (Event Condition Action) rules, which is utilised for guiding the operations for self-reconfiguration. This approach is difficult to manage, and error-prone. Based on and by extending the previous approaches, variability model based approaches \cite{alferez2014dynamic} are proposed to support the adaption of service composition solutions. Meanwhile, some non-EC based existing work \cite{mohanty2010web,salas2006ws,wagner2016robust,yin2010qos} focus on developing efficient methods for service re-selection, which is utilised for replacing services once the negatively contributed services are detected. On the other hand, EC-based techniques have been showing their promise in its behaviour for handling dynamic web service composition to overcome the limitations due to the following reasons: a proper amount of individuals stored could be used for retrieving an alternative composition solution in the case of failure for saving computation cost in the initial planning stage; the stored individuals could be further evolved while taking changes into account and leads to either changes on a concrete service or composition structure. These two advantages of using EC support the adaption of a dynamic environment. Existing EC-based approaches to web service composition have been studied in static scenarios, rather than dynamic ones. Although some works \cite{feng2013dynamic,liu2005dynamic} points out their approaches fit the dynamic problems since the natural features of ACO algorithm (i.e., a continuous optimisation process), there is no definition or discussion about dynamic service composition problems in their papers. Therefore, there is a lack of research in dynamic service composition. Given above discussion, it is very advisable to study the effectiveness of EC approaches in a dynamic composition context.


\subsubsection{Automated Web Service Composition Based on Preconditions and Effects}
Apart from considering the satisfactory inputs and production of outputs, some conditional constraints also determine the executability of services.  These conditional constraints lead to multiple possible paths for execution when services are composed together, since inputs and outputs are not everything required for service execution. For example, in the scenario of an online book shopping system \cite{wang2014automated}, services are composed to provide an operation for book shopping.  Users expect purchasing outcome (e.g. receipt) returned if books and customer details (e.g. title, author, customer id) are given. In this case, the users may have specific constraints. If the customer has enough money to pay for the book in full amount, then they would like to do so. Otherwise, the customer would like to pay by instalments. Therefore, the constraints on their current account balance needs to be handled during the execution of the service composition.

Most of the existing approaches to automated web service composition are approached through services represented by only inputs and outputs. However, the underlying functional knowledge base of services (i.e., in the form of preconditions and effects) is not covered \cite{paliwal2012semantics}. On the one hand, a few approaches \cite{bansal2016generalized,DBLP:journals/soca/BoustilMS14} have been explored to achieve compositions that consider precondition and effects using AI planning, since AI planning ensures both the correctness of functionality and satisfactory of constrains. Meanwhile, exhaustive methods are utilised with AI planning for tackling optimisation problems. These methods suffer hugely in terms of efficiency, scalability, and computation cost. On the other hand, EC techniques (i.e., heuristic methods) are considered to be more flexible and more efficient. Given the benefits of both AI planning and EC-based techniques, they are motivated to be collectively explored for automated web service composition based on preconditions and effects.
 
\section{Research Goals}
The overall goal of this thesis is to \emph{develop new and effective EC-based hybrid approaches for comprehensive quality-aware automated semantic web service composition}. More specifically, the focus will be on: (1) developing EC-based hybrid approaches that could explicitly support a comprehensive quality model that jointly optimises QoS and semantic matchmaking quality, (2) developing multi-objective approaches for optimising the quality criteria that involved in the comprehensive quality model, (3) developing EC-based hybrid composition techniques for dynamic service composition while handling various changes of composition environment, and  (4) developing approaches for semantic web service composition, particularly, considering precondition and effects. This research aim to develop a hybridisation of various composition techniques for effectively handling the several service composition problems discussed above. The research goal described above can be achieved by completing the following set of objectives:


\begin{enumerate}
  \item \label{Obj:1} \textbf{Develop hybrid approaches to comprehensive quality-aware automated web service composition that simultaneously optimises both QoS and semantic matchmaking quality}. Particularly, we extend existing works on QoS-aware service composition by considering jointly optimising the both quality aspects, which is proposed as a comprehensive quality model. Meanwhile, representations of the composition solutions are the key aspect of the approaches, and they must maintain all the required information for the evaluation. Therefore, we will investigate the following sub-objectives to handle this objective.
  \begin{enumerate}
    \item \emph{Propose a comprehensive quality model that addresses QoS and semantic matchmaking quality simultaneously with a desirable balance on both sides.} We aim to establish a quality model with a simple calculation and good performance for the evaluation of our proposed comprehensive quality. Meanwhile, to enable a better evaluation on our approaches, it must support most of existing benchmark datasets, e.g., Web Service Challenge 2009 (WSC09)\cite{kona2009wsc} and OWLS-TC \cite{kuster2008opossum}.
    
    \item \emph{Propose direct and indirect solution representations for comprehensive quality-aware web service composition.} Graph-based and tree-based representations are widely used for directly representing service composition solutions. Graph-based representations are capable of presenting all the semantic matchmaking relationships as edges, but hardly supporting some composition constructs (e.g. loop and choice). Tree-based representations could be more ideal for practical use, since they can present all composition constructs as inner nodes of trees. However, they could hardly maintain all the edge-related relationships supported by graphs. To take advantage of the graph-based and tree-based representations, we aim to propose a tree-like representation representation. In particular, any isomorphic copy in the traditional tree-based representations is removed and labeled with a special symbol $q$, and insert an edge to the root of the copy. Meanwhile, particular genetic operators are developed without breaking the functionality of symbol $q$.
    
    The \emph{indirect representations} do not present the final service composition solutions, they must be decoded to executable service composition. Previous studies have shown their good performances in searching optimal solution for QoS-aware web service composition \cite{da2016memetic,da2016particle}. However, the decoding process could increase the computation time. Apart from that, the indirect representation potentially increases the searching space, due to the changes of the indirect representation may result in the same solutions as discussed in the work \cite{da2016particle}. To overcome these disadvantages, it is advisable to propose more efficient indirect representations.
    
    \item \emph{Propose hybrid methods to effectively and efficiently handle the problem for comprehensive quality-aware automated web service composition.} The reasons of utilising hybrid techniques are briefly discussed in the first motivation. Herein, hybrid approaches are suggested to be developed for supporting both the proposed indirect and indirect representations, as well as the comprehensive quality model. In particularly, we aim to propose hybrid heuristics strategies to provide fast convergence of fitness value and avoid being trapped by the local optimal.
  \end{enumerate}
 
 \item \label{Obj:2} \textbf{Develop multi-objective approaches to optimise the comprehensive quality for fully automated service composition}. In practice, many quality criteria proposed in our comprehensive quality model are often conflicting in natural. Existing works \cite{chen2014partial,xiang2014qos,yin2014hybrid,liu2005dynamic,yu2013efficient,zhang2010qos} mainly concentrated on semi-automated QoS-aware web service composition. Therefore, a study needs to be carried out not only for better understanding of inherent trade-offs among different objectives (e.g., quality of semantic matchmaking and QoS are naturally considered as two conflicting objectives), but also for developing fully automated approaches by utilising cutting-edge multi-objective optimisation algorithms (e.g, NSGA-II \cite{deb2002fast}, SPEA2 \cite{zitzler2001spea2} and MOEA/D \cite{zhang2007moea}). These algorithms are needed for finding a Pareto front of evolved solutions that comprehensively cover users' real interests. Meanwhile, different representations may not perform equally well, so a study on improving the performances of different representations with different fully automated approaches also arouses researchers' interest. Apart from that, SLA consideration needs to be taken into account. It is also necessary to consider customised matchmaking levels to bring the flexibility in meeting different requirements of segmented users (e.g., platinum users, gold users and normal users). The development of this approach can be divided into the following three sub-objectives:
   \begin{enumerate}
   
    \item \label{Obj:2.1} \emph{Develop a EC-based approaches for multi-objective fully automated semantic web service composition}. 
    
    Here we develop a multi-objective optimisation approach by using effective multi-objective EC-based algorithms. (e.g, NSGA-II \cite{deb2002fast}, SPEA2 \cite{zitzler2001spea2} and MOEA/D \cite{zhang2007moea}). We will study different representations and useful modifications of multi-objective EC algorithms simultaneously to improve the effectiveness and efficiency of our service composition system. This sub-objective is also established for in-depth investigation of each quality criteria based on our proposed comprehensive quality model in Objective \ref{Obj:1}.  In particular, both quality of semantic matchmaking and QoS must be optimised independently, since they may represent conflicting interests. It would be interesting to examine different tradeoffs among the service composition solutions with respect to the different quality criterion. Apart from that, fully automated approaches are also developed to overcome the limitation (i.e., semi-automated approaches) in existing works assuming an abstract workflow is given .
   
    \item \emph{Develop hybrid approaches for multi-objective fully automated semantic web service composition}. Once we achieve the sub-objective \ref{Obj:2.1}, the effectiveness and efficiency are the next focus. The EC-based approaches should be extended by introducing some local search.  In particular, some efforts could be made for simultaneously considering the improvements on representations themselves and the combinations with a fast local search. For example, an exhaustive search for the neighbourhood of best the individual within the current population is performed with a relatively higher priority for service selection.

    \item \emph{Develop hybrid approaches for multi-objective fully automated semantic web service composition subject to constraints on SLA and customised matchmaking level}. In real world, satisfaction on given SLA constraints is required in addition to optimising QoS. Therefore, this sub-objective should be further extended to consider some additional constraints on  QoS (i.e., multilevel constraints with lower and upper bound for different individual QoS criterion \cite{yin2014hybrid}). Meanwhile, to satisfy the customised different semantic matchmaking levels (e.g., exact matchmaking level and less strict matchmaking level), extensive methods are also required to cope with the constraints on the different accepted matchmaking level.

   \end{enumerate}
   
\vspace{0.5cm}
 \item \textbf{Develop hybrid techniques to support dynamic semantic web service composition effectively}. Objectives \ref{Obj:1} and \ref{Obj:2} are proposed assuming the environment of service composition is static. In our context, composition environment refers to the registered services in the service repository, non-functional attributes advertised by service providers, and the ontology utilised for describing the resources of web services. On the one hand, existing EC-based approaches are only executed once to generate a composition plan from a given composition task, some factors could significantly impact the execution of the plan. For example, QoS values of services being composed of are fluctuating over time, service chosen at the planning stage may not available to be invoked at the runtime, or newly registered service may need to be considered for reconstructing a better plan. On the other hand, existing non-EC based approaches \cite{nasridinov2012qos,salas2006ws,wagner2016robust,yin2010qos} work on effective and efficient methods for service re-selection for repairing each service employed. Technically, their approaches do not allow the changes of service composition structure. Also, the cost of initialising a composition plan is ignored and separated from their approaches. To effectively handle the two limitations above, three studies are performed as three sub-objectives as follows:

  \begin{enumerate}
 \item \label{Obj:3.1} \emph{Develop EC-based techniques to re-optimise solution candidates for changes in QoS and Ontology.} Traditionally, initial population is created with solution candidates that are further evolved for searching optimal solutions. During the evolutionary computation process, most of service candidates are discarded except the best service candidate identified. Those discarded solution candidates may be promising, since some of them could turn to be alternative best due to the changes in services or ontology. Therefore, instead of discarding the solution candidates, we aim to propose a new and effective EC-based approach to re-optimise these maintained candidates for further use since these candidates preserve both diversity and elitism. 

 \item \emph{Develop hybrid techniques to re-optimise solution candidates for changes in QoS and Ontology.} Once the EC-based techniques to re-optimise solution candidates for changes in QoS and Ontology are achieved,  it should be further studied in developing more efficient and effective approach to handle this problem. We aim to propose an adaptive and hybrid approach to this dynamic problem. Our initial idea is to assign a higher priority to a group of services with changes and a lower priority to a group of services without changes, respectively, for considering of service selection based on a local search during the evolutionary process. The higher priority must be adaptively handled with a proper decreasing rate with respect to each service in the first group. We aim to achieve more effective and efficient performance compared to the EC-based approaches in Objective \ref{Obj:3.1}.
 
 
 \item \emph{Develop hybrid techniques for handling service failure and new service registration using updated candidates in the population.} Apart from the changes in the QoS and Ontology, occasionally existing service may fail and/or new service may be registered. For the case of service failure, some methods must be proposed to replace the un-invokable services or update the plan with new services. We aim to propose some approaches using direct representations, where we could either efficiently mutate the solutions candidates partially on un-invokable atomic services, or its involved parent composition components, or effectively re-generate whole solutions using invokable services in the service repository. For the case of new service registration, giving a priority for newly registered services should be properly considered for service selection. We could discard  a portion of the current population (e.g 50 percentage), and then replenish population based on updated services repository.
 
 \end{enumerate}
   
 \item \textbf{Develop hybrid approaches for semantic web service compositions based on preconditions and effects. (Optional)} We plan to extend most service composition approaches (i.e., satisfactory on inputs and outputs) to include preconditions and effects. These conditional constrains also necessitates the study of various of composition constructs for automated semantic web service composition, e.g., loop and choice. Therefore, three sub-objectives have been proposed as our targets as follow.
 \begin{enumerate}
 
  \item \emph{Develop EC-based techniques for semantic web service composition based on preconditions and effects}. In the problem stated above, inputs and outputs are everything of web services for handling some web services. An initial task is required to be completed. That is, service composition problem is re-modelled by further considering the preconditions and effects. In particular, we need to establish a general matchmaking mechanism of satisfaction on preconditions and effects. Based on the mechanism, sequence and parallel composition constructs are automatically constructed. We aim to develop EC-based approaches to effectively handle this problem. In particular, new representations are needed to be proposed for coping with the newly modelled problem.

  \item \emph{Develop hybrid techniques for semantic web service composition based on preconditions and effects}. Once the EC-based techniques to semantic web service composition based on preconditions and effects are proposed. more effective and efficient techniques shall be developed. In particular, we aim to create hybrid techniques that utilise a hybridisation of various techniques for improving the performances of EC-based techniques for semantic web service composition based on preconditions and effects.
    
   \item \emph{Develop EC-based techniques for semantic web service composition based on preconditions and effects for supporting loops and choice}. We initially extend the matchmaking mechanism of satisfaction on preconditions and effects to support loops and choice composition constructs. To extensively cope with these two constructs, new and effective representations must be studied. Apart from that,  EC-based approaches can be developed to effectively solve this problem.

 
 \end{enumerate}
 
\end{enumerate}

\section{Published Papers}

During the initial stage of this research, the preliminary work was carried out on establishing the comprehensive quality model.  Afterwards, some studies on the direct and indirect representations are completed for one part of Objective \ref{Obj:1}, but the earlier works focus on static web service composition using single-objective optimisation technique. The following are the publications made from the preliminary studies:

\begin{itemize}
 \item WANG, C., MA, H., CHEN, A., AND HARTMANN, S. ''Comprehensive Quality-Aware Automated Semantic Web Service Composition``. \textit{AI 2017: Advances in Artificial Intelligence: 30th Australasian Joint Conference}. 2017, pp. 195-207.
 \item Wang, C., Ma, H., Chen, A., Hartmann, S.: ''GP-Based Approach to Comprehensive quality-aware automated semantic web service composition``. In: SEAL2017: International Conference on Simulated Evolution and Learning(To appear)
\end{itemize}


\section{Organisation of Proposal}The remainder of the proposal is organised as follows: Chapter \ref{C:review} provides a fundamental definition of the web service composition problem and performs a literature review covering a range of works in this field; Chapter \ref{C:preliminary} discusses the preliminary work explores direct and indirect representations for comprehensive quality-aware semantic web service composition using a hybridisation of AI planning techniques and EC-based techniques; Chapter \ref{C:plan} presents a plan detailing this project's intended contributions, a project timeline, and a thesis outline.
